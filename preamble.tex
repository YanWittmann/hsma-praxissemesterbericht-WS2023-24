% pakete
\usepackage{ifthen}

% Deutsch
\usepackage[german]{babel} % deutsch und deutsche Rechtschreibung
\usepackage[backend=biber, style=numeric, sorting=none]{biblatex} % Literaturverzeichnis (sortiert nach Reihenfolge des Auftretens)
\addbibresource{praksem.bib}
\usepackage[utf8]{inputenc} % Unicode Text
\usepackage[T1]{fontenc} % Umlaute und deutsches Trennen
\usepackage{textcomp} % Euro
% statt immer Ab\-schluss\-ar\-beit zu schreiben
% einfach hier sammeln mit -. 
\hyphenation{Ab-schluss-ar-beit}
% Vorsicht bei Umlauten und Bindestrichen
\hyphenation{Ver-st\"ar-ker-aus-gang}
 % eigene Hyphenations, die für das Dokument gelten
\usepackage{amssymb} % Symbole
\usepackage{emptypage} % Wirklich leer bei leeren Seiten

%% Fonts, je ein kompletter Satz an Optionen

% Times New Roman, gewohnter Font, ok tt und serifenlos
%\usepackage{mathptmx} 
%\usepackage[scaled=.95]{helvet}
%\usepackage{courier}

% Palatino mit guten Fonts für tt und serifenlos
\usepackage{mathpazo} % Palatino, mal was anderes
\usepackage[scaled=.95]{helvet}
\usepackage{courier}

% New Century Schoolbook sieht auch nett aus (macht auch tt und serifenlos)
%\usepackage{newcent}

% Oder default serifenlos mit Helvetica 
% ich kann es nicht mehr sehen ...
%\renewcommand{\familydefault}{\sfdefault}

% ein bisschen eine bessere Verteilung der Buchstaben...
\usepackage{microtype}

% Bilder und Listings
\usepackage{graphicx} % wir wollen Bilder einfügen
\usepackage{subfig} % Teilbilder
\usepackage{wrapfig} % vielleicht doch besser vermeiden
\usepackage{listings} % schöne Quellcode-Listings
% ein paar Einstellungen für akzeptable Listings
\lstset{basicstyle=\ttfamily, columns=[l]flexible, mathescape=true, showstringspaces=false, numbers=left, numberstyle=\tiny}
\lstset{language=python} % und nur schöne Programmiersprachen ;-)
% und eine eigene Umgebung für Listings
\usepackage{float}
\newfloat{listing}{htbp}{scl}[chapter]
\floatname{listing}{Listing}

% Seitenlayout
\newcommand{\seitenseitenabstand}{22mm} % 30mm
\usepackage[paper=a4paper,left=\seitenseitenabstand,right=\seitenseitenabstand,height=23cm]{geometry}
\usepackage{setspace}
% \linespread{1.15}
\linespread{1.1}
\setlength{\parskip}{0.5em}
\setlength{\parindent}{0em} % im Deutschen Einrückung nicht üblich, leider

% Seitenmarkierungen 
\newcommand{\phv}{\fontfamily{phv}\fontseries{m}\fontsize{9}{11}\selectfont}
\usepackage{fancyhdr} % Schickere Header und Footer
\pagestyle{fancy}
\renewcommand{\chaptermark}[1]{\markboth{#1}{}}
%\fancyhead[L]{\phv \leftmark}
\fancyhead[RH,LO]{\phv \nouppercase{\leftmark}}
\fancyhead[LH,RO]{\phv \thepage}
% Unten besser auf alles Verzichten
%\fancyfoot[L]{\textsf{\small \kurztitel}}
\fancyfoot[C]{\ } % keine Seitenzahl unten
%\fancyfoot[R]{\textsf{\small Technische Informatik}}

% Include subsubsections in the Table of Contents
% -1 for part
%  0 for chapter (only in report and book document classes)
%  1 for section
%  2 for subsection
%  3 for subsubsection
%  4 for paragraph
%  5 for subparagraph
\setcounter{tocdepth}{3}

% Theorem-Umgebungen
\newtheorem{definition}{Definition}[chapter]
\newtheorem{satz}{Satz}[chapter]
\newtheorem{lemma}[satz]{Lemma} % gleicher Zähler wie Satz
\newtheorem{theorem}{Theorem}[chapter]
\newenvironment{beweis}[1][Beweis]{\begin{trivlist}
                                       \item[\hskip \labelsep {\textit{#1 }}]}{
\end{trivlist}}
\newcommand{\qed}{\hfill \ensuremath{\square}}

%% Quellen
% Eine Alternative wäre Quellen in Literatur und Online-Quellen
% zu teilen
% \usepackage{bibtopic} 

% Hochschule Logo, noch nicht perfekt
\usepackage{hsmalogo}

% Spezialpakete
\usepackage{epigraph}
\setlength{\epigraphrule}{0pt} % kein Trennstrich

% damit wir nicht so viel tippen müssen, nur für Demo 
% \usepackage{blindtext}

% ifthen für sperrvermerk
\newif\ifsperrvermerk

% klickbare links im inhaltsverzeichnis
\usepackage{hyperref}
\hypersetup{
    colorlinks,
    citecolor=black,
    filecolor=black,
    linkcolor=black,
    urlcolor=black
}

% text shortcut variablen
\newcommand{{\metaeffekt}}{\{met\ae ffekt\}}
\newcommand{\bitkom}{bitkom}
\newcommand{\aeclientZEZESE}{Thales}
\newcommand{\headerand}{und\ }
% weekdays
\newcommand{\weekdayMondayLong}{Mo} % Mo, Montag
\newcommand{\weekdayTuesdayLong}{Di} % Di, Dienstag
\newcommand{\weekdayWednesdayLong}{Mi} % Mi, Mittwoch
\newcommand{\weekdayThursdayLong}{Do} % Do, Donnerstag
\newcommand{\weekdayFridayLong}{Fr} % Fr, Freitag
\newcommand{\weekdaySaturdayLong}{Sa} % Sa, Samstag
\newcommand{\weekdaySundayLong}{So} % So, Sonntag
\newcommand{\weekdayMondayShort}{Mo}
\newcommand{\weekdayTuesdayShort}{Di}
\newcommand{\weekdayWednesdayShort}{Mi}
\newcommand{\weekdayThursdayShort}{Do}
\newcommand{\weekdayFridayShort}{Fr}
\newcommand{\weekdaySaturdayShort}{Sa}
\newcommand{\weekdaySundayShort}{So}

% custom commands
\newcommand{\lweekdaymarginpar}[1]{%
    \marginpar{\raisebox{-1.6em}{\underline{#1}}}
}
\newcommand{\sweekdaymarginpar}[1]{%
    \marginpar{\raisebox{-1.92em}{\underline{#1}}}
}
\newcommand{\qt}[1]{„#1“}

\definecolor{codegray}{gray}{0.9}
\newcommand{\code}[1]{\colorbox{codegray}{\texttt{\detokenize{#1}}}}
\newcommand{\codendt}[1]{\colorbox{codegray}{\texttt{#1}}}

% custom environments
\newenvironment{smitemize}
{ \begin{itemize}
      \setlength{\itemsep}{0em}
      \setlength{\topsep}{0em}
      \setlength{\partopsep}{0em} }
      {
\end{itemize} }

% code listings
\input{latex-listings-powershell/src/latex-listings-powershell}
\definecolor{lst-gray}{rgb}{0.98,0.98,0.98}
\definecolor{lst-blue}{RGB}{40,0.0,255}
\definecolor{lst-green}{RGB}{65,128,95}
\definecolor{lst-red}{RGB}{200,0,85}
\lstset{
    commentstyle=\color{lst-green},
    basicstyle=\small\ttfamily,
    backgroundcolor=\color{lst-gray},
    breaklines=true,
    captionpos=b,
    columns=fixed,
    extendedchars=true,
    frame=single,
    framesep=2pt,
    keepspaces=true,
    keywordstyle=\color{lst-blue},
    language={PowerShell},
    numbers=left,
    numberstyle=\small\ttfamily,
    showstringspaces=false,
    stringstyle=\color{lst-red},
    tabsize=2,
}
