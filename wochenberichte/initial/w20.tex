\section{Woche 20 - BSI-Meeting \headerand Integration von Generation 3} \label{sec:bericht-wo-20-initial}

% 2024-01-29 bis 2024-02-02

\lweekdaymarginpar{\weekdayMondayLong}

Montag war ein interessanter Tag, denn heute hatten wir ein Meeting mit Repräsentanten vom BSI bezüglich des CSAF-Standards und unseres CVSS-Rechners.
Beteiligt war vor allem Thomas Schmidt\footnote{\url{https://www.it-meets-industry.de/de/referent-thomas-schmidt}}, welcher bereits Leiter des CSAF-Workshops in München Ende letzten Jahres war, und Herr Von Samson.

Hierfür habe ich den Morgen eine Demo unseres Toolings vorbereitet, welche wir dann am Nachmittag zusammen mit vorbereiteten Fragen durchgesprochen haben.
Insgesamt war es ein sehr konstruktives Meeting, aus dem wir viele weitere Tasks ableiten konnten, vor allem zu Ansätzen zur Integration von CSAF in unseren Workflow.
Thomas Schmidt hat die folgenden Tage noch einige Issues auf unserem Repository des CVSS-Rechners auf GitHub erstellt, die ich Ende der Woche angehen würde.

\sweekdaymarginpar{\weekdayTuesdayShort, \weekdayWednesdayShort, \weekdayThursdayShort}

Die restliche Woche habe ich mich mit der Integration von Generation 3 unseres Monitorings bei den Kunden beschäftigt.
Einer unserer Problempunkte war es bisher, dass teilweise Kunden noch auf Generation 1 standen, diese wollten wir nun alle zusammen einheitlich auf Generation 3 heben.
Insgesamt mussten drei Projekte fürs Erste ge-upgraded werden.

Dafür habe ich von meinem Chef nacheinander Zugriff auf je einen Kundenlaptop bekommen, bei denen je ein- oder mehrere Git-Projekte mit unserem Tooling, welche in deren Pipelines verwendet werden.
Der Prozess war immer der gleiche:
die vorhandenen Versionen und Konfigurationen mussten alle der Reihe nach aktualisiert werden, sodass sie dem neuen Format entsprechen.
Hier war der Migrationsguide, den ich zuvor (für genau diese Situation) verfasst hatte, viel wert.

Es gab natürlich unzählige Komplikationen und genauso viele Änderungen, die noch am Code gemacht werden mussten, die hier aufgelistet den Rahmen sprengen würden.
Mittwochs bin ich auch etwas länger geblieben, aber Donnerstagnachmittag waren alle Projekte aktualisiert und liefen durch.

\sweekdaymarginpar{\weekdayFridayLong}

Freitag bin ich die Issues\footnote{\url{https://github.com/org-metaeffekt/metaeffekt-universal-cvss-calculator/issues?q=is\%3Aissue+is\%3Aclosed}} von Thomas durchgegangen und habe die ersten davon bearbeitet.
Hier gab es nichts, das Probleme bereitet hätte (bis auf Issue \#2, hier hat sich das FIRST\footnote{\url{https://www.first.org/cvss}} bis Praktikumsende nicht zurückgemeldet).

Zudem gab es nach unserem Weekly ein Meeting mit dem Kunden, der die Assessment-Workshops plant bezüglich der Generation 3, der Änderungen und es bei ihnen final in ihre Pipeline zu integrieren.
