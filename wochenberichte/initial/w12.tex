\section{Woche 12 - Integration des Datenmodells in PDF-Report} \label{sec:bericht-wo-12-initial}

% 2023-11-20 bis 2023-11-24

\lweekdaymarginpar{\weekdayMondayLong}

Am Montag nahm ich mir eine kurze Auszeit von der Report-Migration und wendete mich einem anderen Aspekt des Refactorings zu:
Dem Tracking der genauen Matching-Konfigurationen von Schwachstellen, die aus verschiedenen Quellen identifiziert wurden.
Bisher beschränkt sich unser System auf das Tracking der \qt{CPE}-Informationen, allerdings ohne die dazugehörigen Versionsbereiche und auch nur bei diesen, nicht anderen Quellen.
In der Vergangenheit war das ausreichend, da nur die NVD als Datenquelle diente, aber mittlerweile kommen auch GitHub, Microsoft und andere hinzu.

Deshalb verbrachte ich den Tag damit, diese Daten in den verschiedenen Anreicherungsschritten in das von vor zwei Wochen implementierte Tracking-System einzupflegen.
Diese Informationen konnte ich noch am Montag in das VAD integrieren.
Bei dieser Gelegenheit wurde mir erneut bewusst, wie viel einfacher Anpassungen am VAD im Vergleich zum PDF-Report sind.

\sweekdaymarginpar{\weekdayTuesdayShort\ - \weekdayThursdayShort}

In der Mitte der Woche konnte ich mich wieder vollständig auf die Integration des Modells in den Report konzentrieren.
Dieser Prozess ist stets gleich: Für jedes der etwa 20 Velocity-Templates in unserem Prozess überprüfe ich den alten Datenzugriff und suche ich im neuen Modell nach einem entsprechenden Zugriff oder implementiere neue Methoden.
Diese Änderungen mache ich entweder in den Adapterklassen, die als Schnittstelle zwischen Report und Modell dienen, oder direkt im Modell selbst.
In letzterem Fall muss ich die Änderungen sowohl in Core als auch in Artifact Analysis vornehmen.

Eine zusätzliche Herausforderung war, dass ich immer wieder auf Templates stieß, die ich zuvor noch nie gesehen habe und die ich erst verstehen musste.
Um nicht bei jedem Test den Dita-Renderingprozess starten zu müssen, nutze ich das Tool \qt{OxygenXML}, das eine Live-Preview ohne Stilelemente erlaubt.
Dennoch dauert es immer eine Weile, bis ich meinen Testdatensatz angepasst habe, um all diese Dokumente richtig testen zu können.

\sweekdaymarginpar{\weekdayFridayLong}

Freitag begann mit einer unerwarteten Anfrage meines Chefs:
Er fragte mich, ob ich Interesse hätte, alleine an einem Workshop zum CSAF-Standard\footnote{\url{https://web.archive.org/web/20240121120954/https://www.allianz-fuer-cybersicherheit.de/Webs/ACS/DE/Netzwerk-Formate/Veranstaltungen-und-Austausch/CSAFversum/CSAFversum_node.html}} (Common Security Advisory Framework) teilzunehmen, der vom Bundesamt für Sicherheit in der Informationstechnik (BSI) in München organisiert wird.
Nach einer kurzen Recherche zu CSAF fand ich heraus, dass es sich dabei um ein Schema handelt, ähnlich wie OSV\footnote{\url{https://osv.dev}}, das es Herstellern ermöglicht, Security Advisories und Informationen zu Schwachstellen in ihren Produkten zu veröffentlichen.

Der Hintergrund dieser Anfrage war, dass mein Chef plant, dass ich diesen Standard irgendwann in unser System integriere.
Da ich sowohl die Integration von CSAF als sinnvoll erachte, als auch persönlich mich auf eine Reise nach München freuen würde, stimmte ich dem Vorschlag zunächst einmal zu.
Den Rest des Tages sollte ich aufgrund der Abwesenheit eines anderen Kollegen erneut Korrelationsdaten für ein dringendes Inventar anlegen.
