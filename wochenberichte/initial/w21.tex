\section{Woche 21 - Praktikumsbetreuung \headerand Abschluss des Praxissemesters} \label{sec:bericht-wo-21-initial}

% 2024-02-05 bis 2024-02-09

\lweekdaymarginpar{\weekdayMondayLong}

Die letzte Woche für mich im Praktikum bedeutete den Start eines anderen:
Die {\metaeffekt} hat einen BOGY-Praktikanten angenommen, der diese Woche einmal bei uns reinschnuppern durfte.
Ich war zu einem großen Teil für seine Betreuung verantwortlich, nebenbei habe ich noch weitere Probleme mit der Integration von Generation 3 bei den Kunden gelöst, die seit dem Ende der letzten Woche aufgetreten sind.

Mit dem Praktikanten habe ich erst einmal ein wenig programmieren und Konzepte in Java geübt, denn er sollte diese Woche hauptsächlich mit einem der Testing-Frameworks eines anderen Kollegen einige Testfälle für unseren Software-Scanner und Komponenten-Extraktor schreiben.

\sweekdaymarginpar{\weekdayTuesdayLong}

Er schien Konzepte recht schnell aufzunehmen und so konnte er bereits Dienstag mit den eigentlichen Testfällen beginnen.
Natürlich hat er ab und zu auch andere Probleme lösen und Dinge tun dürfen.

Ansonsten ist der Dienstag genau wie der Montag verlaufen - ich durfte wieder Probleme mit Generation 3 beheben.
Allerdings schien dieses Mal ein größeres Problem dabei zu sein:
Unser Prozess ist auf einmal von durchschnittlich 10 Minuten Laufzeit auf Generation 2 zu über zwei Stunden auf der dritten gesprungen.
Das ist natürlich nicht akzeptabel, und ich habe mich gleich an die Recherche gemacht.
Gewundert hat es mich allerdings schon, denn ich hatte bei meinen Tests bei dem neuen Datenmodell speed-ups von bis zu 200\% oder 300\% in optimalen Fällen gemessen.

Glücklicherweise konnte ich das Problem mit nur einer Zeile Code-Änderungen beheben:
Ich habe eine Liste zu einem Set umgewandelt, eine Übersehene nicht vollständig durchgeführte Optimierung meinerseits, die ich scheinbar vergessen hatte, fertig durchzuführen.
Dadurch sind wir bei nur noch zweieinhalb Minuten gelandet, was eher dem entsprach, was ich erwartet hatte.

\sweekdaymarginpar{\weekdayWednesdayShort, \weekdayThursdayShort}

Mittwoch und Donnerstag habe ich noch live in Kooperation mit einem Teams-Meeting die letzten Probleme beheben können, passend zum Ende meines Praktikums war also Generation 3 vollständig integriert.
Natürlich konnte ich auch noch mit dem Praktikanten einige Dinge tun und üben.

\sweekdaymarginpar{\weekdayFridayLong}

An meinem 100sten und damit letzten Tag des Praktikums habe ich zur Feier Kuchen mitgebracht und beim Weekly zusammen mit den restlichen süßen Stückchen verteilt.
Nach dem Weekly konnten wir auch noch alle zusammen aus dem Büro Currywurst essen gehen, es war also ein guter Abschluss zum Praktikum.

Natürlich wurde auch gearbeitet:
Ich habe den Tag dazu genutzt, einem weiteren Issue meines Chefs mit einem unserer Reports zu beheben.
Hier ging es um den periodischen Security-Advisory Report mit Bezug auf die Schwachstellsituation in einem Projekt.
Durch die neue Art zu filtern in Generation 3 hat er bemerkt, dass er nun gerne genauer kontrollieren können möchte, welche Daten in diesem Report angezeigt werden.
Hier habe ich also neue Parameter und alternative Report-Templates angelegt, die diesen Wünschen entsprechen.

Auch für den BOGY-Praktikanten war es der letzte Tag und so sind wir beide mit einem erfolgreich absolvierten Praktikum ins Wochenende gegangen.

Ich habe noch weitere 12 Tage bei dem Unternehmen mit meinen 35 Stunden in den folgenden Wochen gearbeitet und einen weiteren Vertrag für weitere Zusammenarbeit abgeschlossen.
Meine Reise in dieser Welt ist also noch lange nicht vorbei.
