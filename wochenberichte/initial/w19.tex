\section{Woche 19 - Assessment-Policy \headerand Generation 3 Fertigstellung} \label{sec:bericht-wo-19-initial}

% 2024-01-22 bis 2024-01-26

\lweekdaymarginpar{\weekdayMondayShort\ - \weekdayWednesdayShort}

Einer der Kunden der \metaeffekt, den wir mit dem Schwachstell-Monitoring betreuen, geht nun in eine Projektphase, in der er Einschätzungen für Schwachstellen und Maßnahmen für deren Mitigierung vergeben möchte.
Dazu planen sie interne Workshops verpflichtend für mehrere Abteilungen anzubieten, wobei wir sie natürlich unterstützen wollen.

Darum schrieb ich die erste Hälfte der Woche an einer Assessment-Policy als Ausgangspunkt für ihre Überlegungen.
In diesem 6 Seiten langen Dokument berühre ich die folgenden Punkte:
Was ist der allgemeine Prozess, Vulnerability Monitoring durchzuführen?
Wie kann mit CVSS ein kontextualisiertes re-Scoring stattfinden?
Was sind die einzelnen CVSS Metriken der verschiedenen Versionen, die (Kontext-weit) verändert werden können/sollten?
Wie vergibt man einen Status, Risiken und Maßnahmen?
Wie werden Schwachstellen am besten priorisiert?

\sweekdaymarginpar{\weekdayThursdayLong}

Zusammen mit einer Kollegin habe ich dieses Dokument Donnerstag noch einmal überarbeitet und an meinen Chef übergeben, der es auch noch einmal überarbeitet und dann mit unseren Kunden durchgesprochen hat.

\sweekdaymarginpar{\weekdayFridayLong}

Passend dazu war Freitag dann der Tag, an dem der große Pull Request mit der Generation 3 unseres Vulnerability Monitorings ge-merged wurde.
Uns ist es besonders wichtig gewesen, die Änderungen aus dieser Version noch vor den Workshops des Kunden anzuwenden, denn so müssen nicht gleich einen Monat später neue Kurse für die neue Version angeboten werden.
Diese Änderungen zum Kunden zu bringen würde aber erst nächste Woche geschehen.

Den restlichen Tag habe ich noch einige Änderungen aufgrund von Feedback, das wir erhalten haben, an unserem CVSS-Rechner gemacht.
Freitag war außerdem ein alter Kollege zu Besuch, der vor einiger Zeit aufgehört hatte, hier zu arbeiten.
