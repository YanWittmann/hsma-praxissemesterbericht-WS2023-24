\section{Woche 5 - Einarbeitung eines neuen Kollegen} \label{sec:bericht-wo-5}

% Woche 5 (2023-10-04 bis 2023-10-06)

\lweekdaymarginpar{Mittwoch}

Da Dienstag ein Feiertag war und Montag die {\metaeffekt} einen Brückentag hatte, war der Mittwoch der erste Arbeitstag für mich und meinen neuen Kollegen Nils.
Die Einarbeitung von Nils war meine Hauptaufgabe für diese Woche.
Sein Aufgabenbereich wird es sein, die Korrelationsdaten, also die Mappings zwischen unserer internen Darstellung von Software-Produkten und den Produkten in diversen externen Datenbanken, zu pflegen.
Passend dazu hat mein Chef uns einen neuen Datensatz gegeben, ein Inventar an Komponenten, den man einpflegen musste.
Diese Daten sollten bis zum Ende der nächsten Woche fertig sein.
Den Tag habe ich dann also dazu verwendet, mit ihm die Grundlagen unseres Systems und die Benutzung der von mir geschriebenen internen Tools, durchzugehen.

\sweekdaymarginpar{Donnerstag}

Nils hatte beschlossen, in seiner ersten Woche mehr als seine 10 Stunden zu arbeiten, darum konnte ich ihn am Donnerstag gleich wieder im Büro begrüßen.
Wir haben uns an die Software-Inventare gemacht, bei denen ich die etwas komplizierteren Fälle übernommen habe und Nils mit einigen einsteigerfreundlicheren versorgt habe.
Die Herausforderung an dieser Arbeit ist nicht nur die Methodik, sondern auch das gesammelte Wissen, das man über alle Software-Ökosysteme, Betriebssysteme und Software-Pakete haben muss, um die richtigen Entscheidungen treffen zu können.
Genau dieses Wissen ist, was Nils noch fehlt, dennoch hat er sich für einen ersten Tag sehr gut geschlagen.

\sweekdaymarginpar{Freitag}

Freitag startete ich mit den Inventaren zunächst ohne Nils, der erst zum Weekly dazu kommen konnte.
Dieses Mal habe ich habe Nils etwas interessantere Fälle gegeben und natürlich bei Fragen geholfen.
Dies war das Ende der ersten Woche für Nils, der allerdings noch etwas länger dort verblieben ist, da er erst etwas später dazugekommen ist.
