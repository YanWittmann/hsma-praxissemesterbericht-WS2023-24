\section{Woche 2 - Vertiefung in CVSS 4.0 \headerand Korrelationsdaten} \label{sec:bericht-wo-2}

% Woche 2 (2023-09-11 bis 2023-09-15)

\lweekdaymarginpar{Montag}

Die zweite Woche meines Praktikums wollte und sollte ich mich mit der genaueren Funktionsweise von CVSS 4.0 auseinandersetzen.
Ich verbrachte also den Montag damit, die Spezifikation\footnote{\url{htthttps://www.first.org/cvss/v4.0/specification-document}},
die Entwicklungsgeschichte\footnote{\url{https://www.first.org/cvss/v4.0/user-guide\#New-Scoring-System-Development}}
und den Code tiefergehender anzusehen, um zu verstehen, wie die Berechnung des Scores funktioniert.

Ich hatte bereits während der Implementierung einen guten Überblick über den ersten Schritt der Berechnung, den mit den MacroVektoren, erhalten.
Die Unklarheiten lagen eher in dem zweiten Schritt: Der Interpolation zwischen den MacroVektor-Blasen.
Ich konnte selbst nach einem ganzen Tag an Recherche keine zufriedenstellende Erklärung finden, wie die Berechnung in diesem Schritt funktioniert.
Durch meine Recherche konnte ich immerhin weitere Randfall-Tests anlegen und damit zwei Fehler in meiner Implementierung finden.

\sweekdaymarginpar{Dienstag}

Eigentlich wollte ich am Dienstag weiter an meinem Verständnis von CVSS 4.0 arbeiten, aber es gab etwas Wichtigeres:
Da ein Kollege noch im Urlaub war, musste ich einen Teil seiner Arbeit übernehmen.
Es geht um das Pflegen von Korrelationsdaten, die dazu dienen, Software-Komponenten automatisch zu Produkten in fremden Datenbanken (wie den CPE in der NVD\footnote{\url{https://nvd.nist.gov/products/cpe}}) zuzuordnen.
Da dies früher einmal meine Aufgabe im Unternehmen war, habe ich dafür in der Vergangenheit ein Tool geschrieben, das ich kurz vor meinem Praktikum mit viel Feedback von und für meinen Mitarbeiter als Web-UI neu aufgesetzt hatte.
Das gab mir also die Chance, dieses Tool endlich einmal selbst zu verwenden und an einigen Stellen den Workflow zu verbessern.

\sweekdaymarginpar{Mittwoch}

Leider war ich Dienstag nicht mit den zu verarbeitenden Korrelationsdaten fertig geworden, darum habe ich noch den halben Mittwoch damit verbracht.
Was mich positiv überraschte war, wie gewissenhaft mein Kollege diese Arbeit durchgeführt hatte, seitdem er diese von mir übernommen hatte.
Ich musste nur bei wenigen Einträgen die Matching-Informationen überarbeiten, bei den meisten hatten diese bereits gestimmt und ich musste sie nur noch abnicken.

Den restlichen zwei Stunden habe ich dann noch mit meinem Chef über ein interessantes Thema reden können: Dem Abbilden von Vulnerability Chaining in unseren Systemen.
Also: was passiert, wenn man eine kritische Software-Schwachstelle, die man zuvor als \qt{Nicht Ausnutzbar} abgestempelt hatte, über eine andere, für sich alleine stehende harmlose, Schwachstelle ausnutzen kann?
Dieses Thema müssen wir auf Kundenwunsch in den nächsten Monaten abbilden können, aber eine hohe Priorität bekommt es noch nicht.

\sweekdaymarginpar{Donnerstag}

Donnerstag war mein Chef nicht im Büro, da er ein Treffen bei einem Kunden hatte.
Es war auch ein Tag, an dem ich sehr viel Code geschrieben habe, denn ich musste meine Planung für die Woche etwas umpriorisieren:
Ende der Woche musste die Implementierung und Integration von CVSS 4.0 in unser System fertiggestellt sein, darum beschloss ich, mich erst einmal darum zu kümmern und danach erst weiter mein Verständnis aufzubauen.
Die folgenden Punkte mussten bearbeitet werden:

\begin{smitemize}
    \item das Parsing der Vektoren aus verschiedenen Datenquellen/Formate (intern, extern)
    \item das korrekte Verarbeiten, wenn es um die Berechnung von Scores, das Modifizieren von Vektoren, oder das Ablegen in unseren Software-Inventaren geht
    \item das Anzeigen der Ergebnisse in unseren HTML- und PDF-Reports
\end{smitemize}

Da die externen Datenquellen ja noch keine Vektoren für CVSS 4.0 herausgeben, musste ich einige Annahmen über deren Formate machen, die ich bei tatsächlicher Veröffentlichung dann noch anpassen werde.

Bei diesen Arbeiten sind mir erst einige merkwürdige Muster in der Implementierung für die Berechnung der Scores aufgefallen, die ich mehr oder weniger aus der Referenz-Implementierung übernommen hatte.
Ich habe also fast noch einmal die gesamte Implementierung neu geschrieben, indem ich duplizierten Code in eigene Methoden und Klassen extrahiert und die Berechnungen wesentlich eleganter und näher an den dahinterliegenden mathematischen Modellen in meinem Programm abgebildet habe.

Am Ende des Tages hatte ich Pull Requests für die drei Projekte, in denen diese Berechnungen stattfinden, fertiggestellt.

\sweekdaymarginpar{Freitag}

Da ich die Integration bereits Donnerstag fertiggestellt hatte, habe ich mich wieder an das Verständnis über den CVSS 4.0 Standard gesetzt.
Um besser verstehen zu können, wie die MacroVektor Interpolation funktioniert, habe ich mir drei Beispiele herausgesucht und die Berechnungen aufgrund der Spezifikation und meiner Implementierung manuell mehrfach auf verschiedene Arten durchgeführt.
Das hat tatsächlich sehr geholfen und kurz vor dem Weekly hatte ich dann ein ganz gutes Verständnis, wie die Berechnung funktioniert.
Dies habe ich in einem Dokument für späteren Zugriff festgehalten.

Das Weekly war stark gefüllt, nicht nur ich hatte viele Neuigkeiten über die Funktionsweise von CVSS 4.0 zu berichten, sondern zwei meiner Kollegen hatten diese Woche auch in ihren Teilprojekten größere Durchbrüche, die sie sogar live demonstriert und erklärt haben.
Damit wurde das Weekly von einer Stunde auf fast zweieinhalb Stunden verlängert und der Tag war schnell herum.
Ich beendete meine zweite Woche mit einem guten Gefühl, diese Woche viel geschafft zu haben.
