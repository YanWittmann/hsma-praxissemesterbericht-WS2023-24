\section{Woche 2 - Vertiefung in CVSS 4.0 \headerand Korrelationsdaten} \label{sec:bericht-wo-2}

% Woche 2 (2023-09-11 bis 2023-09-15)

\lweekdaymarginpar{Montag}

Ich hatte nun CVSS 4.0 bei uns implementiert, aber wirklich verstanden hatte ich Theorie dahinter noch nicht.
Ich verbrachte also den Montag damit, die Spezifikation\footnote{\url{htthttps://www.first.org/cvss/v4.0/specification-document}},
die Entwicklungsgeschichte\footnote{\url{https://www.first.org/cvss/v4.0/user-guide\#New-Scoring-System-Development}}
und den Code tiefergehender anzusehen, um zu verstehen, wie die Berechnung des Scores funktioniert.
Über den ersten Schritt der Berechnung mit den MacroVektoren hatte ich ein bereits ein gutes Verständnis, die Unklarheiten lagen eher in dem zweiten Schritt mit der Interpolation zwischen den einzelnen MacroVektoren.
Ich konnte selbst nach einem ganzen Tag an Recherche keine zufriedenstellende Erklärung finden, wie die Berechnung in diesem Schritt funktioniert.
Immerhin konnte ich durch weitere Randfall-Tests zwei Fehler in meiner Implementierung finden und beheben.

\sweekdaymarginpar{Di, Mi}

Eigentlich hatte ich geplant, mich Dienstag weiter mit CVSS 4.0 zu beschäftigen, doch es kam anders:
In Abwesenheit eines Kollegen übernahm ich die Pflege von sog. \qt{Korrelationsdaten}, ein von uns gepflegter Datensatz, der Software-Komponenten automatisch Produkten in externen Datenbanken wie den CPEs der NVD\footnote{\url{https://nvd.nist.gov/products/cpe}} zuordnen kann.
Dazu konnte ich ein von mir entwickeltes Tool nutzen, das ich kurz vor meinem Praktikum, basierend auf umfangreichem Feedback des Kollegen, als Web-UI über Spring Boot neu aufgesetzt hatte.
Diese Gelegenheit, das Tool selbst einmal anwenden zu müssen, den Workflow darin zu optimieren.
Positiv überraschend war, wie sorgfältig mein Kollege diese Arbeit durchgeführt hatte.
Nur wenige Einträge mussten angepasst werden; die meisten waren bereits korrekt und bedurften nur meiner Bestätigung.

In den verbleibenden zwei Stunden diskutierte ich mit meinem Chef über das Abbilden von \qt{Vulnerability Chaining} in unseren Systemen – ein Thema, das wir auf Kundenwunsch in den kommenden Monaten angehen müssen, auch wenn es noch keine hohe Priorität hat.
Wir haben dieses Thema nicht mehr in meinem Praktikum aufgenommen.

\sweekdaymarginpar{Donnerstag}

Donnerstag war mein Chef durch einen Kundenbesuch nicht im Büro.
Da die Implementierung und Integration von CVSS 4.0 bis Wochenende abgeschlossen sein musste, musste ich mich mit den folgenden verbleibenden Aufgaben intensiv beschäftigen:

\begin{smitemize}
    \item Parsing der Vektoren aus verschiedenen Quellen/Formaten
    \item Korrektes Verarbeiten der Berechnung von Scores und Modifizieren von Vektoren
    \item Anzeigen der Ergebnisse in unseren HTML- und PDF-Reports
\end{smitemize}

Da keine externe Datenquelle bisher CVSS 4.0 Vektoren bereitstellt, basierten einige meiner Annahmen über deren Formate auf Vermutungen aufgrund von bisherigen Formaten, die später eventuell noch angepasst werden müssen.
Da ich nun noch einmal Zeit hatte, über meine Implementierung zu sehen, konnte ich viele Muster, die ich aus der Referenzimplementierung übernommen hatte, durch viele Refactoring-Operationen Code deduplizieren und die Berechnungen eleganter und objektorientierter gestalten.
Am Ende des Tages stellte ich Pull Requests für die drei betroffenen Code-Projekte fertig.

\sweekdaymarginpar{Freitag}

Nachdem die Integration glücklicherweise bereits am Donnerstag abgeschlossen war, widmete ich mich erneut dem Verständnis von CVSS 4.0.
Unter anderem berechnete ich manuell mehrfach auf unterschiedliche Weisen die drei Beispiele der MacroVektor-Interpolation aus der Spezifikation, was mein Verständnis erheblich verbesserte.
Kurz vor dem Weekly hatte ich ein besseres Verständnis der Berechnungsmethode, das ich für späteren Zugriff dokumentierte.

Das Weekly war voller Informationen, nicht nur von mir, sondern auch von Kollegen, die Fortschritte in ihren Projekten vorstellen konnten.
Das Meeting dehnte sich von einer auf fast zweieinhalb Stunden aus und beschloss damit eine produktive Woche.
