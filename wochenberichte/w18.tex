\section{Woche 18 - CVSS Calculator Web UI} \label{sec:bericht-wo-18}

% 2024-01-15 bis 2024-01-19

\lweekdaymarginpar{Mo, Di}

Den Beginn der Woche verbrache ich damit, die Implementierung von CVSS:4.0 in TypeScript abzuschließen.
Wie auch schon für die vorherigen Versionen war das Definieren der CVSS-Metriken und ihrer möglichen Werte war einer der aufwendigeren Teile, insbesondere bei den über 30 Metriken von CVSS:4.0.
Da ich, im Vergleich zur JavaScript-Referenzimplementierung mit einer einzigen Datei, in meinem Java-Code die Implementierung deutlich objektorientierter in über 7 Klassen gestaltet habe, habe ich zum Übernehmen der Implementierung einen eher ungewöhnlichen Ansatz gewählt:
Ich kopierte die Java-Klassen direkt in das Projekt rüber und übersetzte sie nach und nach in TypeScript.

Dieser Prozess lief natürlich nicht reibungslos, und ich musste oft Konstrukte komplett neu schreiben, aber es war auf jeden Fall einfacher, als alles komplett neu schreiben zu müssen.
Eine unerwartete Herausforderung war, die mir zuvor noch nicht bewusst war, dass TypeScript keine zirkulären Abhängigkeiten zwischen Klassen zulässt.
Um diese zu identifizieren, nutzte ich madge\footnote{\url{https://www.npmjs.com/package/madge}}, um Abhängigkeitsgraphen zu erstellen.
Zur Lösung der Abhängigkeiten verlagerte ich einige Funktionen einer Basisklasse in ein Interface, um den Typ des Interfaces in einer anderen Klasse verwenden zu können, die auch in der Basisklasse verwendet wird.

\sweekdaymarginpar{Mi, Do, Fr}

Nachdem alle drei Versionen die Tests bestanden hatten, konnte ich mich der Implementierung des Web-Interfaces zuwenden.
Einige Wochen zuvor hatten mein Chef und ich bereits Mockups für das Interface auf Papier entworfen.
Das Design, das uns nun am besten gefiel, setzte ich in der zweiten Wochenhälfte mit Bootstrap\footnote{\url{https://getbootstrap.com/}} und ChartJs\footnote{\url{https://www.chartjs.org/}} um.

Ich erstellte eine HTML-Struktur mit verschiedenen Container-Elementen, die später durch JavaScript mit Inhalten befüllt werden sollten.
Das JavaScript teilte ich in fünf Dateien auf, die jeweils für einen Teil des Interfaces, für Zugriffe auf externe Daten oder als Adapter zur CVSS TypeScript-Bibliothek für die Score-Berechnung zuständig sind.

Nachdem die Basisfunktionalität vorhanden war, verbrachte ich den Rest der Woche damit, das UI weiter zu verbessern, Lizenzinformationen hinzuzufügen, die TypeScript-Bibliothek aufzuräumen und Dokumentation zu erstellen.
Am Ende der Woche veröffentlichte ich das Projekt auf GitHub\footnote{\url{https://github.com/org-metaeffekt/metaeffekt-universal-cvss-calculator}} und teilte es auf LinkedIn\footnote{\url{https://www.linkedin.com/feed/update/urn:li:activity:7151175714694729728/}}.
Mein Chef und ich waren sehr erfreut über das Ergebnis, da der CVSS-Rechner ein lang geplantes Projekt war.
Er stellte es noch am selben Tag und in der folgenden Woche unseren Kunden vor, die großes Interesse zeigten und auch, wie zu erwarten war, viele Verbesserungsvorschläge einbrachten.
