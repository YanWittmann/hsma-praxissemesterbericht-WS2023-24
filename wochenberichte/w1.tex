\section{Woche 1 - Einarbeitung in CVSS 4.0} \label{sec:bericht-wo-1}

% Woche 1 (2023-09-04 bis 2023-09-08)

\lweekdaymarginpar{\weekdayMondayLong}

Mein erster Arbeitstag im Praktikums bei der {\metaeffekt} fiel mit dem Ende der Sommerpause des Unternehmens zusammen.
Da ich bereits seit einiger Zeit im Unternehmen arbeite und ich meine eigenständigen Aufgabenbereiche habe, war eine Einführung für mich nicht notwendig.
Bei {\metaeffekt} ist mein Aufgabenbereich als Entwickler ein automatisiertes Vulnerability Monitoring für unsere Kunden in der Programmiersprache Java zu implementieren und zu betreuen.
Als Hauptverantwortlicher für dieses Gebiet bin ich darüber hinaus für den Kundenkontakt für Fragen, Anforderungen und Unterstützung zuständig.
Ich verbrachte den Montag damit, einige während der Sommerpause aufgetretene Fehler in den Systemen von Kundenprojekten zu korrigieren und Gespräche mit Kollegen zu führen, um anstehende Projekte und Aufgaben zu klären.

Ein wichtiges Thema war die anstehende Veröffentlichung des CVSS 4.0-Standards, die für den 31.\ Oktober 2023\footnote{\url{https://www.first.org/cvss/v4-0/}} geplant war.
Die Software-Implementierung der {\metaeffekt} ist bereits in der Lage, die Scores der CVSS-Versionen 2.0 und 3.1 zu berechnen, und wir müssen in der Lage sein, auch die neuen CVSS 4.0-Vektoren zu berechnen, sobald offizielle Datenquellen diese auch bei sich integrieren.
Mit meinem Chef und Betreuer für das Praktikum, Karsten Klein, habe ich zudem vereinbart, während meines Praxis-Semesters tägliche Meetings mit ihm abzuhalten.

\sweekdaymarginpar{\weekdayTuesdayLong}

Am Dienstag startete ich damit, die zu dem Zeitpunkt noch unfertige Dokumentation und Beispiele von CVSS 4.0 zu studieren.
Ich stellte schnell fest, dass es mehr Unterschiede als Gemeinsamkeiten zu den vorherigen Versionen gibt, insbesondere in Bezug auf die mathematischen Hintergründe.
Ich dokumentierte dennoch meine Erkenntnisse in unserem internen Confluence Wiki.
Auf dem offiziellen RedHat GitHub-Repository\footnote{\url{https://github.com/RedHatProductSecurity/cvss-v4-calculator}} fand ich den Quellcode einer Referenz-JavaScript-Implementierung, die noch sehr nützlich werden sollte.

\sweekdaymarginpar{\weekdayWednesdayLong}

Am Mittwoch begann ich mit einem ersten Versuch einer Implementierung der CVSS 4.0-Berechnungen.
Wie ich bereits gestern vermutet hatte, ist die Berechnung bei 4.0 mathematisch deutlich komplexer, mit Hamming-Distanzen zwischen Vektoren und der Interpolation und Skalierung von mehrdimensionalen Räumen, versehen.
Dank der RedHat JavaScript-Implementierung konnte ich Mittwoch das Grundgerüst für meine Implementierung in Java vorbereiten.

\sweekdaymarginpar{\weekdayThursdayLong}

Der Donnerstag startete mit der Behebung einer \qt{OutOfMemoryError}-Exception im Code eines unserer Reports in einem Kundenprojekt, die auftrat, wenn eine zu große Menge an Daten verarbeitet wurde.
Das Problem war, dass während der Serialisierung in ein HTML-Dokument das interne Modell (damit auch der Speicherbedarf) kurzzeitig dupliziert wurde.
Ich konnte das Problem lösen, indem ich einen FileAppender verwende, der den HTML-String des Reports direkt in eine Datei schreibt, anstatt ihn wie zuvor im Speicher zu halten.

Der Hauptteil meines Tages war jedoch der Untersuchung von CVSS 4.0 zugeordnet.
Leider musste ich im Verlauf feststellen, dass die Referenzimplementierung und die Spezifikation im aktuellen Zustand voneinander abweichen, was bei einem unveröffentlichten Standard zwar verständlich, aber nicht hilfreich ist.
Ich meldete dieses Problem zusammen mit inhaltlichen Fragen in einem GitHub-Issue\footnote{\url{https://github.com/RedHatProductSecurity/cvss-v4-calculator/issues/32}} und beendete den Tag mit einer teilweise funktionierenden Implementierung.

\sweekdaymarginpar{\weekdayFridayLong}

Am Freitag erhielt ich recht schnell eine Antwort auf meine Fragen:
Wie erwartet ist die Spezifikation veraltet und die Implementierung korrekt.
Mithilfe dieser Informationen konnte ich die Berechnungen fertigstellen und durch einen Test-Datensatz, den ich durch die Referenzimplementierung erzeugt habe, validieren.
Damit war der erste Teil dieses Tasks erledigt:
Als Nächstes wollte ich mein Verständnis für die Theorie hinter CVSS 4.0 verbessern und zudem musste die neue Version noch richtig in unsere Systeme integriert werden.

Freitagmittag findet bei der {\metaeffekt} ein wöchentliches Meeting statt, das als \qt{Weekly} bezeichnet wird.
Hier berichtete ich über meine Erfahrungen mit der Implementierung von CVSS 4.0 und hörte, was die anderen Teammitglieder in dieser Woche erreicht hatten.
So beendete ich meine erste Praktikumswoche.
