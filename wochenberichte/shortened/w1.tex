\section{Woche 1 - Einarbeitung in CVSS 4.0} \label{sec:bericht-wo-1}

% Woche 1 (2023-09-04 bis 2023-09-08)

\lweekdaymarginpar{\weekdayMondayLong}

Mein erster Arbeitstag im Praktikums bei der {\metaeffekt} fiel mit dem Ende der Sommerpause des Unternehmens zusammen.
Da ich bereits seit einiger Zeit im Unternehmen arbeite und ich meine eigenständigen Aufgabenbereiche habe, war eine Einführung für mich nicht notwendig.
Bei {\metaeffekt} ist mein Aufgabenbereich als Entwickler ein automatisiertes Vulnerability Monitoring für unsere Kunden in der Programmiersprache Java zu implementieren und zu betreuen.
Ich verbrachte den Montag damit, einige während der Sommerpause aufgetretene Fehler in den Systemen von Kundenprojekten zu korrigieren und Gespräche mit Kollegen zu führen.

Ein näherkommendes Thema war die anstehende Veröffentlichung des CVSS 4.0-Standards, die für den 31.\ Oktober 2023\footnote{\url{https://www.first.org/cvss/v4-0/}} geplant war.
Zu den CVSS-Versionen 2.0 und 3.1 soll auch unser System CVSS 4.0-Vektoren berechnen können.
Mit Karsten Klein, meinem Chef und Betreuer für das Praktikum, habe ich zudem vereinbart, während meines Praxis-Semesters zusätzliche tägliche Meetings mit ihm abzuhalten.

\sweekdaymarginpar{\weekdayTuesdayLong}

Am Dienstag startete ich damit, die zu dem Zeitpunkt noch unfertige Dokumentation und Beispiele von CVSS 4.0 zu studieren.
In diesen wurde die offizielle Referenzimplementierung\footnote{\url{https://github.com/RedHatProductSecurity/cvss-v4-calculator}} referenziert, die später noch sehr nützlich werden würde.
Ich dokumentierte meine Erkenntnisse über die Gemeinsamkeiten und Unterschiede in unserem internen Confluence Wiki.

\sweekdaymarginpar{\weekdayWednesdayLong}

Am Mittwoch begann ich mit einem ersten Versuch einer Implementierung der CVSS 4.0-Berechnungen.
Wie ich bereits gestern vermutet hatte, ist die Berechnung bei 4.0 mathematisch deutlich komplexer, mit Hamming-Distanzen zwischen Vektoren und der Interpolation und Skalierung von mehrdimensionalen Räumen, versehen.
Dank der RedHat JavaScript-Implementierung konnte ich Mittwoch das Grundgerüst für meine Implementierung in Java vorbereiten.

\sweekdaymarginpar{\weekdayThursdayLong}

Donnerstag musste ich feststellen, dass die Referenzimplementierung und die Spezifikation voneinander abweichen, was bei einem unveröffentlichten Standard zwar verständlich, aber nicht hilfreich ist.
Ich meldete dieses Problem zusammen mit inhaltlichen Fragen in einem GitHub-Issue\footnote{\url{https://github.com/RedHatProductSecurity/cvss-v4-calculator/issues/32}} und begann den Rest des Tages bereits, die Implementierung zu schreiben.

\sweekdaymarginpar{\weekdayFridayLong}

Am Freitag erhielt ich bereits Antworten darauf:
Wie erwartet ist die online-Spezifikation nicht aktuell.
Den Rest des Tages konnte ich meine Implementierung so weit fertigstellen, dass ich sie durch einen Test-Datensatz tatsächlich bereits validieren konnte.
Als Nächstes wollte ich mein Verständnis von CVSS 4.0 verbessern, bevor ich die neue Version noch richtig in unsere Systeme integriert muss.

Leider hat den restlichen Tag eine \qt{OutOfMemoryError}-Exception in einem Kundenprojekt meine Zeit eingenommen, die auftrat, wenn eine zu große Menge an Daten verarbeitet wurde.
Ich konnte das Problem, dass während der Serialisierung in ein HTML-Dokument das interne Modell (damit auch der Speicherbedarf) kurzzeitig dupliziert wurde, jedoch schnell beheben.
Freitagmittag findet bei der {\metaeffekt} ein wöchentliches Meeting statt (\qt{Weekly}), hier berichtete ich über meine Erfahrungen mit der Implementierung von CVSS 4.0.
So beendete ich meine erste Praktikumswoche.
