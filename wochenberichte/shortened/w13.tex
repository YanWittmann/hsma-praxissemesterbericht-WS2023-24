\section{Woche 13 - Fertigstellung der Integration des Datenmodells \headerand Dokumentation} \label{sec:bericht-wo-13}

% 2023-11-27 bis 2023-12-01

\lweekdaymarginpar{\weekdayMondayShort, \weekdayTuesdayShort}

Gegen Ende der letzten Woche fiel es mir zunehmend schwer, mich auf den Report zu konzentrieren.
Die Pause am Wochenende war anscheinend hilfreich, denn bis Dienstagabend konnte ich nach diesen zwei Wochen die Integration in den Report fast vollständig abschließen.
Ich musste natürlich auch hier noch neue Segmente in den Templates anlegen, die die Herkunft einer Schwachstelle erklären können, wie ich es auch schon im VAD getan hatte.

\sweekdaymarginpar{\weekdayWednesdayShort, \weekdayThursdayShort}

Am Mittwoch startete ich motiviert in den Tag, da nur noch wenige Schritte bis zur Fertigstellung des neuen Prozesses fehlten.
Dazu zählten vor allem die Übersichtsdiagramme mit verschiedenen Statistiken über die gefundenen Schwachstellen, die sowohl im VAD als auch im PDF-Report dargestellt werden.
Im VAD verwenden wir dafür ChartJs\footnote{\url{https://www.chartjs.org}}, während für den PDF-Report SVG-Charts mit JFreeChart\footnote{\url{https://www.jfree.org/jfreechart}} während der Dashboard-Generierung gerendert werden.

Ich stellte schnell fest, dass nie genau definiert wurde, welche Diagramme welche Werte aus welchen Quellen darstellen sollen und wie diese Werte gemappt werden.
Diese Unklarheit erschwerte es mir, die Diagramme im neuen Prozess zu replizieren, da ich die genauen Datenquellen erst durch Reverse-Engineering ermitteln musste.
Bevor ich also mit der Implementierung anfing, begann ich, ein wenig Dokumentation zu diesem Thema zu verfassen.
Tatsächlich konnte ich diese Arbeit bis später am Donnerstag abschließen und hatte damit das große Thema des Refactorings des Datenmodells quasi abgeschlossen.

\sweekdaymarginpar{\weekdayFridayLong}

Weil mir in den vergangenen Tagen die Wichtigkeit von Dokumentation wieder einmal aufgefallen war, nutzte ich den Freitag, um im internen Wiki einen \qt{Migrationsguide} zu erstellen.
Dieser dokumentiert alle Änderungen zwischen der alten und der neuen Generation unseres Systems und enthält zusätzliche Informationen zu einzelnen Themen.
Dazu gehören der Refactor der CVSS-Implementierung mit Unterstützung für mehrere Vektoren gleicher Art, die sonstige Erweiterung der Prozesse um CVSS, das Tracking der Herkunft von CVSS-Vektoren und Schwachstellen, das neue Datenmodell für Schwachstellen und Security Advisories, die zentrale Security Policy Konfiguration, neue Namenskonventionen, geändertes Verhalten und vieles mehr.

Im Weekly Meeting konnte ich berichten, dass der neue Prozess nahezu abgeschlossen ist.
Obwohl es noch einige Monate dauern wird, bis die ersten Kunden diesen nutzen, ist es immer ein gutes Gefühl, ein solches Projekt abzuschließen.
