\section{Woche 7 - Windows-Inventar-Extraktion \headerand Strategieworkshop} \label{sec:bericht-wo-7}

% Woche 7 (2023-10-16 bis 2023-10-20)

\lweekdaymarginpar{\weekdayMondayShort, \weekdayTuesdayShort}

Montag habe ich eine erste version der PowerShell Skripte fertigstellen können, die alle Use-Cases abdeckt.
Im Meeting später am Tag mit den Mitarbeitern von {\aeclientZEZESE} wurden meine Datensammlungs-Skripte dann live auf dem Ziel-Windows-Gerät erfolgreich ausgeführt.
Diese Daten auszuwerten hat gezeigt, dass sie noch nicht reichen, um ein vollständiges Bild zu erhalten, was ich dann den restlichen Tag durch Modifikationen an den Skripten geändert habe.

\sweekdaymarginpar{\weekdayWednesdayShort, \weekdayThursdayShort}

Die Entwicklung eines Java-Prozesses zur Verarbeitung von JSON-Daten aus PowerShell-Befehlen für ein Inventar im Format von {\metaeffekt} machte es nötig, die Ergebnisse der vielen verschiedenen PowerShell-Befehle zu kombinieren.
Wie bei Microsoft-Datenquellen so oft liefern die unterschiedlichen Befehle teils überlappende, teils einzigartige Datensätze, die nur zusammengenommen ein volles Bild ergeben.
Besonders bei Systeminformationen und PNP-Geräten waren Daten aus mehreren Befehlen zu konsolidieren.
Das Ergebnis war dann am Donnerstagabend ein vorläufiges Inventar, das zur Besprechung mit dem Kunden am Freitag noch etwas händisch aufbereitet wurde.

\sweekdaymarginpar{\weekdayFridayLong}

Der Freitag war ein ereignisreicher Tag:
Die {\metaeffekt} hat einen Strategieworkshop gehalten, der das Vorgehen der nächsten 9--12 Monate angeben sollte.
An einem großen Tisch und auf mehreren großen Whiteboard-Blättern wurden Wünsche und Pflichten aufgeschrieben und diskutiert.
Zu den Strategiepunkten, bei denen ich beteiligt sein werde, gehören:

\begin{smitemize}
    \item Eine Java-Implementierung von CVSS:2.0, CVSS:3.1 und CVSS:4.0 soll als Open-Source-Projekt auf GitHub veröffentlicht werden.
    Dazu muss die CVSS:4.0-Implementierung fertiggestellt und zusammen mit den anderen in unser Haupt-Repository verschoben werden.
    \item Ein Open-Source-Projekt, das den CVSS-Standard in den Versionen 2.0, 3.1 und 4.0 in TypeScript implementiert und damit auch im Web nutzbar ist, soll angelegt werden.
    Damit soll ein öffentliches Web-Interface erstellt werden, das die Berechnung und Modifikation von CVSS-Vektoren ermöglicht.
    \item Das interne Datenmodell, das für die Speicherung von Schwachstellen und Security Advisories verwendet wird, soll komplett neu geschrieben werden.
    Es soll mit diesem System zu jeder Zeit möglich sein, die exakte Quelle einer Schwachstelle und der von CVSS-Vektoren programmatisch zu identifizieren.
    \item Eine der Ausgaben unseres Systems ist ein sog. \qt{Vulnerability Assessment Dashboard} (VAD), was Ergebnisse aus dem Schwachstell-Monitoring mit aggregierten Details darstellt.
    Es soll nach der Neuentwicklung des Datenmodells in einer \qt{Generation 3.0} stark verändert werden, so sollen auch Kundenwünsche berücksichtigt werden.
\end{smitemize}

Das Meeting war sehr, hilfreich für mich, da es einen klaren roten Faden für das Semester vorgegeben hat.
Den Mitarbeitern von {\aeclientZEZESE} haben wir im Anschluss die Ergebnisse der Windows-Scans gezeigt und darum gebeten, dass sie die aktualisierten Skripte erneut ausführen, damit wir die vollständigeren Daten zu einem besseren Inventar umwandeln können.
