\section{Woche 2 - Vertiefung in CVSS 4.0 \headerand Korrelationsdaten} \label{sec:bericht-wo-2}

% Woche 2 (2023-09-11 bis 2023-09-15)

\lweekdaymarginpar{\weekdayMondayLong}

Ich verbrachte den Montag damit, die Spezifikation\footnote{\url{htthttps://www.first.org/cvss/v4.0/specification-document}},
die Entwicklungsgeschichte\footnote{\url{https://www.first.org/cvss/v4.0/user-guide\#New-Scoring-System-Development}}
und den Code im Kontext der Berechnung der Scores tiefergehender zu verstehen.
Den ersten Schritt mit den MacroVektoren hatte ich ein bereits gut verstanden, das Problem war eher der zweiten Schritt mit der Interpolation zwischen den einzelnen MacroVektoren.
Ich konnte selbst nach einem ganzen Tag an Recherche keine zufriedenstellende Erklärung finden, wie die Berechnung in diesem Schritt funktioniert.

\sweekdaymarginpar{\weekdayTuesdayShort, \weekdayWednesdayShort}

In Abwesenheit eines Kollegen übernahm ich Dienstag seine Aufgabe, die Pflege von sog. \qt{Korrelationsdaten} (siehe Kap. \ref{subsec:projektbericht-grundlagen-vulnerability-monitoring}).
Dazu konnte ich ein von mir entwickeltes Tool nutzen, das ich kurz vor meinem Praktikum als Web-UI über Spring Boot neu aufgesetzt hatte.

\sweekdaymarginpar{\weekdayThursdayLong}

Da die Implementierung und Integration von CVSS 4.0 bis Wochenende abgeschlossen sein musste, musste ich mich mit den folgenden verbleibenden Aufgaben intensiv beschäftigen:
Das Parsing der Vektoren aus verschiedenen Quellen/Formaten, das korrekte Verarbeiten der Berechnung von Scores und Modifizieren von Vektoren und das Anzeigen der Ergebnisse in unseren HTML- und PDF-Reports.
Da keine externe Datenquelle bisher CVSS 4.0 Vektoren bereitstellt, basierten einige meiner Annahmen über deren Formate auf Vermutungen, die später eventuell noch angepasst werden müssen.
Ich nutzte den Rest des Tages, um viele Code-Muster, die ich aus der Referenzimplementierung übernommen hatte, durch Refactoring-Operationen eleganter und objektorientierter zu gestalten und Code zu deduplizieren.
Am Ende des Tages stellte ich Pull Requests für die drei betroffenen Code-Projekte fertig.

\sweekdaymarginpar{\weekdayFridayLong}

Freitag widmete ich mich erneut dem Verständnis von CVSS 4.0.
Unter anderem berechnete ich manuell mehrfach auf unterschiedliche Weisen die drei Beispiele der MacroVektor-Interpolation aus der Spezifikation, was mein Verständnis erheblich verbesserte.

Das Weekly am Ende des Tages dehnte sich von einer auf fast zweieinhalb Stunden aus, da nicht nur ich, sondern auch einige Kollegen, diese Woche viel erreicht hatten.
