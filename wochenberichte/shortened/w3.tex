\section{Woche 3 - Excel-Limitierungen \headerand Präsentationsvorbereitung} \label{sec:bericht-wo-3}

% Woche 3 (2023-09-18 bis 2023-09-24)

\lweekdaymarginpar{\weekdayMondayLong}

Montag sollte ich unseren automatisierten CPE Matching-Algorithmus zwischen Hardware- und Software-Komponenten zu unterscheiden lassen, sodass Softwarekomponenten nicht mehr Hardware CPEs (und umgekehrt) zugeordnet werden würden.
Ob eine Komponente Hardware oder Software ist, wollten wir durch eine für die einzelnen Komponenten vergebbare definierte Liste an Kategorien ermöglichen.
Die Implementierung war recht simpel, doch der Detailgrad dieser Kategorien war jedoch bisher noch nicht geklärt.
Nach einiger Recherche habe ich drei Vorschläge für meinen Chef vorbereitet und wir konnten uns recht schnell auf einen einigen.
Eine davon deutliche abweichende Drittmeinung eines anderen Kollegen hat die Diskussion den gesamten restlichen Tag einnehmen lassen.
Ohne eine Lösung zu finden, vertagten wir die Diskussion.

\sweekdaymarginpar{\weekdayTuesdayShort\ - \weekdayThursdayShort}

Den Dienstag konnte ich eine lang-ersehnte Verbesserung der Excel-Serialisierung und Deserialisierung unserer Software-Inventare vornehmen.
Das Zeichenlimit von $32.767$\footnote{\url{https://support.microsoft.com/en-gb/office/excel-specifications-and-limits-1672b34d-7043-467e-8e27-269d656771c3}} Zeichen pro Excel-Zelle stellte uns vor Probleme, da unsere Daten oft dieses Limit überschreiten.
Aktuell gibt es einen Workaround, der die Daten bereits im Datenmodell in mehrere Spalten aufteilt, was gewisse Probleme hervorbringt.
Mit einer eleganteren Lösung, die die Datenaufteilung nicht schon im Datenmodell, sondern erst beim (De-)Serialisierungsprozess vornimmt habe ich, zusammen mit einem allgemeinen Styling-Modell für Excel-Dokumente, die alte Implementierung ersetzt.

\sweekdaymarginpar{\weekdayFridayLong}

Freitagmorgen wurde ich von meinem Chef überrascht:
Nächste Woche Dienstag und Mittwoch sollte mit ihm und einer Kollegin nach Erfurt auf das Treffen des Arbeitskreises OpenSource\footnote{\url{https://www.bitkom.org/Bitkom/Organisation/Gremien/Open-Source.html}} unter dem Thema \qt{Open-Source-Communities} und auf das darauf folgende Forum OpenSource\footnote{\url{https://www.bitkom.org/bfoss23}} der {\bitkom} gehen.
Der eigentliche überraschende Punkt war, dass ich bei dem Treffen des Arbeitskreises eine 25-Minütige Präsentation vor 30 Leuten, unter anderem von RedHat, Siemens, DB Systel und anderen großen Unternehmen, über die \qt{Identifikation und Bewertung von Schwachstellen mit Inhalten aus öffentlichen Quellen} halten sollte.

Ich nahm mir den restlichen Tag für die Vorbereitung darauf.
Trotz der Unterstützung meines Chefs bei der Themenauswahl und Strukturierung, war klar, dass die Zeit knapp werden würde.
Ich schaffte es, die Präsentationsfolien weitestgehend zu erstellen, mit dem Skript hatte ich jedoch gerade erst begonnen.

\sweekdaymarginpar{\weekdaySaturdayShort, \weekdaySundayShort}

Und so kam das erste Mal, dass ich an einem Wochenende für die {\metaeffekt} gearbeitet habe.
Das Wochenende verbrachte ich damit, ein 11-seitiges Skript für die Präsentation zu schreiben, die Folien weiter anzupassen und Punkte notiert zu notieren, die ich noch mit meinem Chef am Montag besprechen wollte.
Am Sonntagabend hatte ich ein ausführliches Skript fertiggestellt.
