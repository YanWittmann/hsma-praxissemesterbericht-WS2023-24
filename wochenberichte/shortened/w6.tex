\section{Woche 6 - Daten-Korrelation \headerand PowerShell Skripte} \label{sec:bericht-wo-6}

% Woche 6 (2023-10-09 bis 2023-10-13)

\lweekdaymarginpar{\weekdayMondayShort\ - \weekdayWednesdayShort}

Da mein Kollege nicht das gesamte Inventar den Rest der Woche fertig bekommen würde, spielte sich diese Woche ähnlich wie letzte ab, in der ich meinen Kollegen bei der Arbeit unterstützte.
So konnte ich das Tool, das ich für diese Arbeit vor meinem Praktikum geschrieben hatte, selbst einmal anwenden und entdeckte einige Verbesserungsmöglichkeiten, die ich gleich umsetzte.
Das Tool (\qt{Correlation Utilities}) selbst ist Webapplikation, die mit einem lokal gehosteten Server, in Spring Boot implementiert, interagiert.
Das Ziel des Tools ist es, den Prozess des Mappings von unseren internen Produkten zu denen externer Datenbanken zu unterstützen.
Es aggregiert relevante Informationen automatisiert und macht Empfehlungen, wie am besten mit Fällen umgegangen werden sollte.
Dank des Tools besteht nun bereits fast keine Notwendigkeit, das ursprungs-Inventar zu durchsuchen oder Internet-Suchen zu starten.
Über die drei Tage habe ich es mit einigen weiteren Features erweitert.
Bis Mittwochabend hatten wir die Hälfte der Daten durchgearbeitet, den Rest sollte mein Kollege bis zum Ende der nächsten Woche erledigen.

\sweekdaymarginpar{\weekdayThursdayLong}

Donnerstag hatten mein Chef und ich morgens einen zweistündigen Termin mit einem unserer Kunden, {\aeclientZEZESE}, bei dem es um die automatisierte Erstellung einer SBOM (Software Bill of Materials) mit allen installierten Programmen, Treibern und Hardware-Devices auf Windows-Systemen ging.
Dieser Prozess sollte zweigeteilt sein:
Zunächst sollten über PowerShell-Skripte über Windows-Integrierte Features viele verschiedene Datenquellen angezapft und die rohen Ergebnisse in einem maschinenlesbaren Format in Dateien geschrieben werden.
Danach würde ein Maven-Plugin diese Daten analysieren und ein Inventar erzeugen.

Bis nächsten Montag sollten bereits erste Versionen der Skripte stehen.
Mein Arbeitslaptop selbst ist ein MacBook, also hat mir mein Chef einen zusätzlichen Windows-Laptop zum Entwickeln zur Verfügung gestellt.
An diesem habe mich zunächst einmal darüber informiert, wie man am besten an Windows-Systeminformationen gelangen kann.
Diese ersten Erkenntnisse habe ich zunächst in unserem internen Confluence niedergeschrieben, wo ich auch sonst meine Dokumentation ablege.

\sweekdaymarginpar{\weekdayFridayLong}

Freitag habe mich schnell wieder an das Thema gesetzt und mit den Ergebnissen meiner gestrigen Recherche angefangen, erste Skripte zur Sammelung von registrierten Programmen aus dem Store oder über Installers, die PNP-Devices (\qt{Plug and Play}) und Treibern.
Ich konnte die Hälfte der Use-Cases noch an diesem Tag durch verschiedene Skripte abdecken.
