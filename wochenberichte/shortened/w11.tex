\section{Woche 11 - Transferieren der Datenklassen nach Core} \label{sec:bericht-wo-11}

% 2023-11-13 bis 2023-11-17

\lweekdaymarginpar{\weekdayMondayLong}

Ende letzter Woche hatte ich die CVSS-bezogenen Features implementiert und in den Anreicherungsprozess integriert, sodass das System nun CVSS-Vektoren von beliebigen Datenquellen aufnehmen und deren Quellen nachvollziehbar halten konnte.
Den Montag nutzte ich, um diese Vektoren und deren berechneten Scores im VAD auf eine angereicherte Art anzuzeigen, was erstaunlich gut funktionierte.

\sweekdaymarginpar{\weekdayTuesdayLong}

Dienstagmorgen besprach ich mit meinem Chef die Integration dieser Änderungen in den PDF-Report unseres Core-Projekts.
Wir entschieden uns dazu, vorläufig Teile der Klassen in das andere Projekt zu kopieren, um auch dort Zugriff auf die Parsing-Logik zu haben, was zwar nicht schön ist (Code-Duplizierung), aber für jetzt die einfachere Lösung ist.
Noch am Dienstag konnte ich die relevanten Klassen in das Core-Projekt übernehmen und testen, wobei ich eine Namenskonvention für die kopierten Klassen festlegte und jeweils deren ursprüngliche Herkunft vermerkte.

\sweekdaymarginpar{\weekdayWednesdayShort, \weekdayThursdayShort, \weekdayFridayShort}

Mittwoch und Donnerstag stellte sich heraus, dass der Austausch des Datenmodells hinter dem PDF-Report mit dem kopierten Datenmodell komplexer war als erwartet.
Ich musste einige Abschnitte im Datenmodell leider komplett neu implementieren.
Den Rest der Zeit konnte ich dann das aktualisierte Modell in den Report einbinden.

Wir verwenden Apache Velocity mit einem textbasierten Template-XML-Format, was die Integration des neuen Modells aus mehreren Gründen sehr zeitintensiv machte.
Bis Freitagmittag war die Migration des Reports noch nicht abgeschlossen, allerdings hat war am Nachmittag ein Meeting mit einer Mitarbeiterin von \aeclientZEZESE\ geplant, um die Nutzung unseres VADs und die Bewertung von Schwachstellen in ihren Systemen zu besprechen.
Als Vorbereitung erstellte ich eine HTML-Seite, die unsere verschiedenen öffentlichen JSON-Schema-Dateien dynamisch zusammenfasst.
Das Meeting verlief erfreulicherweise angenehm und war produktiv für beide Seiten.
