\section{Woche 12 - Integration des Datenmodells in PDF-Report} \label{sec:bericht-wo-12}

% 2023-11-20 bis 2023-11-24

\lweekdaymarginpar{\weekdayMondayLong}

Am Montag erweiterte ich das Tracking-System der Matching-Konfigurationen aus verschiedenen Quellen von Schwachstellen, um nicht nur die bisherigen \qt{CPE}-Informationen, sondern auch Versionsbereiche und sogar Quellen wie GitHub und Microsoft zu integrieren.
Diese Informationen konnte ich noch am noch Montag in das VAD visuell ansprechend integrieren, was mir die Einfachheit von Anpassungen am VAD im Vergleich zum PDF-Report noch einmal deutlich machte.

\sweekdaymarginpar{\weekdayTuesdayShort\ - \weekdayThursdayShort}

Die folgenden Tage kehrte ich wieder zur Integration des Datenmodells in den PDF-Report zurück.
Für jedes der etwa 20 Velocity-Templates ist der Prozess recht ähnlich:
Das Überprüfen der alten Datenzugriffe, im neuen Modell nach einem entsprechenden Zugriff suchen oder das Implementierte von neue Methoden, wiederholen sich stets.
Diese Änderungen nahm ich entweder in den Adapterklassen oder direkt im Modell vor, was Änderungen sowohl in Core als auch in Artifact Analysis erforderte.
Um den Dita-Renderingprozess (PDF-Generation aus Dita-Files) nicht bei jedem Test starten zu müssen, nutzte ich \qt{OxygenXML} für eine Live-Preview, doch der Prozess ist noch immer ein langer geblieben.

\sweekdaymarginpar{\weekdayFridayLong}

Freitag begann mit der Anfrage meines Chefs, ob ich an einem Workshop zum CSAF-Standard\footnote{\url{https://web.archive.org/web/20240121120954/https://www.allianz-fuer-cybersicherheit.de/Webs/ACS/DE/Netzwerk-Formate/Veranstaltungen-und-Austausch/CSAFversum/CSAFversum_node.html}} (Common Security Advisory Framework) teilnehmen möchte, der vom BSI in München organisiert wird.
Nach einer kurzen Recherche stimmte ich zu, da die Integration von CSAF in unser System schon länger geplant ist und mich persönlich auf eine Reise nach München freue.
Den Rest des Tages sollte ich aufgrund der Abwesenheit eines anderen Kollegen erneut Korrelationsdaten für ein dringendes Inventar anlegen.
