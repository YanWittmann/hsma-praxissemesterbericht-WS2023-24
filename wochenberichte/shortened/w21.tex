\section{Woche 21 - Praktikumsbetreuung \headerand Abschluss des Praxissemesters} \label{sec:bericht-wo-21}

% 2024-02-05 bis 2024-02-09

\lweekdaymarginpar{\weekdayMondayLong}

Die letzte Woche meines Praktikums sollte ich einen BOGY-Praktikanten bei uns betreuen und die Integration von Generation 3 bei den Kunden fertigstellen.
Mit dem Praktikanten übte Montag ich Programmierkonzepte in Java, da er hauptsächlich Testfälle für unseren Software-Scanner und Komponenten-Extraktor schreiben sollte.

\sweekdaymarginpar{\weekdayTuesdayLong}

Der Praktikant konnte Konzepte schnell aufnehmen und begann Dienstag bereits mit den Testfällen.
Parallel dazu löste ich ein Problem mit der Laufzeit von Generation 3, das durch einen Programmierfehler von durchschnittlich 10 Minuten in Gen.\ 2 auf über zwei Stunden in Gen.\ 3 angestiegen war.
Eine kleine Code-Änderung (die Umwandlung einer Liste in ein Set) reduzierte die Laufzeit auf zweieinhalb Minuten, was uns natürlich alle beruhigt hatte.

\sweekdaymarginpar{\weekdayWednesdayShort, \weekdayThursdayShort}

Mittwoch und Donnerstag behebte ich in einem Teams-Meeting in Live-Kooperation die letzten Probleme, sodass Generation 3 nun endlich überall vollständig integriert war.
Die restliche Zeit verbrachte ich mit dem Praktikanten und diversen Aufgaben.

\sweekdaymarginpar{\weekdayFridayLong}

An meinem letzten Tag brachte ich Kuchen mit und verteilte ihn beim Weekly und nach dem Meeting gingen wir gemeinsam Currywurst essen.
Ich behebte ein Issue mit einem unserer Reports, das durch die neue Filtermethode in Generation 3 entstanden war, und führte neue Parameter und alternative Report-Templates ein.
Der BOGY-Praktikant und ich beendeten mit dieser Woche unser Praktikum erfolgreich.
Ich arbeitete noch weitere 12 Tage bei dem Unternehmen und schloss einen weiteren Vertrag für weitere Zusammenarbeit ab.
