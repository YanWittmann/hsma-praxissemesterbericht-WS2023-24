\section{Woche 20 - BSI-Meeting \headerand Integration von Generation 3} \label{sec:bericht-wo-20}

% 2024-01-29 bis 2024-02-02

\lweekdaymarginpar{\weekdayMondayLong}

Montags fand mit dem BSI ein Meeting bezüglich des CSAF-Standards und unseres CVSS-Rechners statt, mit Thomas Schmidt\footnote{\url{https://www.it-meets-industry.de/de/referent-thomas-schmidt}}, welcher bereits Leiter des CSAF-Workshops in München Ende letzten Jahres war, und Herr Von Samson.
Nach einer Demo unseres Toolings konnten wir nicht nur zur CSAF-Integration, sondern auch zum CVSS-Rechner, einige Tasks ableiten.
Thomas Schmidt erstellte in den folgenden Tagen dazu noch einige Issues auf unserem GitHub-Repository.

\sweekdaymarginpar{\weekdayTuesdayShort\ - \weekdayThursdayShort}

Die restliche Woche startete dann die Integration von Generation 3 unseres Monitorings bei den mehreren Projekten unserer Kunden, um alle einheitlich von Generation 1 und 2 auf die neueste Generation 3 zu heben.
Der Prozess beinhaltete die Aktualisierung der Versionen und vor allem der Konfigurationen zum neuen Format, wobei der zuvor verfasste Migrationsguide wirklich sehr hilfreich war.
Trotz zahlreicher Komplikationen waren bis Donnerstagnachmittag alle Projekte aktualisiert.

\sweekdaymarginpar{\weekdayFridayLong}

Freitag bearbeitete ich die Issues\footnote{\url{https://github.com/org-metaeffekt/metaeffekt-universal-cvss-calculator/issues?q=is\%3Aissue+is\%3Aclosed}} von Thomas Schmidt, wobei bis auf Issue \#2 (keine Rückmeldung von FIRST\footnote{\url{https://www.first.org/cvss}} bis Praktikumsende) keine Probleme auftraten.
Nach unserem Weekly gab es ein Meeting mit einem Kunden zur finalen Integration der Generation 3 in ihre Pipeline.
