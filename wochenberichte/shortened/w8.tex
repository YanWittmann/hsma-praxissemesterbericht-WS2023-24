\section{Woche 8 - Abschluss Windows-Extraktion \headerand Beginn Überarbeitung des Datenformats} \label{sec:bericht-wo-8}

% Woche 8 (2023-10-23 bis 2023-10-27)

\lweekdaymarginpar{\weekdayMondayShort, \weekdayTuesdayShort}

Mit den neuen Daten vom Freitag konnte ich Montag die Erkennung von Treibern, PNP-Devices und optionalen Features und die Performance des umfangreicheren Dateisystem-Scans einiger Skripte verbessern.
Um Dienstag die Windows-Extraktion vorerst abzuschließen, habe ich den restlichen Tag noch die PowerShell-Skripte unter einer MIT-Lizenz auf GitHub veröffentlicht und ein Maven-Plugin für die Inventar-Extraktion in Java geschrieben.

\sweekdaymarginpar{\weekdayWednesdayShort, \weekdayThursdayShort}

Mittwoch konnte ich (endlich) mit der Neuimplementierung der Datenstruktur und Logik dahinter beginnen.
Die einzelnen Tasks, die damit einhergehen, würden mich also die nächsten Wochen beschäftigen.
Bevor ich tatsächlich etwas programmieren konnte, wollte ich meine geplanten Änderungen in unserem internen Wiki dokumentieren und planen:
Begonnen habe ich mit dem Einführen eines Systems, das eine Quelle und Version eines Vektors eindeutig angeben kann.
Details dazu können in Kapitel \ref{subsec:projektbericht-loesungsweg-cvss-source-management} gefunden werden.
Eine erste Implementierung dazu konnte ich bereits Donnerstag fertigstellen.

\sweekdaymarginpar{\weekdayFridayLong}

Freitag habe ich damit verbracht, einem Kollegen zu helfen, der Probleme mit einer Software-Bibliothek hatte, die wir seit geraumer Zeit einsetzen.
Das Problem ließ sich am Ende auf einen internen Cache der Bibliothek zurückführen, den wir den Betreuern der Bibliothek in einem Issue\footnote{\url{https://github.com/spdx/Spdx-Java-Library/issues/215}} mitteilten.
Dieses Issue wurde einige Tage später durch einen Pull Request (PR) von meinem Kollegen behoben.
