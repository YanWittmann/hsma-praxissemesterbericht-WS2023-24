\section{Woche 14 - Neuer Kollege, automatische Korrelationsdaten \headerand Validierung des neuen Prozesses} \label{sec:bericht-wo-14}

% 2023-12-04 bis 2023-12-08

\lweekdaymarginpar{\weekdayMondayLong}

Der Montag war auch der erste Arbeitstag eines neuen Kollegen, der uns bei der Entwicklung einer CI-Pipeline und eines Testing-Frameworks unterstützen sollte.
Ich half ihm vormittags bei der Einrichtung seines Laptops und erklärte ihm unsere Codebasis.
Nachmittags widmete ich mich einem anderen Kollegen, um über seine Änderungen an den Korrelationsdaten zu gehen, was den Rest des Tages in Anspruch nahm.

\sweekdaymarginpar{\weekdayTuesdayLong}

Diese gestrige Session mit den Korrelationsdaten erinnerte mich an die Art und Weise, wie wir Java-Versionen mit diesem System erkennen und, dass es nur ein provisorisches System sein sollte.
Ich entwarf Dienstag also ein System, das automatisch Korrelations-Einträge für alle bekannten Java-Versionen generieren kann.
Nach drei erfolglosen Iterationen über dieses Problem fand ich dann eine akzeptable und funktionierende Lösung und stellte sie dem Kollegen und meinem Chef vor.


\sweekdaymarginpar{\weekdayWednesdayShort, \weekdayThursdayShort}

In den folgenden Tagen nahm ich einen Schritt zurück, um den Refactoring-Prozess des Datenmodells noch einmal zu überprüfen.
Ich stellte fest, dass ich zwei größere Klassen vergessen hatte zu überführen und korrigierte zudem noch einige Fehler, die im Vergleich zu Generation 2 zu (zu stark) abweichenden Ergebnissen führten.
Nach diesen Korrekturen waren die Ergebnisse verbessert, und ich konnte endlich Generation 3 des Vulnerability-Monitorings meinem Chef präsentieren und besprechen.

\sweekdaymarginpar{\weekdayFridayLong}

Der bevorstehende CSAF-Workshop nächste Woche rückte näher, darum widmete ich den Freitag der Recherche über CSAF, indem ich die Dokumentation und einige Beispiele ansah.
Meine Erkenntnisse fasste ich in einem neuen Wiki-Artikel zusammen, wurde jedoch durch kleinere Anfragen und das wöchentliche Meeting immer wieder unterbrochen.
Die Recherche würde ich in der nächsten Woche fortsetzen.
