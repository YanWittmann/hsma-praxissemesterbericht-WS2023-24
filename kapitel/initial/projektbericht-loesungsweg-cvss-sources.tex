\subsection{CVSS-Vektor Quellenmanagement} \label{subsec:projektbericht-loesungsweg-cvss-source-management}

Im Kontext der zuvor diskutierten Grundlagen stößt man unweigerlich auf die Herausforderung, mehrere voneinander abweichende CVSS-Vektoren für eine einzige Schwachstelle zu handhaben.
Ein System zur Verwaltung dieser Vektoren ist daher notwendig, das nicht nur mehrere Vektoren der gleichen Version zulässt, sondern diesen auch die ursprüngliche Quelle und weitere Metadaten mit hoher Eindeutigkeit anhängen kann.
Zudem muss die Serialisierung und Deserialisierung der Informationen sowohl in einem menschenlesbaren Format, als auch maschinenlesbaren möglich sein.
Eine weitere Anforderung ist es, die Quellen der Vektoren auch dann noch nachvollziehen zu können, wenn mehrere Vektoren miteinander kombiniert wurden.
Dessen Konzeption berücksichtigte eine Vielzahl von Quellen, wie unter anderem:

\begin{itemize}
    \item NVD:
    Organisationen und Institutionen können, wenn sie als sogenannte CVE Naming Authorities (CNA) registriert sind, für CVEs beliebige CVSS-Vektoren veröffentlichen.
    Eine öffentlich verfügbare Liste an CNAs ist online\footnote{\url{https://nvd.nist.gov/vuln/cvmap/search}} mit Metadaten verfügbar.
    \item Microsoft Security Response Center (MSRC):
    Sie stellen für jede für Microsoft relevante Schwachstelle und pro betroffene (numerische) Produkt-Id einen CVSS-Vektor zur Verfügung.
    \item CERT-SEI und GitHub Security Advisories (GHSA): Stellen jeweils einen Vektor pro referenzierter Schwachstelle zur Verfügung.
\end{itemize}

Aus den abgeleiteten Anforderungen wurde das in Listing \ref{lst:cvss-source-format} abgebildete Format für Vektor-Quellen entworfen, wobei jede der drei Zeilen eine der alternativen Optionen darstellt.
Eine Quelle besteht $immer$ $mindestens$ mit der \qt{CvssVersion} aus einer Vektor-Version und einer \qt{HostingEntity}, die die Entität darstellt, die den Vektor auf ihrer Platform zur Verfügung stellt.
Eine Entität kann sich z.B.\ auf die NVD, MSRC, GHSA, usw.\ beziehen.
Wenn die \qt{HostingEntity} nicht die Entität ist, die den Vektor initial zur Verfügung gestellt hat, dann muss dieser Herausgeber als \qt{IssuingEntity} zusätzlich gesetzt sein.
In gewissen externen Schwachstelldatenbanken werden die Issuing Entities unter definierten Rollennamen geführt, wie die CNAs bei der NVD\@.
In diesem Fall muss auch die \qt{IssuerRole} angegeben werden.

Die Darstellung dieser Quelleninformationen in einer Datenstruktur erfolgt einfach durch separate Attribute und kann einfach z.B.\ als JSON serialisiert werden.
Für die Serialisierung als menschen- aber auch maschinenlesbarer String müssen jedoch einige Regeln definiert werden.
Vom Konzept her wird einfach das Format aus Listing \ref{lst:cvss-source-format} verwendet; um allerdings die bereits definierten Bindestriche als Trennzeichen weiterhin verwenden zu können, müssen eventuelle Bindestriche (\qt{-}) in Entitäten und Rollenbezeichnern durch \qt{\textbackslash-} ersetzt werden und alle \qt{\textbackslash} durch \qt{\textbackslash\textbackslash}, um Konflikte zu vermeiden.
Das Fehlen von Leerzeichen in CVSS-Versionen erlaubt die einfache Trennung der Version von den übrigen Elementen mittels des ersten Leerzeichens.

\begin{lstlisting}[language={}, label={lst:cvss-source-format}, caption={Alternative CVSS-Quellen Formatdefinitionen}]
CvssVersion HostingEntity
CvssVersion HostingEntity-IssuingEntity
CvssVersion HostingEntity-IssuerRole-IssuingEntity
\end{lstlisting}

Zur Unterstützung kombinierter Vektoren und Quellen wird das Format um die Möglichkeit erweitert, mehrere Quellen mittels der Zeichenfolge \qt{ + } zu trennen, wie in Listing \ref{lst:cvss-source-format-combined} dargestellt.
Plus-Zeichen (\qt{+}) innerhalb der Quellen müssen daher auch entsprechend als \qt{\textbackslash+} kodiert werden.
Trotz der Kombination verschiedener Quellen ist $nur$ $eine$ Angabe der CVSS-Version nötig, da eine Kombination von Vektoren unterschiedlicher Versionen von Grund auf nicht möglich ist.
Die Reihenfolge der Quellen ist dennoch von Bedeutung, da die erste Quelle den Basisvektor darstellt und folgende Quellen in der angegebenen Reihenfolge auf diesen angewandte Modifikationen sind.

Listing \ref{lst:cvss-source-format-examples} zeigt beispielhafte Quellenangaben.
Dieses Format wird im weiteren Verlauf, speziell in Kapitel \ref{subsec:projektbericht-loesungsweg-cvss-selection}, als Grundlage für die Verwaltung und Auswahl von CVSS-Vektoren dienen.

\begin{lstlisting}[language={}, label={lst:cvss-source-format-combined}, caption={CVSS-Quellen Format \(Kombiniert\)}]
CvssVersion HostingEntity-IssuerRole-IssuingEntity + HostingEntity-IssuerRole-IssuingEntity + ...
\end{lstlisting}

\begin{lstlisting}[language={}, label={lst:cvss-source-format-examples}, caption={CVSS Sources Format}]
CVSS:3.1 Microsoft Corporation
CVSS:2.0 Assessment-all
CVSS:2.0 NVD-CNA-NVD
CVSS:3.1 NVD-CNA-Microsoft Corporation
CVSS:4.0 Assessment-lower
\end{lstlisting}

Als Zusatz zu diesem System ist für die Zuordnung von Metadaten wie E-Mail-Adressen oder URLs zu den jeweiligen Entitäten ist der Aufbau eines Datensatzes der bekannten Quellen erforderlich.
Ein dafür entworfener automatisierter Prozess, der auf die Inhalte von CVE.org\footnote{\url{https://www.cve.org/PartnerInformation/ListofPartners}} zugreift, erstellt regelmäßig eine lokale Liste und speichert diese in mehreren JSON-Dateien.
Dazu wird ein zusätzlicher Datensatz gepflegt, um auch Entitäten zu unterstützen, die nicht in dieser Liste geführt werden.
