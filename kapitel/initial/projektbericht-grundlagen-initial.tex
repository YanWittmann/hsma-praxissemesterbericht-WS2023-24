\section{Grundlagen} \label{sec:projektbericht-grundlagen}

Um eine gemeinsame Grundlage zu schaffen, werden in diesem Kapitel die wichtigsten Begriffe und Standards erläutert, die im weiteren Verlauf des Berichts verwendet werden.

\subsection{Software-Inventare} \label{subsec:projektbericht-grundlagen-inventories}

Software-Inventare werden bei der {\metaeffekt} in einem eigens entwickelten, proprietären Format, schlicht \qt{Inventar} genannt, abgelegt.
Meist wird, um ein solches Inventar zu erhalten, ein ebenfalls eigens entwickelter Scanner verwendet, der ganze Dateisysteme nach Komponenten durchsucht und daraus ein Inventar generiert.
Um möglichst breiten Support zu bieten, gibt es allerdings nicht nur den Scanner, sondern auch diverse Konverter, mit denen Inventare von und zu Formaten wie CycloneDX SBOM\footnote{\url{https://cyclonedx.org/specification/overview}} oder SPDX\footnote{\url{https://spdx.dev}} umgewandelt werden können.

Diese Inventare können mehrere Kategorien an Daten enthalten: Software-Komponenten (\qt{Artefakte}), Schwachstell-Informationen, Security Advisories, Lizenzinformationen und einige weitere.

\subsection{NVD / NIST} \label{subsec:projektbericht-grundlagen-nvd-nist}

Software-Schwachstellen sind allgegenwärtig, jedes Softwareprodukt hat sie und meistens es ist nur eine Frage der Zeit, bis sie gefunden, veröffentlicht und im schlimmsten Fall ausgenutzt werden.
Die National Vulnerability Database (NVD)\footnote{\url{https://nvd.nist.gov}} des National Institute of Standards and Technology (NIST) der U.S.\ Regierung stellt mit ihrem CVE-System\textsuperscript{\ref{subsec:projektbericht-grundlagen-cve-cpe}} eine der primären Quellen für Schwachstell-Informationen für Forscher, Unternehmen und automatisierte Tools bereit.
Sobald eine neue Schwachstelle bekannt gegeben wird, nehmen sie diese in ihren Katalog an CVE mit auf, versehen sie mit CVSS- und Matching-Informationen über das CPE-System.
Damit ist die Schwachstelle für alle auf der Welt über eine API oder über ihr User Interface erreichbar und kann, falls ein Projekt betroffen ist, frühzeitig kontextualisiert bewertet werden.

\subsection{CVE / CPE} \label{subsec:projektbericht-grundlagen-cve-cpe}

\qt{CVE} ist ein von dem CVE Project eingeführtes System, das Schwachstellen in Software- und Hardwareprodukten eindeutig identifiziert und beschreibt.
Der Standard wird stark von der NVD des NIST unterstützt.
Im CVE-System bekommt jede Schwachstelle, die von einem Forscher oder einer Organisation gefunden und veröffentlicht wird, eine eindeutige ID von der Form \qt{CVE-YYYY-NNNN} zugewiesen, wobei \qt{YYYY} das Jahr der Veröffentlichung und \qt{NNNN} eine eindeutige, fortlaufende Nummer ist.
CVE definiert eine Schwachstelle als:

\begin{quote}
    A weakness in the computational logic (e.g., code) found in software and hardware components that, when exploited, results in a negative impact to confidentiality, integrity, or availability.
    Mitigation of the vulnerabilities in this context typically involves coding changes, but could also include specification changes or even specification deprecations (e.g., removal of affected protocols or functionality in their entirety).
    \cite{nvdVulnerabilityDefinition}
\end{quote}

Eine Schwachstelle ist also eine Schwäche, eine theoretische Angriffsfläche, die durch Ausnutzen negative Auswirkungen auf die Vertraulichkeit, Integrität oder Verfügbarkeit eines Systems hat.
Es soll kein CVE-Eintrag existieren, der keinen Einfluss auf die Vertraulichkeit, Integrität oder Verfügbarkeit eines Systems hat \cite{nvdVulnerabilityMetrics}.

Jeder CVE werden gewisse Metadaten zugeordnet, wie Beschreibungen, Referenzen und (potenziell mehrere) CVSS-Vektoren zur Bewertung des Schweregrads, aber auch \qt{CPE}s, die Aussagen über die betroffenen Produkte machen und automatisierte Zuordnungen erlauben.
Common Platform Enumeration (CPE) ist ein von der MITRE Corporation\footnote{\url{https://cpe.mitre.org/specification}} entwickeltes System, mit dem Soft- und Hardwareprodukte über den Hersteller, Produktnamen und Version eindeutig identifiziert werden können.
Die CPE Naming Specification Version 2.3 \cite[Seite 37, Kapitel 6.2]{NISTIR7695} definiert die Syntax eines CPE-Strings, in der mindestens \qt{part} (Komponenten-Typ), \qt{vendor} (Hersteller) und \qt{product} (Produkt) gesetzt sein müssen:

\code{cpe : 2.3 : part : vendor : product : version : update : edition}\newline\hphantom{\code{cpe}}\code{: language : sw_edition : target_sw : target_hw:other}

Auf dem NVD Dashboard\footnote{\url{https://nvd.nist.gov/general/nvd-dashboard}} ist geführt, dass es am 08.03.2024 insgesamt 240899 eindeutige CVEs und 1261617 CPEs gibt.
Diese Zahl wächst exponentiell und zeigt, wie wichtig es ist, Schwachstellen automatisiert zu verarbeiten.

\subsection{CVSS} \label{subsec:projektbericht-grundlagen-cvss}

Das Common Vulnerability Scoring System (CVSS) wird (in Version 4.0) \cite{CVSSv4.0Specification} vom FIRST\footnote{\url{https://www.first.org}} (Forum of Incident Response and Security Teams) als ein offener Standard für die Bewertung der Sicherheitsanfälligkeit von Software- und Hardwarekomponenten beschrieben.
Dieser Standard legt fest, wie sog. CVSS-Vektoren, definiert als Sammlung von Schlüssel-Wert Paaren, zur möglichst objektiven Klassifizierung von Schwachstelleneigenschaften genutzt werden können.
Solche Vektoren können textuell als \code{CVSS:VERSION/KEY:VALUE/KEY:VALUE/...} formatiert werden, wobei jeder Vektor mit \qt{CVSS:} gefolgt von der Vektor-Version beginnen und jedes Wertepaar durch \qt{/} getrennt werden muss.

Allgemein reichen die Scores, die aus diesen Vektoren berechnet werden, von 0 bis 10 und ordnen die Schwachstellen in die Kategorien \qt{Low}, \qt{Medium}, \qt{High} und \qt{Critical} ein, wobei höhere Werte eine dringendere Bewertung erfordern.
Diese Berechnungen lassen sich leicht mit Online-Rechnern\footnote{\url{https://metaeffekt.com/security/cvss/calculator}} oder Software-Tools, verfügbar in allen populären Programmiersprachen, durchführen.
Ein Beispiel für einen Basisvektor in Version 3.1, der einen mittleren Schweregrad (\qt{Medium}) mit einem Score von 6.0 darstellt, kann folgendermaßen aussehen:\newline\code{CVSS:3.1/AV:L/AC:L/PR:H/UI:N/S:U/C:H/I:H/A:N}.

Die versionsabhängigen Metriken der Vektoren ordnen jeweils eine Charakteristik einer Schwachstelle einem Wert zu.
So steht beispielsweise \code{AV:L} für einen \qt{Attack Vector} mit dem Wert \qt{Adjacent Network}, was die räumliche Nähe beschreibt, die ein Angreifer benötigt, um die Schwachstelle auszunutzen.
Andere Werte wie \qt{Network} (aus dem Internet ausnutzbar, höchster Schweregrad), \qt{Local} oder \qt{Physical} (physikalischer Zugang benötigt, am weigsten schlimm) spezifizieren weitere Angriffsszenarien für diese Metrik.

Zusätzlich zu den allgemeinen, umgebungsunabhängigen Basismetriken, die von den Herausgebern der Schwachstelleninformationen festgelegt werden, gibt es noch die \qt{Temporal}/\qt{Threat}- und \qt{Environmental}-Metriken.
Diese Metriken reflektieren die spezifische Umgebung und den Kontext, in dem eine Schwachstelle existiert, und werden in betroffenen Projekten von den Anwendern kontextabhängig gesetzt.
Damit erlaubt es CVSS, eine initiale, kontextunabhängige Bewertung eines Systems, und durch Hinzufügen von Kontextinformationen auch eine kontextualisierte Sicht auf das System zu erhalten.
Einer Schwachstelle können von verschiedenen Parteien potenziell mehrere, unterschiedliche Vektoren zugeordnet werden, dieser Aspekt wird später wichtig.

\subsection{Automatisiertes Vulnerability Monitoring} \label{subsec:projektbericht-grundlagen-vulnerability-monitoring}

Das bei der {\metaeffekt} eingesetzte automatisierte Vulnerability Monitoring, hat das Ziel, für ein gegebenes Software-Inventar nicht nur die relevanten Schwachstellen (\qt{Vulnerabilities}) zu identifizieren, sondern auch zugehörige Ratgeber (\qt{Security Advisories}) als Hilfestellung zu der darauf folgenden manuellen Bewertung bereitzustellen.
Der Prozess, dargestellt in Grafik \ref{fig:vulnerability-monitoring-overview-figure}, in zwei Hauptphasen mit mehreren Unterschritten gegliedert, wird im Folgenden beschrieben.
Als Eingabe dient in jedem Fall ein Inventar gefüllt mit einer Liste an Software-Komponenten und deren Metadaten, wie Version, Quelle und Ökosystem-spezifischen Informationen.

Der erste Prozessschritt ist die Identifikation von Schwachstellen und wird getrennt voneinander auf alle Komponenten im Inventar angewandt.
Die in unserem Prozess verwendeten externen Schwachstellendatenbanken nutzen abstrahierte Darstellungen von Produkten und Software-Komponenten für ihre internen Verweise zwischen Produkten und Schwachstellen.
Beispielsweise verwendet die NVD CPEs (siehe Kapitel \ref{subsec:projektbericht-grundlagen-cve-cpe}), GitHub Security Advisories\footnote{\url{https://github.com/advisories}} folgen dem OSV-Schema\footnote{\url{https://osv.dev}} mit Ökosystem-abhängigen Informationen, und Microsoft (MSRC)\footnote{\url{https://msrc.microsoft.com/update-guide}} arbeitet mit numerischen Identifikatoren, die nur durch Download und Extraktion aller monatlichen Zusammenfassungen über ihre API zugänglich sind.
Die Zuordnung unserer realen Komponenten zu diesen abstrahierten Produkten stellt daher immer eine Herausforderung dar.
Unser Vulnerability Monitoring setzt verschiedene Algorithmen ein, um diese Zuordnungen so weit wie möglich automatisch vorzunehmen, was jedoch nicht immer fehlerfrei gelingt.
Falsch positive und falsch negative Ergebnisse erfordern dann eine manuelle Korrektur durch einen gepflegten Datensatz (\qt{Korrelationsdaten}), was einen erheblichen Zeitaufwand bedeutet.

Wenn dieser Schritt erfolgreich abgeschlossen ist, können Abfragen an die jeweiligen Datenbanken gestartet werden, um Schwachstellen durch Produkt- und Versionsabgleichen zu identifizieren.
Dieses Zwischeninventar mit Schwachstellinformationen kann bereits für Berichte genutzt werden, jedoch wird meistens ein weiterer Schritt dahinter geschaltet:
Weitere Abfragen an einen größeren Satz an Datenbanken werden gestartet, um zu den gefundenen Schwachstellen entsprechende Ratgeber zuzuordnen.
Diese Abfragen sind in der Regel einfacher, da die Ratgeber-Einträge meistens direkt auf die betroffenen Schwachstellen-IDs Bezug nehmen.
Das Ergebnisinventar kann dann als Grundlage für die Erstellung von Reports und Statistiken benutzt werden.

\begin{figure}[htbp] % here, top, bottom, separate page
    \centering
    \includegraphics[width=1\textwidth, keepaspectratio]{res/grafiken/vulnerability-monitoring-overview}
    \caption{Schwachstellsuche mit einem Software-Inventar}
    \label{fig:vulnerability-monitoring-overview-figure}
\end{figure}

\subsection{Schwachstellen-Priorisierung durch CVSS} \label{subsec:projektbericht-grundlagen-vulnerability-assessment}

Wenn eine Liste an Schwachstellen durch den Prozess in \ref{subsec:projektbericht-grundlagen-vulnerability-monitoring} als Minimalanforderung in einem Projekt für ein Inventar erfasst wurde, muss nun ein zweiter Schritt folgen:
Die manuelle Bewertung der Schwachstellen, indem eine Einschätzung, ein Status und erforderliche Maßnahmen angegeben werden.
Jedoch kann es sowohl bei großen, aber auch schon bei kleinen Projekten mit kritischen Abhängigkeiten zu einer großen Menge an Schwachstellen kommen (1000 und aufwärts), die oftmals nicht alle sofort von einer Person oder einem Bewertungs-Team durchgearbeitet werden können.
Es werden also objektive und automatisierbare Faktoren benötigt, anhand denen die Schwachstellen möglichst ohne weiteres Zutun in eine priorisierte Liste eingeordnet werden können, um die kritischsten zuerst abarbeiten zu können.

Diese Faktoren sind Projekt- und Umgebungsspezifisch und können stark variieren.
In diesem Kapitel wird nur CVSS als Priorisierungsfaktor betrachtet, wie sie in der CVSS-Spezifikation beschrieben werden.
In Kapitel \ref{subsec:projektbericht-grundlagen-cvss} wurde bereits auf die Metrik-Gruppen eingegangen.
Da jede Schwachstelle aus dem CVE-System und den meisten anderen Systemen mindestens einen CVSS-Vektor von den Herausgebern angehängt bekommen hat, kann bereits ohne großen Mehraufwand eine nach dem daraus errechneten Basis-Score sortierte Liste als initiale Reihenfolge dienen.

Vor allem die Environmental-Metrikgruppe kann nun zur Kontextualisierung von Schwachstellen in einem Projekt verwendet werden.
Das Beispiel bezüglich des \qt{Attack Vector}s in \ref{subsec:projektbericht-grundlagen-cvss}, bei dem \code{AV} auf \code{L} kann hier weiter getrieben werden:
Wenn eine aus einem lokalen Netzwerk ausnutzbare Schwachstelle in einer Applikation vorhanden ist, die keinerlei Netzwerkzugang hat, dann kann diese auch nicht aus einem Netzwerk ausgenutzt werden, sondern eventuell nur durch physikalischen Zugriff.
Allgemein kann dann also eine globale, Inventar-weite CVSS-Vektoränderung von \code{AV:P} angewendet werden, aus der eine neue, kontextualisierte Liste an Schwachstellen hervorgeht, die nun akkurater die tatsächlichen Risiken eines Systems darstellen.
Weitere Metriken wie das Risiko eines Verlusts an Konfidentialität, Integrität oder Verfügbarkeit können für weitere Anpassungen verwendet werden.
