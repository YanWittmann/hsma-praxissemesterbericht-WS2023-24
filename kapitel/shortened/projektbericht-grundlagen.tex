\section{Grundlagen} \label{sec:projektbericht-grundlagen}

Um eine gemeinsame Grundlage zu schaffen, werden in diesem Kapitel die wichtigsten Begriffe und Standards erläutert, die im weiteren Verlauf des Berichts verwendet werden.

\subsection{Software-Inventare} \label{subsec:projektbericht-grundlagen-inventories}

Software-Inventare werden bei der {\metaeffekt} in einem eigens entwickelten, proprietären Format, schlicht \qt{Inventar} genannt, abgelegt.
Meist wird, um ein solches Inventar zu erhalten, ein ebenfalls eigens entwickelter Scanner verwendet, der ganze Dateisysteme nach Komponenten durchsucht und daraus ein Inventar generiert.
Um möglichst breiten Support zu bieten, gibt es allerdings nicht nur den Scanner, sondern auch diverse Konverter, mit denen Inventare von und zu Formaten wie CycloneDX SBOM\footnote{\url{https://cyclonedx.org/specification/overview}} oder SPDX\footnote{\url{https://spdx.dev}} umgewandelt werden können.

Diese Inventare können mehrere Kategorien an Daten enthalten: Software-Komponenten (\qt{Artefakte}), Schwachstell-Informationen, Security Advisories, Lizenzinformationen und weitere.

\subsection{NVD / NIST} \label{subsec:projektbericht-grundlagen-nvd-nist}

Die National Vulnerability Database (NVD)\footnote{\url{https://nvd.nist.gov}} des National Institute of Standards and Technology (NIST) der U.S.\ Regierung stellt mit ihrem CVE-System\textsuperscript{\ref{subsec:projektbericht-grundlagen-cve-cpe}} eine der primären Quellen für Schwachstell-Informationen bereit.
Neue Schwachstellen werden in den CVE-Katalog aufgenommen und mit CVSS- und Matching-Informationen versehen.
Die Schwachstellen sind damit global über eine API oder über die Website erreichbar.

\subsection{CVE / CPE} \label{subsec:projektbericht-grundlagen-cve-cpe}

\qt{CVE} (Common Vulnerabilities and Exposures) ist ein vom CVE Project eingeführtes System, das Schwachstellen in Software- und Hardwareprodukten eindeutig identifiziert und beschreibt.
Im CVE-System bekommt jede Schwachstelle eine eindeutige ID der Form \qt{CVE-YYYY-NNNN} zugewiesen und wird nur als Schwachstelle bezeichnet, wenn sie ausgenutzt wird, die Vertraulichkeit, Integrität oder Verfügbarkeit negativ beeinflusst.
Die Behebung solcher Schwachstellen kann Codeanpassungen, Änderungen in den Spezifikationen oder sogar die komplette Entfernung betroffener Protokolle oder Funktionen erfordern \cite{nvdVulnerabilityDefinition}.
Es soll kein CVE-Eintrag existieren, der keinen Einfluss auf die Vertraulichkeit, Integrität oder Verfügbarkeit eines Systems hat \cite{nvdVulnerabilityMetrics}.

Auf dem NVD Dashboard\footnote{\url{https://nvd.nist.gov/general/nvd-dashboard}} wird angezeigt, dass es am 08.03.2024 insgesamt 240899 eindeutige CVEs und 1261617 CPEs gibt.
Diese Zahl wächst exponentiell und zeigt, wie wichtig es ist, Schwachstellen automatisiert zu verarbeiten.

\subsection{CVSS} \label{subsec:projektbericht-grundlagen-cvss}

Das Common Vulnerability Scoring System (CVSS) wird (in Version 4.0) \cite{CVSSv4.0Specification} vom FIRST\footnote{\url{https://www.first.org}} als ein offener Standard für die Bewertung der Sicherheitsanfälligkeit von Software- und Hardwarekomponenten beschrieben.
Dieser Standard legt fest, wie CVSS-Vektoren zur möglichst objektiven Klassifizierung von Schwachstelleneigenschaften genutzt werden können.
Sie können als \code{CVSS:VERSION/KEY:VALUE/KEY:VALUE/...} formatiert werden, wobei jeder Vektor mit \qt{CVSS:} gefolgt von der Version beginnen und jedes Wertepaar durch \qt{/} getrennt werden muss.

Allgemein reichen die Scores, die aus diesen Vektoren berechnet werden, von 0 bis 10 und ordnen die Schwachstellen in die Kategorien \qt{Low}, \qt{Medium}, \qt{High} und \qt{Critical} ein, wobei höhere Werte eine dringendere Analyse erfordern.
Diese Berechnungen lassen sich leicht mit Online-Rechnern\footnote{\url{https://metaeffekt.com/security/cvss/calculator}} oder Software-Tools, verfügbar in allen populären Programmiersprachen, durchführen.

Die versionsabhängigen Metriken der Vektoren ordnen jeweils eine Charakteristik einer Schwachstelle einem Wert zu.
So steht beispielsweise \code{AV:L} für einen \qt{Attack Vector} mit dem Wert \qt{Adjacent Network}, was die räumliche Nähe beschreibt, die ein Angreifer benötigt, um die Schwachstelle auszunutzen.
Andere Werte wie \qt{Network} (aus dem Internet ausnutzbar, höchster Schweregrad), \qt{Local} oder \qt{Physical} (physikalischer Zugang benötigt, am wenigsten ernst) spezifizieren weitere Angriffsszenarien für diese Metrik.

\subsection{Schwachstellen-Priorisierung durch CVSS} \label{subsec:projektbericht-grundlagen-vulnerability-assessment}

Nachdem eine Liste an Schwachstellen bekannt ist, muss die manuelle Bewertung der Schwachstellen durch Einschätzungen, einem Status und erforderlichen Maßnahmen angegeben werden.
Objektive und automatisierbare Faktoren helfen, die Schwachstellen in eine priorisierte Liste einzuordnen.
Diese Faktoren sind Projekt- und Umgebungsspezifisch und können stark variieren.
Hier wird nur CVSS als Priorisierungsfaktor betrachtet, wie sie in der CVSS-Spezifikation beschrieben werden.
Jede Schwachstelle besitzt i.d.R. mindestens einen CVSS-Vektor und kann damit bereits ohne großen Mehraufwand eine nach dem Score sortierte Liste als initiale Reihenfolge erzeugen.

Zur Kontextualisierung dieser Liste kann danach die Environmental-Metrikgruppe dienen.
Angenommen, in einem Vektor ist \code{AV} auf \code{L} gesetzt und die betroffene lokale Schwachstelle ist in einer netzwerkisolierten Applikation netzwerktechnisch nicht ausnutzbar, dann kann eine Anpassung des CVSS-Vektors zu \code{AV:P} erfolgen.
Zusätzliche Metriken wie Vertraulichkeits-, Integritäts- oder Verfügbarkeitsrisiken unterstützen weitere Feinabstimmungen.
