%! Author = Yan Wittmann


\chapter{Firmenumfeld und Zielsetzung} \label{ch:firmenumfeld-zielsetzung}


\section{Firma} \label{sec:firma-beschreibung}

Die {\metaeffekt} GmbH ist ein spezialisiertes Unternehmen, das sich auf die kontinuierliche Inventarisierung, Dokumentation und Bewertung von Risiken in Softwarebestandteilen von Produkten und Projekten konzentriert, um Unternehmen bei der Software Composition Analysis zu unterstützen.
Die {\metaeffekt} arbeitet eng mit verschiedenen Fachbereichen und Verantwortlichen der Kunden zusammen, um maßgeschneiderte Lösungen anzubieten.
Darüber hinaus bietet die {\metaeffekt} Schulungen und Seminare zu Themen wie License Compliance Awareness, License Compliance Management, Vulnerability Monitoring und Vulnerability Assessment an.

Die {\metaeffekt} hebt sich durch einen Fokus auf die tatsächlich genutzten Softwarebestandteile im Gegensatz zu einem spezifikationsbasierten Scannen von anderen Tools und Unternehmen ab.
Durch den Einsatz fallspezifischer Werkzeuge und Beratung erreicht das Unternehmen eine hohe Datenqualität, die für die Umsetzung der License Compliance in der Lieferkette und für die Identifikation sowie Überwachung von öffentlichen Schwachstellen wichtig ist.

Die Abteilung \qt{Vulnerability Monitoring}, in der das Praktikum stattfand, entwickelt Werkzeuge und Prozesse, um die automatisierte kontinuierliche Überwachung von Schwachstellen in Softwareprodukten durch die erfassten Software-Inventare zu ermöglichen.

Es gibt andere Organisationen, die Teile der Dienstleistungen von {\metaeffekt} anbieten.
Dazu gehören Unternehmen wie
Black Duck (Sicherheits- und Lizenzanforderungsanalyse von Open Source-Komponenten),
WhiteSource (Lizenzanforderungsanalyse von Open Source-Komponenten und Compliance),
Snyk (Identifizierung und Behebung von Sicherheitslücken in Open Source-Komponenten) und
Sonatype (Identifizierung und Behebung von Sicherheitslücken und Lizenzproblemen in Open Source-Komponenten).


\section{Zielsetzung} \label{sec:zielsetzung-des-praktikums}

Das Ziel des Praktikums war es, die größtenteils von mir zuvor entwickelte Softwarelösung für die automatisierte Überwachung von Schwachstellen in Softwareprodukten bei den Kunden zu betreuen, zu verbessern und zu erweitern.
Spezifische Ziele haben sich während den ersten Wochen herausgebildet.

Das erste Ziel sollte das Ersetzen der bisherigen \qt{Generation 2} unseres Vulnerability Monitoring durch eine \qt{Generation 3} sein.
Dies beinhaltet sowohl die Implementierung des Standards CVSS in der Version 4.0 in Java, als auch eine komplette Überarbeitung unseres Datenmodells hinter dem Schwachstellen-Management.
Die Integration all dieser Änderungen in unsere Systeme und Reports soll auch Teil davon sein.

Das zweite Ziel war es, eine TypeScript-Implementierung aller aktuellen CVSS-Versionen zu schreiben, die in einer Web-UI als Online-Rechner zur Verfügung gestellt wird.

Weiter gibt es neben den Entwicklungsaufgaben auch Integrations- und Konfigurationsarbeiten in Kundenprojekten und allgemein die Beratung und Unterstützung der Kunden.
