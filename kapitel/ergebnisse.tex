%! Author = Yan Wittmann

\chapter{Ergebnisse} \label{ch:ergebnisse}

In dem Praxissemester wurden unter anderem die folgenden Haupt-Ergebnisse erzielt:

\begin{itemize}
    \item Implementierung des CVSS 4.0 Standards in die vorhandene Softwarelösung und Integration in den Daten-Mirror und die Reporting-Tools.
    \item Überarbeitung des Datenmodells zur Unterstützung von mehreren CVSS-Vektoren je Schwachstelle, verbessertem Tracking der Quellen von Schwachstellen und der Generalisierung der Datenzugriffe, um Redundanzen in den Implementierungen zu vermeiden.
    \item Implementierung der CVSS-Versionen 2.0, 3.0, 3.1 und 4.0 in TypeScript.
    \item Entwickeln eines online-CVSS-Rechners, der aus den eigenen Reporting-Tools referenziert wird und bereits bei verschiedenen Events vorgestellt wurde.
    \item Integration der durchgeführten Änderungen in die entsprechenden Kundenprojekte an den Kundensystemen und im konstanten Dialog mit den Zuständigen.
    \item Besuchen des Arbeitskreises OpenSource des {\bitkom} und halten eines Vortrages.
    \item Besuchen des OpenSource Forums des {\bitkom} in Erfurt.
    \item Besuchen des zweitägigen CSAF-Workshops in München im ISH (Information Security Hub), durchgeführt vom BSI\@.
\end{itemize}

Die folgenden konkreten öffentliche Ergebnisartefakte wurden dabei unter anderem erzeugt (mit Links hinterlegt):

\begin{itemize}[noitemsep]
    \item \href{https://www.metaeffekt.com/security/cvss/calculator}{Universal CVSS Calculator}
    \item \href{https://github.com/org-metaeffekt/metaeffekt-universal-cvss-calculator}{GitHub Repository der TypeScript Implementierung}
    \item \href{https://youtu.be/R2S0_6-NQGQ?si=d7zpxbJ7P4R26nRb&t=2801}{Webinar zum CVSS Calculator}
    \item \href{https://mvnrepository.com/artifact/com.metaeffekt.artifact.analysis/ae-artifact-analysis}{Weiterführung des aritfact-analysis Projektes}
    \item \href{https://github.com/org-metaeffekt/metaeffekt-core}{Weiterführung des Core Projektes}
\end{itemize}
