\documentclass[11pt,a4paper,hyphens]{report}
\usepackage{color}
\usepackage{caption}
\usepackage{translit}
\usepackage{amsmath}
% dasselbe Template wie Thesis mit nur leichten Anpassungen
% Nehmen Sie das Thesis-Template für die Thesis!
% Lesen Sie Hinweise zum Umgang mit LaTeX und zum Schreiben
% von Berichten im Thesis-Template nach
% => Moodle => PraxissemesterThesis => LaTeXThesis.zip
%    https://moodle.hs-mannheim.de/course/view.php?id=2500


% Für doppelseitigen Ausdruck (nur bei > 60 Seiten sinnvoll)
% \usepackage{ifthen}
% \setboolean{@twoside}{true}
% \setboolean{@openright}{true} 

% pakete
\usepackage{ifthen}

% Deutsch
\usepackage[german]{babel} % deutsch und deutsche Rechtschreibung
\usepackage[backend=biber, style=numeric, sorting=none]{biblatex} % Literaturverzeichnis (sortiert nach Reihenfolge des Auftretens)
\addbibresource{praksem.bib}
\usepackage[utf8]{inputenc} % Unicode Text
\usepackage[T1]{fontenc} % Umlaute und deutsches Trennen
\usepackage{textcomp} % Euro
% statt immer Ab\-schluss\-ar\-beit zu schreiben
% einfach hier sammeln mit -. 
\hyphenation{Ab-schluss-ar-beit}
% Vorsicht bei Umlauten und Bindestrichen
\hyphenation{Ver-st\"ar-ker-aus-gang}
 % eigene Hyphenations, die für das Dokument gelten
\usepackage{amssymb} % Symbole
\usepackage{emptypage} % Wirklich leer bei leeren Seiten

%% Fonts, je ein kompletter Satz an Optionen

% Times New Roman, gewohnter Font, ok tt und serifenlos
%\usepackage{mathptmx} 
%\usepackage[scaled=.95]{helvet}
%\usepackage{courier}

% Palatino mit guten Fonts für tt und serifenlos
\usepackage{mathpazo} % Palatino, mal was anderes
\usepackage[scaled=.95]{helvet}
\usepackage{courier}

% New Century Schoolbook sieht auch nett aus (macht auch tt und serifenlos)
%\usepackage{newcent}

% Oder default serifenlos mit Helvetica 
% ich kann es nicht mehr sehen ...
%\renewcommand{\familydefault}{\sfdefault}

% ein bisschen eine bessere Verteilung der Buchstaben...
\usepackage{microtype}

% Bilder und Listings
\usepackage{graphicx} % wir wollen Bilder einfügen
\usepackage{subfig} % Teilbilder
\usepackage{wrapfig} % vielleicht doch besser vermeiden
\usepackage{listings} % schöne Quellcode-Listings
% ein paar Einstellungen für akzeptable Listings
\lstset{basicstyle=\ttfamily, columns=[l]flexible, mathescape=true, showstringspaces=false, numbers=left, numberstyle=\tiny}
\lstset{language=python} % und nur schöne Programmiersprachen ;-)
% und eine eigene Umgebung für Listings
\usepackage{float}
\newfloat{listing}{htbp}{scl}[chapter]
\floatname{listing}{Listing}

% Seitenlayout
\newcommand{\seitenseitenabstand}{22mm} % 30mm
\usepackage[paper=a4paper,left=\seitenseitenabstand,right=\seitenseitenabstand,height=23cm]{geometry}
\usepackage{setspace}
% \linespread{1.15}
\linespread{1.1}
\setlength{\parskip}{0.5em}
\setlength{\parindent}{0em} % im Deutschen Einrückung nicht üblich, leider

% Seitenmarkierungen 
\newcommand{\phv}{\fontfamily{phv}\fontseries{m}\fontsize{9}{11}\selectfont}
\usepackage{fancyhdr} % Schickere Header und Footer
\pagestyle{fancy}
\renewcommand{\chaptermark}[1]{\markboth{#1}{}}
%\fancyhead[L]{\phv \leftmark}
\fancyhead[RH,LO]{\phv \nouppercase{\leftmark}}
\fancyhead[LH,RO]{\phv \thepage}
% Unten besser auf alles Verzichten
%\fancyfoot[L]{\textsf{\small \kurztitel}}
\fancyfoot[C]{\ } % keine Seitenzahl unten
%\fancyfoot[R]{\textsf{\small Technische Informatik}}

% Include subsubsections in the Table of Contents
% -1 for part
%  0 for chapter (only in report and book document classes)
%  1 for section
%  2 for subsection
%  3 for subsubsection
%  4 for paragraph
%  5 for subparagraph
\setcounter{tocdepth}{3}

% Theorem-Umgebungen
\newtheorem{definition}{Definition}[chapter]
\newtheorem{satz}{Satz}[chapter]
\newtheorem{lemma}[satz]{Lemma} % gleicher Zähler wie Satz
\newtheorem{theorem}{Theorem}[chapter]
\newenvironment{beweis}[1][Beweis]{\begin{trivlist}
                                       \item[\hskip \labelsep {\textit{#1 }}]}{
\end{trivlist}}
\newcommand{\qed}{\hfill \ensuremath{\square}}

%% Quellen
% Eine Alternative wäre Quellen in Literatur und Online-Quellen
% zu teilen
% \usepackage{bibtopic} 

% Hochschule Logo, noch nicht perfekt
\usepackage{hsmalogo}

% Spezialpakete
\usepackage{epigraph}
\setlength{\epigraphrule}{0pt} % kein Trennstrich

% damit wir nicht so viel tippen müssen, nur für Demo 
% \usepackage{blindtext}

% ifthen für sperrvermerk
\newif\ifsperrvermerk

% klickbare links im inhaltsverzeichnis
\usepackage{hyperref}
\hypersetup{
    colorlinks,
    citecolor=black,
    filecolor=black,
    linkcolor=black,
    urlcolor=black
}

% text shortcut variablen
\newcommand{{\metaeffekt}}{\{met\ae ffekt\}}
\newcommand{\bitkom}{bitkom}
\newcommand{\aeclientZEZESE}{Thales}
\newcommand{\headerand}{und\ }
% weekdays
\newcommand{\weekdayMondayLong}{Mo} % Mo, Montag
\newcommand{\weekdayTuesdayLong}{Di} % Di, Dienstag
\newcommand{\weekdayWednesdayLong}{Mi} % Mi, Mittwoch
\newcommand{\weekdayThursdayLong}{Do} % Do, Donnerstag
\newcommand{\weekdayFridayLong}{Fr} % Fr, Freitag
\newcommand{\weekdaySaturdayLong}{Sa} % Sa, Samstag
\newcommand{\weekdaySundayLong}{So} % So, Sonntag
\newcommand{\weekdayMondayShort}{Mo}
\newcommand{\weekdayTuesdayShort}{Di}
\newcommand{\weekdayWednesdayShort}{Mi}
\newcommand{\weekdayThursdayShort}{Do}
\newcommand{\weekdayFridayShort}{Fr}
\newcommand{\weekdaySaturdayShort}{Sa}
\newcommand{\weekdaySundayShort}{So}

% custom commands
\newcommand{\lweekdaymarginpar}[1]{%
    \marginpar{\raisebox{-1.6em}{\underline{#1}}}
}
\newcommand{\sweekdaymarginpar}[1]{%
    \marginpar{\raisebox{-1.92em}{\underline{#1}}}
}
\newcommand{\qt}[1]{„#1“}

\definecolor{codegray}{gray}{0.9}
\newcommand{\code}[1]{\colorbox{codegray}{\texttt{\detokenize{#1}}}}
\newcommand{\codendt}[1]{\colorbox{codegray}{\texttt{#1}}}

% custom environments
\newenvironment{smitemize}
{ \begin{itemize}
      \setlength{\itemsep}{0em}
      \setlength{\topsep}{0em}
      \setlength{\partopsep}{0em} }
      {
\end{itemize} }

% code listings
\input{latex-listings-powershell/src/latex-listings-powershell}
\definecolor{lst-gray}{rgb}{0.98,0.98,0.98}
\definecolor{lst-blue}{RGB}{40,0.0,255}
\definecolor{lst-green}{RGB}{65,128,95}
\definecolor{lst-red}{RGB}{200,0,85}
\lstset{
    commentstyle=\color{lst-green},
    basicstyle=\small\ttfamily,
    backgroundcolor=\color{lst-gray},
    breaklines=true,
    captionpos=b,
    columns=fixed,
    extendedchars=true,
    frame=single,
    framesep=2pt,
    keepspaces=true,
    keywordstyle=\color{lst-blue},
    language={PowerShell},
    numbers=left,
    numberstyle=\small\ttfamily,
    showstringspaces=false,
    stringstyle=\color{lst-red},
    tabsize=2,
}
 % alle Pakete und Einstellungen

% Hier anpassen 
\newcommand{\autor}{Yan Wittmann}
\newcommand{\matrikelnummer}{2121578}
\newcommand{\fachsemester}{5IB} % im wie vielten Semester waren Sie?
%\newcommand{\studiengang}{Medizintechnik}
%\newcommand{\studiengang}{Technische Informatik}
\newcommand{\studiengang}{Informationstechnik}
\newcommand{\firma}{\metaeffekt\ GmbH}
\newcommand{\standort}{Heidelberg}
\newcommand{\abteilung}{Automated Vulnerability Monitoring}
\newcommand{\betreuer}{Karsten Klein}
\newcommand{\pbeginn}{01.09.2023}
\newcommand{\pende}{29.02.2024}
\newcommand{\tage}{124} % arbeitstagerechner verwenden!
\newcommand{\titel}{Bericht zum praktischen Studiensemester}
\newcommand{\kurztitel}{Praxisbericht}
% \sperrvermerktrue % Kommentar am Anfang der Zeile löschen für Sperrvermerk

% entweder der vollständige oder der gekürzte Bericht
\newif\ifshortenedReport
\shortenedReporttrue

% Wenn jemand unbedingt ein Glossar will, die nächsten drei Zeilen...
%\usepackage{glossaries} % oder schlimmer mit [toc], damit es im TOC auftaucht
%\makeglossaries
%\newglossaryentry{Computer}{name=Computer,
    description={Eine programmierbare Maschine, die Eingaben erhält, Daten speichert und manipuliert
    und Ausgaben in einem sinnvollem Format ausgibt. (Und wer so was in ein Glossar eines Berichts
    für einen technischen Studiengang schreibt hat es nicht verstanden)}}
\newglossaryentry{naiv}{name=na\"{\i}ve,
description={Ein franzöisches Lehnswort (Adjektiv, Form von naïf)
Erweckt den Eindruch oder hat mangelnde Erfahrung, mangelndes Verständnis oder mangelnde Skills}}
\newglossaryentry{Linux}{name=Linux,
description={Generischer Ausdruck für eine Familie von Unix-artigen Betriebssystemen die den
Linux-Kernel verwenden},
plural=Linuces}
\newacronym[longplural={Frames per Second}]{fpsLabel}{FPS}{Frame per Second}
\newacronym{acme}{ACME}{A Company Making Everything}
\newglossaryentry{Praktisches Studiensemester}{name=Praktisches Studiensemester,
description={Im Rahmen des Ingenieursstudiums ein Semester in der betrieblichen Praxis
zur Ergänzung und Vertiefung des Studienwissens durch selbstständige ingenieurnahe Tätigkeit
betreut durch einen Ingenieur des Betriebes}}
 % In dieser Datei die Einträge definieren
% und noch ganz unten printglossaries auskommentieren
% Damit jetzt ein Glossar gezeigt wird noch \gls{label} verwenden

\begin{document}
    \begin{titlepage}
    \hsmalogo[1] \hfill
    \parbox[b]{60mm}{
        Fakultät Informationstechnik\\
        Studiengang \studiengang}
    \begin{center}
        \rule{1\textwidth}{1pt}\\[-3mm]
        \parbox[t][64mm]{110mm}{% 11 cm für Breite 13, ca. 7 für Höhe 6
            \Large{\ } \\[8mm]
            Bericht zum praktischen Studiensemester \\[4mm]
            \begin{tabular}{rl}
                \large{Vorgelegt von} & \large{\autor} \\[2mm]
                \large{Studiengang}   & \large{\studiengang} \\[2mm]
                \large{Firma}         & \large{\firma} \\[2mm]
            \end{tabular}
        }
        \rule{\textwidth}{1pt}
        \vfill
    \end{center}

    \vspace{2em}
    \begin{tabular}{ll}
        Name               & \autor                   \\
        Matrikelnummer     & \matrikelnummer          \\
        Studiengang        & \studiengang             \\
        Fachsemester       & \fachsemester \\[8mm]

        Praktikumszeitraum & \pbeginn\ \ bis \ \pende \\
        Präsenztage        & \tage \\[8mm]

        Firma              & \firma                   \\
        Standort           & \standort                \\
        Abteilung          & \abteilung               \\
        Betreuer           & \betreuer                \\
    \end{tabular}

    \vspace{8em}
    \noindent\begin{tabular}{p{0.48\textwidth}p{0.48\textwidth}}
                 \rule{0.46\textwidth}{0.5pt} & \rule{0.46\textwidth}{0.5pt} \\
                 Datum, \betreuer             & Firmenstempel
    \end{tabular}
    \vfill
\end{titlepage}
\cleardoublepage


% Erklärung gemäß der Prüfungsordnung
\thispagestyle{empty}
\subsection*{Selbstständigkeitserklärung}

Ich versichere, dass ich diesen Bericht zum praktischen Studiensemester
selbstständig und nur unter Verwendung der angegebenen Quellen und
Hilfsmittel angefertigt habe.
Die Stellen, an denen Inhalte aus den Quellen verwendet wurden, sind
als solche eindeutig gekennzeichnet.
Die Arbeit hat in gleicher oder ähnlicher Form bei keinem anderen
Prüfungsverfahren vorgelegen.

\vspace{6em}
\noindent\begin{tabular}{p{0.48\textwidth}p{0.48\textwidth}}
             \rule{0.42\textwidth}{0.5pt} & \rule{0.48\textwidth}{0.5pt} \\
             Datum, Ort                   & \makebox[1cm]{\ } \autor
\end{tabular}

\vfill

\ifsperrvermerk
\subsection*{Sperrvermerk}

Der vorliegende Bericht enthält interne und teilweise vertrauliche Daten
der Firma \newline \firma.
Der Bericht darf daher zu keinen anderen als Prüfungszwecken verwendet werden.
Insbesondere ist die Vervielfältigung und Veröffentlichung von Berichtinhalten
oder Teilen davon nur mit Zustimmung des Unternehmens erlaubt.
\vfill
\fi

\cleardoublepage

 % Titelseite, Erklärungen, etc.

    \begin{abstract}

        Im Rahmen des praktischen Studiensemesters bei der {\metaeffekt} GmbH in Heidelberg fokussierte sich Yan Wittmann auf die Weiterentwicklung des automatisierten Vulnerability Monitorings.
        Die vorhandene Softwarelösung konnte durch die Integration des neuen CVSS 4.0 Standards, die Überarbeitung des internen Datenmodells und die Implementierung der CVSS-Versionen in TypeScript, verbessert werden.
        Die Verbesserungen sowohl in Kundenprojekte integriert, als auch als Beitrag zur Open-Source-Community veröffentlicht.
        Die aktive Teilnahme des Praktikanten an Fachveranstaltungen trug zudem zum Austausch und zur Vernetzung im Bereich der IT-Sicherheit bei.

    \end{abstract}

    \tableofcontents

    \ifshortenedReport
    \vspace*{20px}
    \fbox{
        \begin{minipage}{1\textwidth}
            \textbf{Hinweis:}
            Dieser Bericht wurde aus Gründen der Übersichtlichkeit und der maximalen Seitenzahl gekürzt.
            Der vollständige Bericht ist alternativ auf Anfrage verfügbar.
        \end{minipage}
    }
    \fi


    %! Author = Yan Wittmann


\chapter{Firmenumfeld und Zielsetzung} \label{ch:firmenumfeld-zielsetzung}


\section{Firma} \label{sec:firma-beschreibung}

Die {\metaeffekt} GmbH ist ein spezialisiertes Unternehmen, das sich auf die kontinuierliche Inventarisierung, Dokumentation und Bewertung von Risiken in Softwarebestandteilen von Produkten und Projekten konzentriert, um Unternehmen bei der Software Composition Analysis zu unterstützen.
Die {\metaeffekt} arbeitet eng mit verschiedenen Fachbereichen und Verantwortlichen der Kunden zusammen, um maßgeschneiderte Lösungen anzubieten.
Darüber hinaus bietet die {\metaeffekt} Schulungen und Seminare zu Themen wie License Compliance Awareness, License Compliance Management, Vulnerability Monitoring und Vulnerability Assessment an.

Die {\metaeffekt} hebt sich durch einen Fokus auf die tatsächlich genutzten Softwarebestandteile im Gegensatz zu einem spezifikationsbasierten Scannen von anderen Tools und Unternehmen ab.
Durch den Einsatz fallspezifischer Werkzeuge und Beratung erreicht das Unternehmen eine hohe Datenqualität, die für die Umsetzung der License Compliance in der Lieferkette und für die Identifikation sowie Überwachung von öffentlichen Schwachstellen wichtig ist.

Die Abteilung \qt{Vulnerability Monitoring}, in der das Praktikum stattfand, entwickelt Werkzeuge und Prozesse, um die automatisierte kontinuierliche Überwachung von Schwachstellen in Softwareprodukten durch die erfassten Software-Inventare zu ermöglichen.

Es gibt andere Organisationen, die Teile der Dienstleistungen von {\metaeffekt} anbieten.
Dazu gehören Unternehmen wie
Black Duck (Sicherheits- und Lizenzanforderungsanalyse von Open Source-Komponenten),
WhiteSource (Lizenzanforderungsanalyse von Open Source-Komponenten und Compliance),
Snyk (Identifizierung und Behebung von Sicherheitslücken in Open Source-Komponenten) und
Sonatype (Identifizierung und Behebung von Sicherheitslücken und Lizenzproblemen in Open Source-Komponenten).


\section{Zielsetzung} \label{sec:zielsetzung-des-praktikums}

Das Ziel des Praktikums war es, die größtenteils von mir zuvor entwickelte Softwarelösung für die automatisierte Überwachung von Schwachstellen in Softwareprodukten bei den Kunden zu betreuen, zu verbessern und zu erweitern.
Spezifische Ziele haben sich während den ersten Wochen herausgebildet.

Das erste Ziel sollte das Ersetzen der bisherigen \qt{Generation 2} unseres Vulnerability Monitoring durch eine \qt{Generation 3} sein.
Dies beinhaltet sowohl die Implementierung des Standards CVSS in der Version 4.0 in Java, als auch eine komplette Überarbeitung unseres Datenmodells hinter dem Schwachstellen-Management.
Die Integration all dieser Änderungen in unsere Systeme und Reports soll auch Teil davon sein.

Das zweite Ziel war es, eine TypeScript-Implementierung aller aktuellen CVSS-Versionen zu schreiben, die in einer Web-UI als Online-Rechner zur Verfügung gestellt wird.

Weiter gibt es neben den Entwicklungsaufgaben auch Integrations- und Konfigurationsarbeiten in Kundenprojekten und allgemein die Beratung und Unterstützung der Kunden.


    \includeReport{wochenberichte/shortened/wochenberichte}{wochenberichte/initial/wochenberichte}

    %! Author = Yan Wittmann


\chapter{Projektbericht} \label{ch:projektbericht}

In der Einleitung (Kapitel\ \ref{sec:projektbericht-projektziel}) wird zunächst auf die Abteilung im Unternehmen, die aktuelle Situation und die Ziele für das Projektsemester eingegangen.
Bei den darauf folgenden Grundlagen (Kapitel\ \ref{sec:projektbericht-grundlagen}) werden die relevanten Themen, Begriffe und Standards erklärt.

\section{Einleitung, Problemstellungen und Projektziel} \label{sec:projektbericht-projektziel}

Die für dieses Praktikum relevante Abteilung in der {\metaeffekt} stellt ein automatisiertes Vulnerability Monitoring her und integriert es bei diversen Kunden, deren Wünsche und Anforderungen priorisiert in die Systeme zurückgeführt werden.
Das Vulnerability Monitoring wird intern durch eine Aneinanderreihung von Prozessschritten modelliert, die auch als \qt{Inventory Enrichment Pipeline} bezeichnet wird.
Die Prozessschritte erhalten jeweils ein Software-Inventar als Eingabe, welches sie auf eine bestimmte definierte Weise modifizieren und für den nächsten Schritt bereitstellen, sodass ein Inventar, das alle Schritte durchlaufen hat, alle nötigen Schwachstell-Informationen angereichert bekommen hat.
Auf diese Pipeline wird im Kapitel\ \ref{subsec:projektbericht-grundlagen-vulnerability-monitoring} kurz eingegangen, sie soll allerdings nicht Hauptbestandteil dieses Berichts sein und dient nur zur besseren Einordnung der anderen Prozesse.

In dem Praktikum liegt der Schwerpunkt vor allem auf dem CVSS-Standard\textsuperscript{\ref{subsec:projektbericht-grundlagen-cvss}}, vor allem die Verbesserung unseres Supports für neuere Versionen dieses und konzeptionelle Änderungen, wie wir mit diesen umgehen wollen.
Konkret geht es um die folgenden Punkte:

\begin{smitemize}
    \item Die Veröffentlichung der neuesten Version 4.0 des CVSS-Standards am 31.\ Oktober 2023\footnote{\url{https://www.first.org/cvss/v4-0/}} wird über die nächsten Monate zur Folge haben, dass CVSS-Vektoren der Version 4.0 für Schwachstellen in den öffentlichen Datenbanken auftauchen werden, die wir in der Lage sein müssen, zu parsen und zu berechnen.
    Es ist dazu also notwendig, die aktuelle Implementierung der Versionen 2.0 und 3.1 um die vierte zu erweitern.
    Jedoch reicht hier nicht einfach die Implementierung der Berechnungslogik, um diesen Task abzuschließen, sie muss auch noch in die vorhandenen Systeme integriert und die Theorie dahinter verstanden werden, damit man gegenüber Kunden aussagekräftig die Unterschiede und die Vorteile begründen kann.
    \item Über die Monate vor dem Praktikum ist es bereits immer deutlicher geworden, dass das Datenmodell hinter dem Vulnerability Monitoring fast komplett neu geschrieben werden muss, um neue Anforderungen und Erkenntnisse effizient und korrekt unterstützen zu können.
    Die Anforderung an das Datenmodell, die hier relevant ist, ist eine bessere Art, die CVSS-Vektoren abzulegen und zu verarbeiten.
    Es wurde erkannt, dass meistens nicht nur ein CVSS-Vektor einer Quelle pro Schwachstelle (CVE, \ldots) vorhanden ist, sondern mehrere, die von mehreren Organisationen und Institutionen vergeben werden, da ihre Meinungen über den Schweregrad voneinander abweichen können.
    Bisher wird mit diesen zusätzlichen Vektoren nicht bewusst unterschiedlich umgegangen, es wird einfach der erste verarbeitet, der vorhanden ist.
    Um diese Situation zu verbessern, soll ein System eingeführt werden, das über die einzelnen Schritte der Inventory Enrichment Pipeline nur die Vektoren aggregiert und noch nicht verarbeitet oder berechnet.
    Erst zum Ende, wenn es darum geht die Reports (PDF, HTML) zu generieren, sollen die Vektoren ausgewählt, eventuell kombiniert und deren Scores berechnet werden.
    Bei dem Überarbeiten des Datenmodells muss also darauf geachtet werden, diese Anforderung zu unterstützen.
    Zudem muss ein CVSS-Selektor geschrieben werden, der die darzustellenden Vektoren berechnen kann.
    \item Zuletzt ging es noch um die Implementierung des CVSS-Standards in TypeScript, die Open Source gestellt werden sollte und mit einem Web-UI als online verfügbarer, interaktiver CVSS-Rechners verfügbar sein soll.
    Dieser soll dann aus den eigenen Reports verlinkt werden können.
    Der Grund hierfür ist simpel: Es gibt bisher keinen online CVSS-Rechner, der alle unsere Anforderungen erfüllt.
    Es gibt keinen Rechner, der alle Versionen zugleich unterstützt, keinen, der mehrere Vektoren gleichzeitig gut vergleichbar zulässt und leider haben viele der offiziellen Rechner auch Probleme, die URL-Parameter korrekt zu erkennen.
\end{smitemize}

In diesem Bericht wird ein Fokus auf die CVSS-seitigen Arbeiten gelegt, da sie den Großteil des Semesters eingenommen haben.

\section{Grundlagen} \label{sec:projektbericht-grundlagen}

\subsection{Software-Inventare} \label{subsec:projektbericht-grundlagen-inventories}

\subsection{NVD / NIST} \label{subsec:projektbericht-grundlagen-nvd-nist}

\subsection{CVE / CPE} \label{subsec:projektbericht-grundlagen-cve-cpe}

\subsection{CPE Derivation} \label{subsec:projektbericht-grundlagen-cpe-derivation}

\subsection{CVSS} \label{subsec:projektbericht-grundlagen-cvss}

\subsection{Vulnerability Assessment} \label{subsec:projektbericht-grundlagen-vulnerability-assessment}

\subsection{Automatisiertes Vulnerability Monitoring} \label{subsec:projektbericht-grundlagen-vulnerability-monitoring}

Die relevanten Prozessschritte sind das automatische Zuordnen von CPEs zu den Produkten aus den Kundeninventaren (CPE Derivation\textsuperscript{\ref{subsec:projektbericht-grundlagen-cpe-derivation}})


    %! Author = Yan Wittmann

\chapter{Ergebnisse} \label{ch:ergebnisse}

In dem Praxissemester wurden unter anderem die folgenden Haupt-Ergebnisse erzielt:

\begin{itemize}
    \item Implementierung des CVSS 4.0 Standards in die vorhandene Softwarelösung und Integration in den Daten-Mirror und die Reporting-Tools.
    \item Überarbeitung des Datenmodells zur Unterstützung von mehreren CVSS-Vektoren je Schwachstelle, verbessertem Tracking der Quellen von Schwachstellen und der Generalisierung der Datenzugriffe um Redundanzen in den Implementierungen zu vermeiden.
    \item Implementierung der CVSS-Versionen 2.0, 3.0, 3.1 und 4.0 in TypeScript.
    \item Entwickeln eines online-CVSS-Rechners, der aus den eigenen Reporting-Tools referenziert wird und bereits bei verschiedenen Events vorgestellt wurde.
    \item Integration der durchgeführten Änderungen in die entsprechenden Kundenprojekte an den Kundensystemen und im konstanten Dialog mit den Zuständigen.
    \item Besuchen des Arbeitskreises OpenSource des {\bitkom} und halten eines Vortrages vor anderen Firmen.
    \item Besuchen des Open Source Forums des {\bitkom} in Erfurt.
    \item Besuchen des zweitägigen CSAF-Workshops in München im ISH (Information Security Hub), durchgeführt vom BSI\@.
\end{itemize}

Die folgenden konkreten öffentliche Ergebnisartefakte wurden dabei unter anderem erzeugt (mit Links hinterlegt):

\begin{itemize}[noitemsep]
    \item \href{https://www.metaeffekt.com/security/cvss/calculator}{Universal CVSS Calculator}
    \item \href{https://github.com/org-metaeffekt/metaeffekt-universal-cvss-calculator}{GitHub Repository der TypeScript Implementierung}
    \item \href{https://youtu.be/R2S0_6-NQGQ?si=d7zpxbJ7P4R26nRb&t=2801}{Webinar zum CVSS Calculator}
    \item \href{https://mvnrepository.com/artifact/com.metaeffekt.artifact.analysis/ae-artifact-analysis}{Weiterführung des aritfact-analysis Projektes}
    \item \href{https://github.com/org-metaeffekt/metaeffekt-core}{Weiterführung des Core Projektes}
\end{itemize}


    %! Author = Yan Wittmann

\chapter{Ausblick} \label{ch:ausblick}

Meine Mitarbeit bei der {\metaeffekt} wird nach dem Praxissemester weiterhin fortgeführt.
In diesem Zusammenhang werde ich mit den Systemen, Kunden und Kollegen auch weiterhin in Kontakt bleiben und die Entwicklungen begleiten.

Konkret werden in der nächsten Zeit die folgenden Punkte relevant:

\begin{itemize}
    \item Integration von KEV (Known Exploited Vulnerabilities) in die bestehenden Systeme.
    \item Integration von CSAF (Common Security Advisory Framework) als Advisory-Quelle und Support für Export von CSAF-Dateien aus gescannten Software-Inventaren.
    \item Besuch des KIT, um mit einem Professor über das Thema Vulnerability Chaining zu sprechen, welches in nächster Zeit bei uns mit einem eigenen Konzept integriert werden soll.
    \item Neuschreiben des VAD (Vulnerability Assessment Dashboards) mit TypeScript.
    \item Erstellen eines \qt{Overview Reports} in HTML und TypeScript, für eine Übersicht über mehrere Kontexte.
    \item Um unsere Systeme mehr eine Art Tooling-Chain bauen, um den Einstieg in die Verwendung unserer Tools zu erleichtern.
    \item Präsentieren unseres Toolings auf der großen Bühne beim Open Source Forum des {\bitkom} in Erfurt im September.
    \item Erstellung eines Test-Frameworks für das Vulnerability Monitoring.
\end{itemize}


    \newpage

% Listen wenn überhaupt ans Ende und nicht an den Anfang.
% Meist ist das aber unnötig.
    \listoffigures % Liste der Abbildungen
%\begingroup % aahh nicht noch ein pagebreak
%\let\clearpage\relax %
    % \listoftables % Liste der Tabellen
%\endgroup

% Glossar kommt auch ans Ende
%\glsaddall % das fügt alle Glossar-Einträge ein
%\printglossaries % nicht vergessen "makeglossaries praksem" aufzurufen
%\gls{Computer}
%\newpage

    \addcontentsline{toc}{chapter}{Literaturverzeichnis}
    %\bibliographystyle{plain}
    %\bibliography{praksem}
    \printbibliography
% \bibliography{praksem,online} # wenn man zwei Dateien hätte

% Das wäre die Alternative mit geteilten Quellen (preamble muss auch
% angepasst werden) und die Literatur muss in die Datei praksem.bib
% und die Online-Quellen müssen in die Datei online.bib.
%\begin{btSect}{praksem} % mit bibtopic Quellen trennen
%\section*{Literaturverzeichnis}
%\addcontentsline{toc}{chapter}{Literaturverzeichnis}
%\btPrintCited
%\end{btSect}
%\begin{btSect}{online}
%\section*{Online-Quellen}
%\addcontentsline{toc}{chapter}{Online-Quellen}
%\btPrintCited
%\end{btSect}
% dann ab und zu "bibtex praksem1" und "bibtex praksem2" aufrufen

\end{document}
;;; Local Variables:
;;; ispell-local-dictionary: "de_DE-neu"
;;; End:
