\documentclass[11pt,a4paper,hyphens]{report}
\usepackage{color}
\usepackage{caption}
\usepackage{translit}
\usepackage{amsmath}
% dasselbe Template wie Thesis mit nur leichten Anpassungen
% Nehmen Sie das Thesis-Template für die Thesis!
% Lesen Sie Hinweise zum Umgang mit LaTeX und zum Schreiben
% von Berichten im Thesis-Template nach
% => Moodle => PraxissemesterThesis => LaTeXThesis.zip
%    https://moodle.hs-mannheim.de/course/view.php?id=2500


% Für doppelseitigen Ausdruck (nur bei > 60 Seiten sinnvoll)
% \usepackage{ifthen}
% \setboolean{@twoside}{true}
% \setboolean{@openright}{true} 

% pakete
\usepackage{ifthen}

% Deutsch
\usepackage[german]{babel} % deutsch und deutsche Rechtschreibung
\usepackage[backend=biber, style=numeric, sorting=none]{biblatex} % Literaturverzeichnis (sortiert nach Reihenfolge des Auftretens)
\addbibresource{praksem.bib}
\usepackage[utf8]{inputenc} % Unicode Text
\usepackage[T1]{fontenc} % Umlaute und deutsches Trennen
\usepackage{textcomp} % Euro
\input{hyphenations} % eigene Hyphenations, die für das Dokument gelten
\usepackage{amssymb} % Symbole
\usepackage{emptypage} % Wirklich leer bei leeren Seiten

%% Fonts, je ein kompletter Satz an Optionen

% Times New Roman, gewohnter Font, ok tt und serifenlos
%\usepackage{mathptmx} 
%\usepackage[scaled=.95]{helvet}
%\usepackage{courier}

% Palatino mit guten Fonts für tt und serifenlos
\usepackage{mathpazo} % Palatino, mal was anderes
\usepackage[scaled=.95]{helvet}
\usepackage{courier}

% New Century Schoolbook sieht auch nett aus (macht auch tt und serifenlos)
%\usepackage{newcent}

% Oder default serifenlos mit Helvetica 
% ich kann es nicht mehr sehen ...
%\renewcommand{\familydefault}{\sfdefault}

% ein bisschen eine bessere Verteilung der Buchstaben...
\usepackage{microtype}

% Bilder und Listings
\usepackage{graphicx} % wir wollen Bilder einfügen
\usepackage{subfig} % Teilbilder
\usepackage{wrapfig} % vielleicht doch besser vermeiden
\usepackage{listings} % schöne Quellcode-Listings
% ein paar Einstellungen für akzeptable Listings
\lstset{basicstyle=\ttfamily, columns=[l]flexible, mathescape=true, showstringspaces=false, numbers=left, numberstyle=\tiny}
\lstset{language=python} % und nur schöne Programmiersprachen ;-)
% und eine eigene Umgebung für Listings
\usepackage{float}
\newfloat{listing}{htbp}{scl}[chapter]
\floatname{listing}{Listing}
\usepackage{enumitem} % für bessere Listen
\usepackage{seqsplit} % für den automatischen Zeilenumbruch von langen Zeichenketten

% Seitenlayout
\newcommand{\seitenseitenabstand}{22mm} % 30mm
\usepackage[paper=a4paper,left=\seitenseitenabstand,right=\seitenseitenabstand,height=23cm]{geometry}
\usepackage{setspace}
% \linespread{1.15}
\linespread{1.1}
\setlength{\parskip}{0.5em}
\setlength{\parindent}{0em} % im Deutschen Einrückung nicht üblich, leider

% Seitenmarkierungen 
\newcommand{\phv}{\fontfamily{phv}\fontseries{m}\fontsize{9}{11}\selectfont}
\usepackage{fancyhdr}
\pagestyle{fancy}
\fancyhf{} % initialize: clear all header and footer fields
\fancyhead[L]{\phv Yan Wittmann}
\fancyhead[C]{\phv Bericht zum praktischen Studiensemester}
\fancyhead[R]{\phv \thepage} % page number on the right
\renewcommand{\headrulewidth}{0.4pt} % line under the header
\fancyfoot{} % clears the footer

% Include subsubsections in the Table of Contents
% -1 for part
%  0 for chapter (only in report and book document classes)
%  1 for section
%  2 for subsection
%  3 for subsubsection
%  4 for paragraph
%  5 for subparagraph
\setcounter{tocdepth}{3}

% Theorem-Umgebungen
\newtheorem{definition}{Definition}[chapter]
\newtheorem{satz}{Satz}[chapter]
\newtheorem{lemma}[satz]{Lemma} % gleicher Zähler wie Satz
\newtheorem{theorem}{Theorem}[chapter]
\newenvironment{beweis}[1][Beweis]{\begin{trivlist}
                                       \item[\hskip \labelsep {\textit{#1 }}]}{
\end{trivlist}}
\newcommand{\qed}{\hfill \ensuremath{\square}}

%% Quellen
% Eine Alternative wäre Quellen in Literatur und Online-Quellen
% zu teilen
% \usepackage{bibtopic} 

% Hochschule Logo, noch nicht perfekt
\usepackage{hsmalogo}

% Spezialpakete
\usepackage{epigraph}
\setlength{\epigraphrule}{0pt} % kein Trennstrich

% damit wir nicht so viel tippen müssen, nur für Demo 
% \usepackage{blindtext}

% ifthen für sperrvermerk
\newif\ifsperrvermerk

% klickbare links im inhaltsverzeichnis
\usepackage{hyperref}
\hypersetup{
    colorlinks,
    citecolor=black,
    filecolor=black,
    linkcolor=black,
    urlcolor=black
}

% text shortcut variablen
\newcommand{{\metaeffekt}}{\{met\ae ffekt\}}
\newcommand{\bitkom}{bitkom}
\newcommand{\aeclientZEZESE}{Thales}
\newcommand{\headerand}{und\ }
% weekdays
\newcommand{\weekdayMondayLong}{Mo} % Mo, Montag
\newcommand{\weekdayTuesdayLong}{Di} % Di, Dienstag
\newcommand{\weekdayWednesdayLong}{Mi} % Mi, Mittwoch
\newcommand{\weekdayThursdayLong}{Do} % Do, Donnerstag
\newcommand{\weekdayFridayLong}{Fr} % Fr, Freitag
\newcommand{\weekdaySaturdayLong}{Sa} % Sa, Samstag
\newcommand{\weekdaySundayLong}{So} % So, Sonntag
\newcommand{\weekdayMondayShort}{Mo}
\newcommand{\weekdayTuesdayShort}{Di}
\newcommand{\weekdayWednesdayShort}{Mi}
\newcommand{\weekdayThursdayShort}{Do}
\newcommand{\weekdayFridayShort}{Fr}
\newcommand{\weekdaySaturdayShort}{Sa}
\newcommand{\weekdaySundayShort}{So}

% custom commands
\newcommand{\lweekdaymarginpar}[1]{%
    \marginpar{\raisebox{-1.6em}{\underline{#1}}}
}
\newcommand{\sweekdaymarginpar}[1]{%
    \marginpar{\raisebox{-1.92em}{\underline{#1}}}
}
\newcommand{\qt}[1]{„#1“}

\definecolor{codegray}{gray}{0.9}
\newcommand{\code}[1]{\colorbox{codegray}{\texttt{\detokenize{#1}}}}
\newcommand{\codendt}[1]{\colorbox{codegray}{\texttt{#1}}}

% conditional commands
\newcommand{\includeReport}[2]{
    \ifshortenedReport
    \input{#1} % Inkludiert den ersten Parameter, wenn shortenedReport wahr ist
    \else
    \input{#2} % Inkludiert den zweiten Parameter, wenn shortenedReport falsch ist
    \fi
}

% custom environments
\newenvironment{smitemize}
{ \begin{itemize}
      \setlength{\itemsep}{0em}
      \setlength{\topsep}{0em}
      \setlength{\partopsep}{0em} }
      {
\end{itemize} }

% code listings
\input{latex-listings-powershell/src/latex-listings-powershell}
\definecolor{lst-gray}{rgb}{0.98,0.98,0.98}
\definecolor{lst-blue}{RGB}{40,0.0,255}
\definecolor{lst-green}{RGB}{65,128,95}
\definecolor{lst-red}{RGB}{200,0,85}
\lstset{
    commentstyle=\color{lst-green},
    basicstyle=\small\ttfamily,
    backgroundcolor=\color{lst-gray},
    breaklines=true,
    captionpos=b,
    columns=fixed,
    extendedchars=true,
    frame=single,
    framesep=2pt,
    keepspaces=true,
    keywordstyle=\color{lst-blue},
    language={PowerShell},
    numbers=left,
    numberstyle=\small\ttfamily,
    showstringspaces=false,
    stringstyle=\color{lst-red},
    tabsize=2,
}
 % alle Pakete und Einstellungen

% Hier anpassen 
\newcommand{\autor}{Yan Wittmann}
\newcommand{\matrikelnummer}{2121578}
\newcommand{\fachsemester}{5IB} % im wie vielten Semester waren Sie?
%\newcommand{\studiengang}{Medizintechnik}
%\newcommand{\studiengang}{Technische Informatik}
\newcommand{\studiengang}{Informationstechnik}
\newcommand{\firma}{\metaeffekt\ GmbH}
\newcommand{\standort}{Heidelberg}
\newcommand{\abteilung}{Automated Vulnerability Monitoring}
\newcommand{\betreuer}{Karsten Klein}
\newcommand{\pbeginn}{01.09.2023}
\newcommand{\pende}{29.02.2024}
\newcommand{\tage}{124} % arbeitstagerechner verwenden!
\newcommand{\titel}{Bericht zum praktischen Studiensemester}
\newcommand{\kurztitel}{Praxisbericht}
% \sperrvermerktrue % Kommentar am Anfang der Zeile löschen für Sperrvermerk

% entweder der vollständige oder der gekürzte Bericht
\newif\ifshortenedReport
\shortenedReporttrue

% Wenn jemand unbedingt ein Glossar will, die nächsten drei Zeilen...
%\usepackage{glossaries} % oder schlimmer mit [toc], damit es im TOC auftaucht
%\makeglossaries
%\input{glossar} % In dieser Datei die Einträge definieren
% und noch ganz unten printglossaries auskommentieren
% Damit jetzt ein Glossar gezeigt wird noch \gls{label} verwenden

\begin{document}
    \include{vorspann} % Titelseite, Erklärungen, etc.

    \begin{abstract}

        Im Rahmen des praktischen Studiensemesters bei der {\metaeffekt} GmbH in Heidelberg fokussierte sich Yan Wittmann auf die Weiterentwicklung des automatisierten Vulnerability Monitorings.
        Durch die Integration des neuen CVSS 4.0 Standards, die Überarbeitung des internen Datenmodells und die Implementierung der CVSS-Versionen in TypeScript, konnte die vorhandene Softwarelösung verbessert werden.
        Die Verbesserungen wurden nicht nur in Kundenprojekte integriert, sondern auch als Beitrag zur Open-Source-Community veröffentlicht.
        Die aktive Teilnahme an Fachveranstaltungen trug zudem zum Austausch und zur Vernetzung im Bereich der IT-Sicherheit bei.

    \end{abstract}

    \tableofcontents

    \ifshortenedReport
    \vspace*{20px}
    \fbox{
        \begin{minipage}{1\textwidth}
            \textbf{Hinweis:}
            Dieser Bericht wurde aus Gründen der Übersichtlichkeit und der maximalen Seitenzahl gekürzt.
            Der vollständige Bericht ist alternativ auf Anfrage verfügbar.
        \end{minipage}
    }
    \fi


    %! Author = Yan Wittmann


\chapter{Einleitung} \label{ch:einleitung}


\section{Firma} \label{sec:firma-beschreibung}

Die {\metaeffekt} GmbH ist ein spezialisiertes Unternehmen, das sich auf die kontinuierliche Inventarisierung, Dokumentation und Bewertung von Risiken in Softwarebestandteilen von Produkten und Projekten konzentriert, um Unternehmen bei der Software Composition Analysis zu unterstützen.
Das Unternehmen arbeitet eng mit verschiedenen Fachbereichen und Verantwortlichen zusammen, um maßgeschneiderte Lösungen anzubieten.
Darüber hinaus bietet die {\metaeffekt} Schulungen und Seminare zu Themen wie License Compliance Awareness, License Compliance Management, Vulnerability Monitoring und Vulnerability Assessment.

Die {\metaeffekt} hebt sich durch einen Fokus auf die tatsächlich genutzten Softwarebestandteile im Gegensatz zu einem spezifikationsbasierten Scannen von anderen Tools und Unternehmen ab.
Durch den Einsatz fallspezifischer Werkzeuge und Beratung erreicht das Unternehmen eine hohe Datenqualität, die für die Umsetzung der License Compliance in der Lieferkette und für die Identifikation sowie Überwachung von öffentlichen Schwachstellen wichtig ist.

Die Abteilung \qt{Vulnerability Monitoring}, in der das Praktikum stattfand, entwickelt Tooling und Prozesse, um die automatisierte kontinuierliche Überwachung von Schwachstellen in Softwareprodukten durch die erfassten Software-Inventare zu ermöglichen.

Es gibt mehrere andere Organisationen, die Teile der Dienstleistungen von {\metaeffekt} anbieten.
Dazu gehören Unternehmen wie
Black Duck (Sicherheits- und Lizenzanforderungsanalyse von Open Source-Komponenten),
WhiteSource (Lizenzanforderungsanalyse von Open Source-Komponenten und Compliance),
Snyk (Identifizierung und Behebung von Sicherheitslücken in Open Source-Komponenten) und
Sonatype (Identifizierung und Behebung von Sicherheitslücken und Lizenzproblemen in Open Source-Komponenten).


\section{Zielsetzung} \label{sec:zielsetzung-des-praktikums}

Das Ziel meines Praktikums war es vor allem, die von mir zuvor entwickelte Softwarelösung für die automatisierte Überwachung von Schwachstellen in Softwareprodukten bei den Kunden zu betreuen, zu verbessern und zu erweitern.
Spezifische Ziele haben sich während den ersten Wochen herausgebildet.

Das erste Ziel sollte das Ersetzen der bisherigen \qt{Generation 2} unseres Vulnerability Monitoring durch eine \qt{Generation 3} sein.
Dies beinhaltet nicht nur die Implementierung des Standards CVSS in der Version 4.0 in Java, sondern auch eine komplette Überarbeitung unseres Datenmodells hinter dem Schwachstellen-Management und die Integration all dieser Änderungen in unsere Systeme und Reports, aufgrund von Erkenntnissen, die über die letzten Monate entstanden sind.

Das zweite Ziel war es, eine TypeScript-Implementierung aller aktuellen CVSS-Versionen zu schreiben, die dann in einer Web-UI als Online-Rechner zur Verfügung gestellt werden sollte.

Natürlich gibt es als Aufgabe neben den Entwicklungstasks auch Integrations- und Konfigurationsarbeit bei Kundenprojekten unseres Toolings, genau wie die Beratung und allgemeine Unterstützung der Kunden.


    \includeReport{wochenberichte/shortened/wochenberichte}{wochenberichte/initial/wochenberichte}

    %! Author = Yan Wittmann


\chapter{Projektbericht} \label{ch:projektbericht}

In der Einleitung (Kapitel\ \ref{sec:projektbericht-projektziel}) wird zunächst auf die Abteilung im Unternehmen, die aktuelle Situation und die Ziele für das Projektsemester eingegangen.
Bei den darauf folgenden Grundlagen (Kapitel\ \ref{sec:projektbericht-grundlagen}) werden die relevanten Themen, Begriffe und Standards erklärt.


\section{Einleitung, Problemstellungen und Projektziel} \label{sec:projektbericht-projektziel}

Die für dieses Praktikum relevante Abteilung in der {\metaeffekt} stellt ein automatisiertes Vulnerability Monitoring her und integriert es bei diversen Kunden, deren Wünsche und Anforderungen priorisiert in die Systeme zurückgeführt werden.
Das Vulnerability Monitoring wird intern durch eine Aneinanderreihung von Prozessschritten modelliert, die auch als \qt{Inventory Enrichment Pipeline} bezeichnet wird.
Die Prozessschritte erhalten jeweils ein Software-Inventar als Eingabe, welches sie auf eine bestimmte definierte Weise modifizieren und für den nächsten Schritt bereitstellen, sodass ein Inventar, das alle Schritte durchlaufen hat, alle nötigen Schwachstell-Informationen angereichert bekommen hat.
Auf diese Pipeline wird im Kapitel\ \ref{subsec:projektbericht-grundlagen-vulnerability-monitoring} kurz eingegangen, sie soll allerdings nicht Hauptbestandteil dieses Berichts sein und dient nur zur besseren Einordnung der anderen Prozesse.

In dem Praktikum liegt der Schwerpunkt vor allem auf dem CVSS-Standard\textsuperscript{\ref{subsec:projektbericht-grundlagen-cvss}}, insbesondere in der Verbesserung unseres Supports für neuere Versionen dieses und in konzeptionelle Änderungen, wie das System mit diesen umgehen sollte.
Konkret geht es um die folgenden Punkte:

\begin{smitemize}
    \item Die Veröffentlichung der neuesten Version 4.0 des CVSS-Standards am 31.\ Oktober 2023\footnote{\url{https://www.first.org/cvss/v4-0/}} hat bereits und wird über die nächsten Monate zur Folge haben, dass CVSS-Vektoren der Version 4.0 für Schwachstellen in den öffentlichen Datenbanken auftauchen werden, die die {\metaeffekt} in der Lage sein muss, zu parsen und zu berechnen.
    Es ist dazu also notwendig, die aktuelle Implementierung der Versionen 2.0 und 3.1 um die vierte zu erweitern.
    Zu der Implementierung der Berechnungslogik muss sie auch noch in die vorhandenen Systeme integriert und die Theorie dahinter verstanden werden, auch damit man gegenüber Kunden aussagekräftig die Unterschiede und die Vorteile begründen kann.
    \item Über die Monate vor dem Praktikum ist es bereits immer deutlicher geworden, dass das Datenmodell hinter dem Vulnerability Monitoring fast komplett neu geschrieben werden muss, um neue Anforderungen und Erkenntnisse effizient und korrekt unterstützen zu können.
    Die relevante Anforderung an das Datenmodell ist es, die Art und Weise, wie die CVSS-Vektoren abzulegen und verarbeitet werden, komplett neu zu planen und zu schreiben.
    Es wurde erkannt, dass meistens nicht nur ein CVSS-Vektor einer Quelle pro Schwachstelle (CVE, \ldots) vorhanden ist, sondern mehrere, die von mehreren Organisationen und Institutionen vergeben werden, da ihre Meinungen über den Schweregrad voneinander abweichen können.
    Bisher wird mit diesen zusätzlichen Vektoren nicht bewusst unterschiedlich umgegangen, es wird einfach der erste verarbeitet, der vorhanden ist.
    Um diese Situation zu verbessern, soll ein System eingeführt werden, das über die einzelnen Schritte der Inventory Enrichment Pipeline nur die Vektoren aggregiert und noch nicht verarbeitet oder berechnet.
    Erst zum Ende, wenn es darum geht die Reports (PDF, HTML) zu generieren, sollen die Vektoren ausgewählt, eventuell kombiniert und deren Scores berechnet werden.
    Bei dem Überarbeiten des Datenmodells muss also darauf geachtet werden, diese Anforderung zu unterstützen.
    Zudem muss ein CVSS-Selektor geschrieben werden, der die darzustellenden Vektoren berechnen kann.
    \item Zuletzt ging es noch um die Implementierung des CVSS-Standards in TypeScript, die Open Source gestellt werden sollte und mit einem Web-UI als online verfügbarer, interaktiver CVSS-Rechners verfügbar sein soll.
    Dieser soll dann aus den eigenen Reports verlinkt werden können.
    Der Grund hierfür ist simpel: Es gibt bisher keinen online CVSS-Rechner, der alle unsere Anforderungen erfüllt.
    Es gibt keinen Rechner, der alle Versionen zugleich unterstützt, keinen, der mehrere Vektoren gleichzeitig gut vergleichbar zulässt und leider haben viele der offiziellen Rechner auch Probleme, die URL-Parameter korrekt zu erkennen.
\end{smitemize}

In diesem Bericht wird ein Fokus auf die CVSS-seitigen Arbeiten im Unternehmen gelegt, da sie den Großteil des Semesters eingenommen haben.


% Grundlagen


\section{Grundlagen} \label{sec:projektbericht-grundlagen}

\subsection{Software-Inventare} \label{subsec:projektbericht-grundlagen-inventories}

Software-Inventare werden bei der {\metaeffekt} in einem eigens entwickelten, proprietären Format, schlicht \qt{Inventar} genannt, abgelegt.
Meist wird, um ein solches Inventar zu erhalten, ein ebenfalls eigens entwickelter Scanner verwendet, der ganze Dateisysteme nach Komponenten durchsucht und daraus ein Inventar generiert.
Um möglichst breiten Support zu bieten, gibt es allerdings nicht nur den Scanner, sondern auch diverse Konverter, mit denen Inventare von und zu Formaten wie CycloneDX SBOM\footnote{\url{https://cyclonedx.org/specification/overview}} oder SPDX\footnote{\url{https://spdx.dev}} umgewandelt werden können.

Diese Inventare können mehrere Kategorien an Daten enthalten: Software-Komponenten (\qt{Artefakte}), Schwachstell-Informationen, Security Advisories, Lizenzinformationen und einige weitere.

\subsection{NVD / NIST} \label{subsec:projektbericht-grundlagen-nvd-nist}

Software-Schwachstellen sind allgegenwärtig, jedes Softwareprodukt hat sie und meistens es ist nur eine Frage der Zeit, bis sie gefunden, veröffentlicht und im schlimmsten Fall ausgenutzt werden.
Die National Vulnerability Database (NVD)\footnote{\url{https://nvd.nist.gov}} des National Institute of Standards and Technology (NIST) der U.S.\ Regierung stellt mit ihrem CVE-System\textsuperscript{\ref{subsec:projektbericht-grundlagen-cve-cpe}} eine der primären Quellen für Schwachstell-Informationen für Forscher, Unternehmen und automatisierte Tools bereit.
Sobald eine neue Schwachstelle bekannt gegeben wird, nehmen sie diese in ihren Katalog an CVE mit auf, versehen sie mit CVSS- und Matching-Informationen über das CPE-System.
Damit ist die Schwachstelle für alle auf der Welt über eine API oder über ihr User Interface erreichbar und kann, falls ein Projekt betroffen ist, frühzeitig kontextualisiert bewertet werden.

\subsection{CVE / CPE} \label{subsec:projektbericht-grundlagen-cve-cpe}

\qt{CVE} ist ein von dem CVE Project eingeführtes System, das Schwachstellen in Software- und Hardwareprodukten eindeutig identifiziert und beschreibt.
Der Standard wird stark von der NVD des NIST unterstützt.
Im CVE-System bekommt jede Schwachstelle, die von einem Forscher oder einer Organisation gefunden und veröffentlicht wird, eine eindeutige ID von der Form \qt{CVE-YYYY-NNNN} zugewiesen, wobei \qt{YYYY} das Jahr der Veröffentlichung und \qt{NNNN} eine eindeutige, fortlaufende Nummer ist.
CVE definiert eine Schwachstelle als:

\begin{quote}
    A weakness in the computational logic (e.g., code) found in software and hardware components that, when exploited, results in a negative impact to confidentiality, integrity, or availability.
    Mitigation of the vulnerabilities in this context typically involves coding changes, but could also include specification changes or even specification deprecations (e.g., removal of affected protocols or functionality in their entirety).
    \cite{nvdVulnerabilityDefinition}
\end{quote}

Eine Schwachstelle ist also eine Schwäche, eine theoretische Angriffsfläche, die durch Ausnutzen negative Auswirkungen auf die Vertraulichkeit, Integrität oder Verfügbarkeit eines Systems hat.
Es soll kein CVE-Eintrag existieren, der keinen Einfluss auf die Vertraulichkeit, Integrität oder Verfügbarkeit eines Systems hat \cite{nvdVulnerabilityMetrics}.

Jeder CVE werden gewisse Metadaten zugeordnet, wie Beschreibungen, Referenzen und (potenziell mehrere) CVSS-Vektoren zur Bewertung des Schweregrads, aber auch \qt{CPE}s, die Aussagen über die betroffenen Produkte machen und automatisierte Zuordnungen erlauben.
Common Platform Enumeration (CPE) ist ein von der MITRE Corporation\footnote{\url{https://cpe.mitre.org/specification}} entwickeltes System, mit dem Soft- und Hardwareprodukte über den Hersteller, Produktnamen und Version eindeutig identifiziert werden können.
Die CPE Naming Specification Version 2.3 \cite[Seite 37, Kapitel 6.2]{NISTIR7695} definiert die Syntax eines CPE-Strings, in der mindestens \qt{part} (Komponenten-Typ), \qt{vendor} (Hersteller) und \qt{product} (Produkt) gesetzt sein müssen:

\code{cpe : 2.3 : part : vendor : product : version : update : edition}\newline\hphantom{\code{cpe}}\code{: language : sw_edition : target_sw : target_hw:other}

Auf dem NVD Dashboard\footnote{\url{https://nvd.nist.gov/general/nvd-dashboard}} ist gelistet, dass es am 08.03.2024 insgesamt 240899 eindeutige CVEs und 1261617 CPEs gibt.
Diese Zahl wächst exponentiell und zeigt, wie wichtig es ist, Schwachstellen automatisiert zu verarbeiten.

\subsection{CVSS} \label{subsec:projektbericht-grundlagen-cvss}

Das Common Vulnerability Scoring System (CVSS) wird in Version 4.0 \cite{CVSSv4.0Specification} vom FIRST\footnote{\url{https://www.first.org}} (Forum of Incident Response and Security Teams) als ein offener Standard für die Bewertung der Sicherheitsanfälligkeit von Software- und Hardwarekomponenten beschrieben.
Dieser Standard legt fest, wie sog\. CVSS-Vektoren, definiert als Sammlung von Schlüssel-Wert Paaren, zur möglichst objektiven Darstellung von Schwachstelleneigenschaften genutzt werden können.
Solche Vektoren können textuell als \code{CVSS:VERSION/KEY:VALUE/KEY:VALUE/...} formatiert werden, wobei jedes Wertepaar durch \qt{/} getrennt und mit \qt{CVSS:} gefolgt von der Vektor-Version beginnen muss.

Ein Beispiel für einen Basisvektor in Version 3.1, der einen mittleren Schweregrad (\qt{Medium}) mit einem Score von 6.0 darstellt, kann folgendermaßen aussehen: \code{CVSS:3.1/AV:L/AC:L/PR:H/UI:N/S:U/C:H/I:H/A:N}.
Allgemein reichen die Scores, die aus diesen Vektoren berechnet werden, von 0 bis 10 und ordnen die Schwachstellen in die Kategorien \qt{Low}, \qt{Medium}, \qt{High} und \qt{Critical} ein, wobei höhere Werte eine dringendere Bewertung erfordern.
Diese Berechnungen lassen sich leicht mit Online-Rechnern\footnote{\url{https://metaeffekt.com/security/cvss/calculator}} oder Software-Tools, verfügbar in allen populären Programmiersprachen, durchführen.

Die versionsabhängigen Metriken der Vektoren ordnen jeweils eine Charakteristik einer Schwachstelle einem Wert zu.
So steht beispielsweise \code{AV:L} für einen \qt{Attack Vector} mit dem Wert \qt{Adjacent Network}, was die räumliche Nähe beschreibt, die ein Angreifer benötigt, um die Schwachstelle auszunutzen.
Andere Werte wie \qt{Network} (aus dem Internet ausnutzbar, höchster Schweregrad), \qt{Local} oder \qt{Physical} (physikalischer Zugang benötigt, am weigsten schlimm) spezifizieren weitere Angriffsszenarien für diese Metrik.

Zusätzlich zu den allgemeinen, umgebungsunabhängigen Basismetriken, die von den Herausgebern der Schwachstelleninformationen festgelegt werden, gibt es noch die \qt{Temporal}/\qt{Threat}- und \qt{Environmental}-Metriken.
Diese Metriken reflektieren die spezifische Umgebung und den Kontext, in dem eine Schwachstelle existiert, und werden in betroffenen Projekten von den Anwendern kontextabhängig gesetzt.
Damit erlaubt es CVSS, eine initiale, kontextunabhängige Bewertung eines Systems, und durch Hinzufügen von Kontextinformationen auch eine kontextualisierte Sicht auf das System zu erhalten.
Einer Schwachstelle können von verschiedenen Parteien potenziell mehrere, unterschiedliche Vektoren zugeordnet werden, dieser Aspekt wird später wichtig.

\subsection{Automatisiertes Vulnerability Monitoring} \label{subsec:projektbericht-grundlagen-vulnerability-monitoring}

Das bei der {\metaeffekt} eingesetzte automatisierte Vulnerability Monitoring, hat das Ziel, für ein gegebenes Software-Inventar nicht nur die relevanten Schwachstellen (\qt{Vulnerabilities}) zu identifizieren, sondern auch zugehörige Ratgeber (\qt{Security Advisories}) als Hilfestellung zu der darauf folgenden manuellen Bewertung bereitzustellen.
Der Prozess, dargestellt in Grafik \ref{fig:vulnerability-monitoring-overview-figure}, in zwei Hauptphasen mit mehreren Unterschritten gegliedert, wird im Folgenden beschrieben.
Als Eingabe dient in jedem Fall ein Inventar gefüllt mit einer Liste an Software-Komponenten und deren Metadaten, wie Version, Quelle und Ökosystem-spezifischen Informationen.

Der erste Prozessschritt ist die Identifikation von Schwachstellen und wird getrennt voneinander auf alle Komponenten im Inventar angewandt.
Die in unserem Prozess verwendeten externen Schwachstellendatenbanken nutzen abstrahierte Darstellungen von Produkten und Software-Komponenten für ihre internen Verweise zwischen Produkten und Schwachstellen.
Beispielsweise verwendet die NVD CPEs (siehe Kapitel \ref{subsec:projektbericht-grundlagen-cve-cpe}), GitHub Security Advisories\footnote{\url{https://github.com/advisories}} folgen dem OSV-Schema\footnote{\url{https://osv.dev}} mit Ökosystem-abhängigen Informationen, und Microsoft (MSRC)\footnote{\url{https://msrc.microsoft.com/update-guide}} arbeitet mit numerischen Identifikatoren, die nur durch Download und Extraktion aller monatlichen Zusammenfassungen über ihre API zugänglich sind.
Die Zuordnung unserer realen Komponenten zu diesen abstrahierten Produkten stellt daher immer eine Herausforderung dar.
Unser Vulnerability Monitoring setzt verschiedene Algorithmen ein, um diese Zuordnungen so weit wie möglich automatisch vorzunehmen, was jedoch nicht immer fehlerfrei gelingt.
Falsch positive und falsch negative Ergebnisse erfordern dann eine manuelle Korrektur durch einen gepflegten Datensatz (\qt{Korrelationsdaten}), was einen erheblichen Zeitaufwand bedeutet.

Wenn dieser Schritt erfolgreich abgeschlossen ist, können Abfragen an die jeweiligen Datenbanken gestartet werden, um Schwachstellen durch Produkt- und Versionsabgleichen zu identifizieren.
Dieses Zwischeninventar mit Schwachstellinformationen kann bereits für Berichte genutzt werden, jedoch wird meistens ein weiterer Schritt dahinter geschaltet:
Weitere Abfragen an einen größeren Satz an Datenbanken werden gestartet, um zu den gefundenen Schwachstellen entsprechende Ratgeber zuzuordnen.
Diese Abfragen sind in der Regel einfacher, da die Ratgeber-Einträge meistens direkt auf die betroffenen Schwachstellen-IDs Bezug nehmen.
Das Ergebnisinventar kann dann als Grundlage für die Erstellung von Reports und Statistiken benutzt werden.

\begin{figure}[htbp] % here, top, bottom, separate page
    \centering
    \includegraphics[width=1\textwidth, keepaspectratio]{res/grafiken/vulnerability-monitoring-overview}
    \caption{Schwachstellsuche mit einem Software-Inventar}
    \label{fig:vulnerability-monitoring-overview-figure}
\end{figure}

\subsection{Schwachstellen-Priorisierung durch CVSS} \label{subsec:projektbericht-grundlagen-vulnerability-assessment}

Wenn eine Liste an Schwachstellen durch den Prozess in \ref{subsec:projektbericht-grundlagen-vulnerability-monitoring} als Minimalanforderung in einem Projekt für ein Inventar erfasst wurde, muss nun ein zweiter Schritt folgen:
Die manuelle Bewertung der Schwachstellen, indem eine Einschätzung, ein Status und erforderliche Maßnahmen angegeben werden.
Jedoch kann es sowohl bei großen, aber auch schon bei kleinen Projekten mit kritischen Abhängigkeiten zu einer großen Menge an Schwachstellen kommen (1000 und aufwärts), die oftmals nicht alle sofort von einer Person oder einem Bewertungs-Team durchgearbeitet werden können.
Es werden also objektive und automatisierbare Faktoren benötigt, anhand denen die Schwachstellen möglichst ohne weiteres Zutun in eine priorisierte Liste eingeordnet werden können, um die kritischsten zuerst abarbeiten zu können.

Diese Faktoren sind Projekt- und Umgebungsspezifisch und können stark variieren.
In diesem Kapitel wird nur CVSS als Priorisierungsfaktor betrachtet, wie sie in der CVSS-Spezifikation beschrieben werden \code{[TODO: CITATION NEEDED]}.
In Kapitel \ref{subsec:projektbericht-grundlagen-cvss} wurde bereits auf die Metrik-Gruppen eingegangen.
Da jede Schwachstelle aus dem CVE-System und den meisten anderen Systemen mindestens einen CVSS-Vektor von den Herausgebern angehängt bekommen hat, kann bereits ohne großen Mehraufwand eine nach dem daraus errechneten Basis-Score sortierte Liste als initiale Reihenfolge dienen.

Vor allem die Environmental-Metrikgruppe kann nun zur Kontextualisierung von Schwachstellen in einem Projekt verwendet werden.
Das Beispiel bezüglich des \qt{Attack Vector}s in \ref{subsec:projektbericht-grundlagen-cvss}, bei dem \code{AV} auf \code{L} kann hier weiter getrieben werden:
Wenn eine aus einem lokalen Netzwerk ausnutzbare Schwachstelle in einer Applikation vorhanden ist, die keinerlei Netzwerkzugang hat, dann kann diese auch nicht aus einem Netzwerk ausgenutzt werden, sondern eventuell nur durch physikalischen Zugriff.
Allgemein kann dann also eine globale, Inventar-weite CVSS-Vektoränderung von \code{AV:P} angewendet werden, aus der eine neue, kontextualisierte Liste an Schwachstellen hervorgeht, die nun akkurater die tatsächlichen Risiken eines Systems darstellen.
Weitere Metriken wie das Risiko eines Verlusts an Konfidentialität, Integrität oder Verfügbarkeit können für weitere Anpassungen verwendet werden.


% Lösungswege


\section{Lösungsweg} \label{sec:projektbericht-loesungsweg}

Dieses Kapitel wird in drei Unterkapitel gegliedert, die sich jeweils auf eine der drei Problemstellungen aus Kapitel \ref{sec:projektbericht-projektziel} beziehen.

\subsection{Implementierung von CVSS 4.0} \label{subsec:projektbericht-loesungsweg-cvss-4-implementierung}

\subsection{CVSS-Vektor Quellenmanagement} \label{subsec:projektbericht-loesungsweg-cvss-source-management}

Um das Problem von multiplen CVSS-Vektoren pro Schwachstelle zu lösen, muss zunächst ein System entworfen werden, mit dem eine Quelle eines CVSS-Vektors eindeutig nach Aufnahme weiterhin identifiziert werden kann.
Einige Anforderungen an das Format sind, dass sie in einen einfachen menschenlesbaren String serialisiert und von diesem wieder deserialisiert werden kann, aber auch die Eindeutigkeit der Quellen so hoch ist, dass sich zwei in irgendeiner Art unterschiedliche Quellen auch in der textuellen Repräsentation unterscheiden und aus einem externen Datensatz eindeutig weitere Metadaten hinzugefügt werden können.
Um dieses Format zu bilden, wurden unter anderem die folgenden Quellen berücksichtigt:

\begin{itemize}
    \item NVD:
    Organisationen und Institutionen können, wenn sie als sogenannte CVE Naming Authorities (CNA) registriert sind, für CVEs beliebige CVSS-Vektoren veröffentlichen.
    Eine öffentlich verfügbare Liste an CNAs ist online [TODO: URL] mit Metadaten verfügbar.
    \item MSRC:
    Sie stellen für jede für Microsoft relevante Schwachstelle und pro betroffene (numerische) Produkt-Id einen CVSS-Vektor zur Verfügung.
    \item CERT-SEI und GHSA: Stellen jeweils einen Vektor pro referenzierter Schwachstelle zur Verfügung.
\end{itemize}

Aus diesen Anforderungen wurde das in Listing \ref{lst:cvss-source-format} abgebildete Format entworfen, wobei jede der drei Optionen gültig ist.
Eine Quelle besteht immer aus der Vektor-Version (CvssVersion) und der Entität, die den Vektor auf ihrer Platform zur Verfügung stellen (HostingEntity).
Eine Entität kann sich beispielsweise auf die NVD, MSRC, GHSA, usw\ldots beziehen.
Wenn die herausgebende Entität (HostingEntity) und die Entität, die den Vektor initial zur Verfügung gestellt hat (IssuingEntity), sich unterscheiden, müssen diese beide mit angegeben werden.
In gewissen externen Schwachstelldatenbanken werden die Issuing Entities unter einem gewissen Rollennamen geführt, wie die CNAs bei der NVD\.
In diesem Fall muss auch die IssuerRole in der Mitte angegeben werden.

Zum Ablegen dieser Werte getrennt voneinander kann eine beliebige Datenstruktur genutzt werden, falls die Quelle jedoch als String serialisiert werden soll, muss sie auf eine besondere Weise formatiert werden:
In den Entities und im Rollenbezeichner müssen zunächst alle Bindestriche (\qt{-}) mit einem \qt{\\}-Symbol escaped werden und alle restlichen \qt{\\} durch \qt{\\\\}.
Damit ist es möglich, die in Listing \ref{lst:cvss-source-format} angegeben Bindestriche als Trennzeichen für die einzelnen Elemente zu verwenden.
Da eine CVSS-Version kein Leerzeichen enthält, kann hier einfach das erste Leerzeichen als Trennung zwischen Version und den restlichen Elementen verwendet werden.

Das System muss allerdings auch kombinierte Vektoren und deren kombinierte Quellen unterstützen.
Hierfür wird eine Erweiterung des Formats eingeführt, bei dem die Zeichenfolge \qt{ + } als Trennung für mehrere Quellen dienen kann, wie in Listing \ref{lst:cvss-source-format-combined} zu sehen.
Natürlich macht es dies nötig, auch Plus-Zeichen (\qt{+}) in den Quellen mit \qt{\\+} zu ersetzen.
Die Vektor-Version muss dennoch nur einmal angegeben werden, da es unmöglich ist, Vektoren unterschiedlicher Version miteinander zu kombinieren.
Laut Format-Definition müssen die Quellen ist die Reihenfolge der Quellen signifikant, da die erste Quelle den ursprünglichen Vektor repräsentiert und die darauf folgenden, in gelisteter Reihenfolge, auf diesen angewandt wurden,

\begin{lstlisting}[language={}, label={lst:cvss-source-format}, caption={CVSS Sources Format}]
CvssVersion HostingEntity
CvssVersion HostingEntity-IssuingEntity
CvssVersion HostingEntity-IssuerRole-IssuingEntity
\end{lstlisting}

\begin{lstlisting}[language={}, label={lst:cvss-source-format-combined}, caption={CVSS Sources Format}]
CvssVersion HostingEntity-IssuerRole-IssuingEntity + HostingEntity-IssuerRole-IssuingEntity + ...
\end{lstlisting}

Einige Beispiele für Quellen sind in Listing \ref{lst:cvss-source-format-examples} zu finden.
Dieses Format wird in Kapitel \ref{subsec:projektbericht-loesungsweg-cvss-selection} weiter verwendet.

\begin{lstlisting}[language={}, label={lst:cvss-source-format-examples}, caption={CVSS Sources Format}]
CVSS:3.1 Microsoft Corporation
CVSS:2.0 Assessment-all
CVSS:2.0 NVD-CNA-NVD
CVSS:3.1 NVD-CNA-Microsoft Corporation
CVSS:4.0 Assessment-lower
\end{lstlisting}

Um den einzelnen Entitäten Metadaten, wie die E-Mail Adressen der zuständigen Institutionen und Organisationen oder URLs zu deren Homepages oder Datenquellen, zuordnen zu können, muss ein Datensatz aller bekannter Datenquellen aufgebaut werden, der diesen zugeordnet werden kann.
Hierfür wurde ein Prozess entworfen, der automatisiert die Liste aller CNAs der NVD\footnote{\url{https://nvd.nist.gov/vuln/cvmap/search}} und einem Mirror davon, gehostet auf CVE.org\footnote{\url{https://www.cve.org/PartnerInformation/ListofPartners}}, abfragt und daraus eine JSON-Datei generiert, in der diese Informationen abgelegt und später wieder eingelesen werden können.
Zu einigen ausgewählten Entitäten wird ein zusätzlicher, manueller Datensatz gepflegt, der weitere Informationen zu diesen trägt.

\subsection{CVSS-Vektor Selektion} \label{subsec:projektbericht-loesungsweg-cvss-selection}

Mit dem Format aus Kapitel \ref{subsec:projektbericht-loesungsweg-cvss-source-management} werden ab Generation 3 des Vulnerability Monitorings der {\metaeffekt} ALLE Vektoren zu den gefundenen Schwachstellen mit in das Inventar der Komponenten aufgenommen.
Generation 2 dieses Systems hat diese Multiplizität der Vektoren einfach übergangen, und als Nebeneffekt des zuständigen Codes ein \qt{First-Come-First-Serve}-Verfahren für die Vektoren angewendet, was eine bewusste Entscheidung eines interessierten Bewertungsteams von Anfang an ausschloss.
Mit diesem neuen System, bei dem zunächst alle Vektoren aggregiert und erst in den Schritten ausgewertet werden, in denen sie visualisiert werden müssen, kann also zu jedem Zeitpunkt die genaue Quelle eines Vektors nachvollzogen werden.
Die Funktionsweise der CVSS-Vektor-Selektion wird in diesem Kapitel erklärt.

Konzeptionell ist vor Beginn erneut zu betonen, dass die Auswahl der, von den vorhergegangenen Prozessschritten angesammelten, Vektoren wirklich erst ab den Schritten geschieht, in denen sie auf egal welcher Art und Weise dargestellt oder für weitere Berechnungen benötigt werden.
Die Selektion bezieht sich immer auf die Vektoren einer einzigen, vollständigen Schwachstelle.
Dieser Prozess wird in zwei Teile geteilt, da für einige Use-Cases bereits das Zwischenergebnis von Interesse ist.

Ein Schritt, der vor der Auswahl stattfinden muss, ist es, erst die relevanten Vektoren einer Schwachstelle zusammenzusammeln.
Hierfür gibt es einige Fälle die beachtet werden müssen:
Natürlich werden alle Vektoren, die direkt auf der Schwachstelle sind, direkt hinzugefügt.
Einige Vektoren werden nicht von der Schwachstelle selbst, sondern von referenzierten Security Advisories beigetragen, also müssen auch diese eingesammelt werden.
Doch nicht immer sind alle Vektoren anwendbar, zum Beispiel bei Vektoren von Microsoft gibt es Bedingungen, wann ein Vektor anwendbar ist.
Diese Bedingungen müssen hier erst ausgewertet werden, wodurch Vektoren eventuell herausgefiltert werden.
Zudem kann es sein, dass ein projektinternes Bewertungsteam eigene Vektoren für eine Schwachstelle vergeben hat, diese müssen natürlich auch berücksichtigt werden.

Nachdem nun alle Vektoren angesammelt wurden, kann die Selektion losgehen.
Die beiden Schritte, die diese Selektion ausmachen, sind die Folgenden:

... siehe enumerate unten ... TODO

---

Relevant ist auch noch, dass zu einer Schwachstelle pro Vektor-Version optional noch eine Bewertung angehängt werden kann.
Dies führt zu den initialen (ab sofort als \qt{initial} bezeichnet) Vektoren auch noch modifizierte Vektoren ein (\qt{context}), die getrennt voneinander betrachtet werden müssen, da beide unterschiedliche Informationen vermitteln.

Das Ziel dieses Prozesses ist es, zwei Vektoren aus der potenziell großen Menge an Vektoren unterschiedlicher Versionen zu erhalten, einen für initial und einen für context.
Der Prozess muss komplett modular und konfigurierbar sein, er muss es erlauben zwischen verschiedenen Quellen priorisiert zu unterscheiden, Vektoren unterschiedlicher Quellen miteinander (potenziell bedingt) zu kombinieren und muss Logik enthalten, in bestimmten Fällen auch keinen Vektor zurückzugeben.
Da jeder Selektor allerdings pro CVSS-Version einen Vektor zurückgibt und nicht nur einen, da er auf jeder Version unabhängig voneinander operiert, muss ein weiterer Schritt dahintergeschaltet werden, der aus diesen dann einen finalen auswählt.
Allgemein soll der Prozess also folgendermaßen sowohl für den effektiven initial-, als auch für den context-Vektor ablaufen:

\begin{enumerate}
    \item Ansammeln aller CVSS-Vektoren einer Schwachstelle.
    Dies inkludiert zunächst alle Vektoren, die direkt von der Schwachstelle selbst referenziert werden, also von den Schwachstellen-Herausgebern angehängt wurden.
    Dann werden alle Security Advisories ausgewertet, die von der Schwachstelle referenziert werden, um deren Vektoren auch hinzuzufügen.
    Hier ist zu beachten, dass manche Vektoren nur bedingt hinzugefügt werden können, wie bei den Microsoft-Vektoren, die nur dann anwendbar sind, wenn eine entsprechende Produkt-Id auf einer Software-Komponente vorhanden ist.
    \item Nutze einen Selektor, um aus allen aggregierten Vektoren eine reduzierte Menge zu bilden, die pro Version nur einen Vektor enthält.
    \item Eine weitere Regel wird dann benutzt, um zwischen den unterschiedlichen Versionen zu entscheiden und nur einen zurückzugeben, der dann für weitere Berechnungen verwendet werden kann.
\end{enumerate}

Um die Einordnung des Themas zu vereinfachen, wird dieser Prozess anhand eines Beispiels mit dem Betriebssystem \qt{Windows 10 Pro} mit frei erdachten, jedoch realistischen Schwachstellen, Security Advisories und Vektor-Quellen durchgeführt.
Da das System eine Eingabe braucht, wird zunächst, begleitet durch Schaubild \ref{fig:cvss-selection-process-collection}, ein angereichertes Inventar gebaut.
Dieses Betriebssystem, als Komponente, durchläuft die Inventory Enrichment Pipeline, wie in Kapitel \ref{subsec:projektbericht-grundlagen-vulnerability-monitoring} beschrieben, und bekommt damit die gefundenen Schwachstellen und Security Advisories angehängt.
Diese haben jeweils wieder potentiell mehrere CVSS-Vektoren, die sie wiederrum der Komponente beitragen können.
Zudem werden alle Vektor-Modifikationen von manuellen Bewertungen als \qt{Assessment}-Vektoren ebenfalls mitaufgenommen.
Diese Ergebnisliste an Vektoren dient nun dem eigentlichen Auswahlprozess als Eingabe und wird im Folgenden erklärt.

\begin{figure}[htbp] % here, top, bottom, separate page
    \centering
    \includegraphics[width=1\textwidth, keepaspectratio]{res/grafiken/cvss-selection-process-collection}
    \caption{Initialer Zuordnungs- und Sammelungsschritt für Schwachstelldaten und CVSS-Vektoren}
    \label{fig:cvss-selection-process-collection}
\end{figure}

Die nun gesammelten Vektoren werden nun den Selektionsprozess durchlaufen, wie auch bei einem realen Fall, in Schaubild \ref{fig:cvss-selection-process-selection} zu verfolgen.

\begin{enumerate}
    \item Zunächst müssen alle Vektoren, die im vorherigen Schritt gesammelt wurden, auf Anwendbarkeit gefiltert.
    Hierbei fällt einer der Vektoren von Microsoft weg, da er nicht mit der Produkt-Id aus der Komponente übereinstimmt.
    Die restlichen Vektoren werden einfach übernommen, da sie entweder keine Bedingung enthalten oder die Bedingung zutrifft.
    \item Danach werden die beiden Selektoren ausgewertet, mit dem Ziel, je einen (eventuell kombinierten) Vektor pro Version zu erhalten.
    TODO
\end{enumerate}

\begin{figure}[htbp] % here, top, bottom, separate page
    \centering
    \includegraphics[width=1\textwidth, keepaspectratio]{res/grafiken/cvss-selection-process-selection}
    \caption{Selektionsvorgang für CVSS-Vektoren zur Reduktion auf einen pro Selektor}
    \label{fig:cvss-selection-process-selection}
\end{figure}

\subsection{Universal CVSS Calculator} \label{subsec:projektbericht-loesungsweg-typescript-cvss-online-calculator}


    %! Author = Yan Wittmann

\chapter{Ergebnisse} \label{ch:ergebnisse}

In dem Praxissemester wurden unter anderem die folgenden Haupt-Ergebnisse erzielt:

\begin{itemize}
    \item Implementierung des CVSS 4.0 Standards in die vorhandene Softwarelösung und Integration in den Daten-Mirror und die Reporting-Tools.
    \item Überarbeitung des Datenmodells zur Unterstützung von mehreren CVSS-Vektoren je Schwachstelle, verbessertem Tracking der Quellen von Schwachstellen und der Generalisierung der Datenzugriffe, um Redundanzen in den Implementierungen zu vermeiden.
    \item Implementierung der CVSS-Versionen 2.0, 3.0, 3.1 und 4.0 in TypeScript.
    \item Entwickeln eines online-CVSS-Rechners, der aus den eigenen Reporting-Tools referenziert wird und bereits bei verschiedenen Events vorgestellt wurde.
    \item Integration der durchgeführten Änderungen in die entsprechenden Kundenprojekte an den Kundensystemen und im konstanten Dialog mit den Zuständigen.
    \item Besuchen des Arbeitskreises OpenSource des {\bitkom} und halten eines Vortrages.
    \item Besuchen des OpenSource Forums des {\bitkom} in Erfurt.
    \item Besuchen des zweitägigen CSAF-Workshops in München im ISH (Information Security Hub), durchgeführt vom BSI\@.
\end{itemize}

Die folgenden konkreten öffentliche Ergebnisartefakte wurden dabei unter anderem erzeugt (mit Links hinterlegt):

\begin{itemize}[noitemsep]
    \item \href{https://www.metaeffekt.com/security/cvss/calculator}{Universal CVSS Calculator}
    \item \href{https://github.com/org-metaeffekt/metaeffekt-universal-cvss-calculator}{GitHub Repository der TypeScript Implementierung}
    \item \href{https://youtu.be/R2S0_6-NQGQ?si=d7zpxbJ7P4R26nRb&t=2801}{Webinar zum CVSS Calculator}
    \item \href{https://mvnrepository.com/artifact/com.metaeffekt.artifact.analysis/ae-artifact-analysis}{Weiterführung des aritfact-analysis Projektes}
    \item \href{https://github.com/org-metaeffekt/metaeffekt-core}{Weiterführung des Core Projektes}
\end{itemize}


    %! Author = Yan Wittmann

\chapter{Ausblick} \label{ch:ausblick}

Meine Mitarbeit bei der {\metaeffekt} wird nach dem Praxissemester weiterhin fortgeführt.
In diesem Zusammenhang werde ich mit den Systemen, Kunden und Kollegen auch weiterhin in Kontakt bleiben und die Entwicklungen begleiten.

Konkret werden die folgenden Punkte in den nächsten Monaten relevant:

\begin{itemize}
    \item Integration von KEV (Known Exploited Vulnerabilities) in die bestehenden Systeme.
    \item Integration von CSAF (Common Security Advisory Framework) als Advisory-Quelle und Support für Export von CSAF-Dateien aus gescannten Software-Inventaren.
    \item Besuch des KIT, um mit einem Professor über das Thema Vulnerability Chaining zu sprechen, welches in nächster Zeit bei uns mit einem eigenen Konzept integriert werden soll.
    \item Reimplementierung des VAD (Vulnerability Assessment Dashboards) mit TypeScript.
    \item Erstellen eines \qt{Overview Reports} in HTML und TypeScript, für eine Übersicht über mehrere Kontexte.
    \item Unsere Systeme mehr als eine Art Tooling-Chain darstellen, um den Einstieg in die Verwendung unserer Tools zu erleichtern.
    \item Präsentieren unseres Toolings auf der großen Bühne beim Open Source Forum des {\bitkom} in Erfurt im September.
    \item Erstellung eines Test-Frameworks für das Vulnerability Monitoring.
\end{itemize}


    \newpage

% Listen wenn überhaupt ans Ende und nicht an den Anfang.
% Meist ist das aber unnötig.
    \listoffigures % Liste der Abbildungen
%\begingroup % aahh nicht noch ein pagebreak
%\let\clearpage\relax %
    % \listoftables % Liste der Tabellen
%\endgroup

% Glossar kommt auch ans Ende
%\glsaddall % das fügt alle Glossar-Einträge ein
%\printglossaries % nicht vergessen "makeglossaries praksem" aufzurufen
%\gls{Computer}
%\newpage

    \addcontentsline{toc}{chapter}{Literaturverzeichnis}
    %\bibliographystyle{plain}
    %\bibliography{praksem}
    \printbibliography
% \bibliography{praksem,online} # wenn man zwei Dateien hätte

% Das wäre die Alternative mit geteilten Quellen (preamble muss auch
% angepasst werden) und die Literatur muss in die Datei praksem.bib
% und die Online-Quellen müssen in die Datei online.bib.
%\begin{btSect}{praksem} % mit bibtopic Quellen trennen
%\section*{Literaturverzeichnis}
%\addcontentsline{toc}{chapter}{Literaturverzeichnis}
%\btPrintCited
%\end{btSect}
%\begin{btSect}{online}
%\section*{Online-Quellen}
%\addcontentsline{toc}{chapter}{Online-Quellen}
%\btPrintCited
%\end{btSect}
% dann ab und zu "bibtex praksem1" und "bibtex praksem2" aufrufen

\end{document}
;;; Local Variables:
;;; ispell-local-dictionary: "de_DE-neu"
;;; End:
