%! Author = Yan Wittmann


\chapter{Tätigkeitsbeschreibung} \label{ch:wochenberichte}


\section{Erste Woche - Einarbeitung in CVSS 4.0} \label{sec:bericht-wo-1}

Der erste Arbeitstag meines Praktikums bei \metaeffekt fiel mit dem Ende der Sommerpause für viele meiner Kollegen zusammen.
Da ich schon seit einiger Zeit im Unternehmen arbeite, war eine ausführliche Einführung für mich nicht notwendig.
Ich habe bereits meine eigenen Aufgabenbereiche, um die ich mich kümmern muss.
Allerdings war auch für mich der Montag der erste Tag nach dem Urlaub, daher musste ich mich erst wieder orientieren und herausfinden, welche Aufgaben als nächstes anstehen.

Ich verbrachte den Montag damit, einige Fehler zu korrigieren, die während der Urlaubszeit in unserem System in Kundenprojekten aufgetreten waren.
Ich führte auch Gespräche mit Kollegen, um zu klären, welche Projekte als nächstes anstehen.
Ein wichtiges Thema war die anstehende Veröffentlichung des CVSS 4.0-Standards, die für den 31. Oktober 2023\footnote{\url{https://www.first.org/cvss/v4-0/}} geplant ist.
Es wurde entschieden, dass ich diesen Standard in den kommenden Tagen in unser System einpflegen sollte.
Unsere Software kann bereits die Score-halfBerechnungen für die CVSS-Versionen 2 und 3 durchführen, und wir möchten in der Lage sein, auch die neuen CVSS 4.0-Vektoren zu interpretieren und zu berechnen, sobald diese in den öffentlichen Datenbanken auftauchen.

Mit meinem Vorgesetzten und Betreuer für das Praktikum, Karsten Klein, habe ich vereinbart, während meines Vollzeitsemesters bei \metaeffekt tägliche Meetings abzuhalten.
Diese haben bisher jeden Tag durchgeführt und sowohl über berufliche als auch private Themen gesprochen.
Ich werde die genauen Inhalte dieser Meetings hier nicht immer aufführen, da sie meist über die bereits behandelten Themen des Tages gehen.

Am Dienstag startete ich mit der Recherche zu den theoretischen Grundlagen und den Einzelheiten von CVSS 4.0.
Ich stellte schnell fest, dass es mehr Unterschiede als Gemeinsamkeiten zu den vorherigen Versionen CVSS 2 und 3 gibt, insbesondere in Bezug auf Berechnung, Implementierung und Theorie.
Den Rest des Tages verbrachte ich damit, die noch unfertige Dokumentation und Beispiele zu studieren und das Konzept des neuen Standards zu verstehen.
Dies erwies sich als komplexer als zunächst angenommen.

Am Mittwoch begann ich mit der Implementierung der CVSS 4.0-Berechnung.
Nach weiterer Recherche fand ich den Quellcode einer Beispiel-Implementierung eines CVSS 4.0-Rechners in JavaScript auf dem offiziellen RedHat GitHub-Repository\footnote{\url{https://github.com/RedHatProductSecurity/cvss-v4-calculator}}, die stark an der Entwicklung des Standards beteiligt waren.
Dieses Beispiel hat mir geholfen, das Grundgerüst für die Implementierung in unserem Java-System vorzubereiten, das bereits Klassen für CVSS 2 und 3 enthält.

Der Donnerstag startete mit einer Fehlermeldung in einem unserer Kundenprojekte.
Es handelte sich um eine „OutOfMemoryError“-Exception, die auftrat, wenn unser Schwachstellen-Report aus einer großen Menge an Daten generiert werden sollte.
Ich löste das Problem durch einen File Appender, der den HTML-String des Reports direkt in eine Datei schreibt, anstatt ihn wie zuvor im Speicher zu halten und habe so die kurzzeitige Verdoppelung den Hauptspeicher-Bedarf entfernt.

Der Hauptteil des Tages war jedoch weiterhin dem CVSS 4.0-Standard gewidmet.
Der Code der Referenz-Implementierung in JavaScript war komplizierter zu verstehen als erwartet.
Der Code ist nicht sehr leserlich geschrieben, wenig modular und enthält unnötige Operationen, die teilweise später ignoriert oder rückgängig gemacht werden.
Ich fand auch einige Abweichungen zwischen der Spezifikation und der Implementierung und meldete diese zusammen mit einigen inhaltlichen Fragen in einem Issue auf dem entsprechenden GitHub-Repository.
Ich beendete den Tag mit einer teilweise funktionierenden Implementierung.

Der Freitag sollte der Tag sein, an dem ich die Implementierung abschließe.
Ich erhielt eine Antwort auf meine GitHub-Fragen: Die Spezifikation ist veraltet, aber die Implementierung ist korrekt.
Mithilfe dieser Informationen und einigem weiteren herumexperimentieren konnte ich die Berechnungen fertigstellen.
Ich validierte sie, indem ich einen großen Datensatz an CVSS-Vektoren durch die Referenz-Implementierung und meine eigene Implementierung laufen ließ und die Ergebnisse verglich.

Am Freitagmittag findet bei \metaeffekt ein wöchentliches Meeting statt, das als „Weekly“ bezeichnet wird.
In diesem Meeting berichtet jedes Teammitglied über seine Fortschritte, Herausforderungen und nächsten Schritte.
Ich berichtete über meine Erfahrungen mit der Implementierung von CVSS 4.0 und hörte, was die anderen Teammitglieder in dieser Woche erreicht hatten.

Da ich am Freitagnachmittag private Verpflichtungen hatte, begann ich den Tag etwas früher und beendete meine erste Praktikumswoche nach dem „Weekly“-Meeting.